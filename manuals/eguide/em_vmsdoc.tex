\module{exportbck.com}%
\noindent Delete junk from miriad directories
\newline \ 
\newline \abox{File:} \$MIR/src/scripts/vms/exportbck.com
\newline \abox{Keywords:} system operation
\newline{\tenpoint\newline
Delete junk from miriad directories
\par}
\module{mirbug.com}%
\noindent Send mail about bugs to appropriate person
\newline \ 
\newline \abox{File:} \$MIR/src/scripts/vms/mirbug.com
\newline \abox{Responsible:} Bart Wakker
\newline \abox{Keywords:} user utility
\newline{\tenpoint\newline
Usage: @mir\_bug [selection] [taskname] [file]

Mirbug allows the user to send mail to the appropriate persons.

[selection] should be one of ``bug'' or ``feedback''. This influences
the information prepended to a report file. ``mirbug'' is an alias
for ``mir.bug.csh bug''; `` mirfeedback'' is an alias for ``mir.bugcsh.
feedback''

[taskname] should be one of the miriad tasks/tools/scripts/subs.
It is used to find out who is responsible for the code and then the
report is mailed to that person. If it is not given mirbug complains,
but continues.

[file] is the name of an optional, previously created, file
containing the report. Some extra information is prepended and an
editor is started. The default editor is vi on unix and edt on VMS.
However, by including 'setenv EDITOR mem' in the .cshrc file (unix)
or 'editor:==mem' in the login.com file (VMS) it is possible to
specify to use, for instance, mem (or any other editor).

[file] must be the third argument.

If mirbug.com is aborted and a template was already generated, it is
saved to allow extra editing, otherwise it is deleted.
The generated mail message is sent to the responsible person and to
a central address at the site of the programmer. If no responsible
person can be found it is sent to all the central addresses. A copy
is also sent to the user. 
\par}
\module{mirbug}%
\noindent Send mail about bugs to appropriate person
\newline \ 
\newline \abox{File:} \$MIR/src/scripts/vms/mirbug.com
\newline \abox{Responsible:} Bart Wakker
\newline \abox{Keywords:} user utility
\newline{\tenpoint\newline
Usage: mirbug [taskname] [file]

Mirbug allows the user to send mail to the appropriate persons.

[taskname] should be one of the miriad tasks/tools/scripts/subs.
It is used to find out who is responsible for the code and then the
report is mailed to that person. If it is not given mirbug complains,
but continues.

[file] is the name of an optional, previously created, file
containing the report. Some extra information is prepended and an
editor is started. The default editor is vi on unix and edt on VMS.
However, by including 'setenv EDITOR mem' in the .cshrc file (unix)
or 'editor:==mem' in the login.com file (VMS) it is possible to
specify to use, for instance, mem (or any other editor).

[file] must be the third argument.

If mir.bug.csh is aborted and a template was already generated, it is
saved to allow extra editing, otherwise it is deleted.
The generated mail message is sent to the responsible person and to
a central address at the site of the programmer. If no responsible
person can be found it is sent to all the central addresses. A copy
is also sent to the user. 
\par}
\module{mirfeedback}%
\noindent Send feedback to the appropriate person
\newline \ 
\newline \abox{File:} \$MIR/src/scripts/vms/mirbug.com
\newline \abox{Responsible:} Bart Wakker
\newline \abox{Keywords:} user utility
\newline{\tenpoint\newline
Usage: mirfeedback [taskname] [file]

This is a way to send feedback to a programmer. For a description
of how it works, see ``mirbug''. The difference between these two is
in the prepended information.
\par}
\module{mirhelp.com}%
\noindent Online help utility; prints tasks or routine documentation
\newline \ 
\newline \abox{File:} \$MIR/src/scripts/vms/mirhelp.com
\newline \abox{Responsible:} Bart Wakker
\newline \abox{Keywords:} programmer tool
\newline{\tenpoint\newline
online help utility; prints tasks or routine documentation

Usage: mirhelp topic [-s][-n]
where topic is a task or routine name occuring in
MIR:[.doc.prog], MIR:[.doc.subs] or MIR:[.src.subs].
All tasks/routines are found whose name matches all or part of the
input topic.
The use of existing preformatted files is given precedence over
searching through the subroutine files.
Using the -s flag forces mirhelp to search through subroutine files
instead of using preformatted documentation files.
Option -n makes mirhelp for the full name of the specified file.
\par}
\module{MIRRC.com}%
\noindent MIRRC file for VAX
\newline \ 
\newline \abox{File:} \$MIR/src/scripts/vms/MIRRC.com
\newline \abox{Keywords:} system operation
\newline{\tenpoint\newline
Miriad's MIRRC file.
\par}
\module{mkmanuals.com}%
\noindent Construct file to make miriad manual
\newline \ 
\newline \abox{File:} \$MIR/src/scripts/vms/mkmanuals.com
\newline \abox{Keywords:} system operation
\newline{\tenpoint\newline
This is a command procedure to construct various files which form
part of the Miriad manual. The output from each of these steps is a
.latex, .toc or .idx file, which is later included within the user
or programmer manual. It takes one optional parameter, which can be:
{\eightpoint\begintt
  CONTENTS   Makes the table-of-contents of the user manual.
  TASKS      Makes the chapter of the user manual, describing tasks.
   SUBS      Makes the appendix of the programmers manual, describing
             subroutines.
   INDEX     Makes the index of the programmers manual.
\endtt}
If the parameter is missing, all parts are constructed.
\par}
\module{note.com}%
\noindent Adds a note to the updates.dat file.
\newline \ 
\newline \abox{File:} \$MIR/src/scripts/vms/note.com
\newline \abox{Keywords:} system operation
\newline{\tenpoint\newline
This adds a note to the updates.dat file.
Inputs:
{\eightpoint\begintt
  p1   Your comments.
\endtt}
\par}
\module{oneoff.com}%
\noindent Perform selective updates
\newline \ 
\newline \abox{File:} \$MIR/src/scripts/vms/oneoff.com
\newline{\tenpoint\newline
This is a command procedure to perform selective updates of particular
files. No check is made that they are in the correct hierarchy, nor
of their modification times. 
Inputs:
{\eightpoint\begintt
 p1          Name of machine to update.
 p2          List of files to update.
\endtt}
\par}
\module{rmdir.com}%
\noindent Command procedure to remove a directory tree.
\newline \ 
\newline \abox{File:} \$MIR/src/scripts/vms/rmdir.com
\newline \abox{Keywords:} system operation
\newline{\tenpoint\newline
Command procedure to remove a directory tree.
\par}
\module{update.com}%
\noindent Updates on vax
\newline \ 
\newline \abox{File:} \$MIR/src/scripts/vms/update.com
\newline \abox{Keywords:} system operation
\newline{\tenpoint\newline
This determines which files are newer in Werongs microVAX directories
than on some other machine.
Inputs:
{\eightpoint\begintt
 p1    Machine name.
\endtt}
\par}
\module{vsubmit.com}%
\noindent Submit vsubproc.com
\newline \ 
\newline \abox{File:} \$MIR/src/scripts/vms/vsubmit.com
\newline \abox{Keywords:} system operation
\newline{\tenpoint\newline
This file is spawned by vmenu to submit the vsubproc.com file
IN: P1 = the batchque name   P2 = the miriad command
OUT: para1 = current directory  para2 = miriad command
The batch log file will be named sys\$login:vsubproc.log
\par}
\module{vsubproc.com}%
\noindent Generic subprocess command file for vmenu
\newline \ 
\newline \abox{File:} \$MIR/src/scripts/vms/vsubproc.com
\newline \abox{Keywords:} system operation
\newline{\tenpoint\newline
generic subprocess command file for vmenu (this is submitted by
vsubmit.com)

{\eightpoint\begintt
 P1 = current directory      P2 = the miriad command to be run
\endtt}
\par}
