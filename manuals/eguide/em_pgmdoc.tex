\module{avmaths}%
\noindent Operate on cube with averaged plane from cube
\newline \ 
\newline \abox{Responsible:} Neil Killeen
\newline \abox{Keywords:} analysis
\newline{\tenpoint\newline
AVMATHS averages designated planes from a cube, and then 
performs some mathematical operation on the cube with the 
averaged plane. Currently subtraction and optical depth 
calculations only have been coded.  Undefined output pixels 
are blanked.
\keyword{\bf IN}
The input image. No default.
\keyword{\bf OUT}
The output image. No default.
\keyword{\bf REGION}
Specify the channels to average with a command of the region
of interest format, eg.
{\eightpoint\begintt
  region=image(1,5),image(120,128)
\endtt}
This would average channels 1:5 and 120:128 from the
cube.  
\keyword{\bf OPTIONS}
Should be one of
'sub' for subtraction:  OUT(i,j,k) = IN(i,j,k) - AV(i,j)
'od'  for optical depth:  OUT(i,j,k) = LN (AV(i,j) / IN(i,j,jk))
Pixels are blanked if the inout pixel is blanked, the averaged
channel pixel is blanked, or the output is undefined ('od')
\par}
\module{boxspec}%
\noindent Write spectra (from vxy cube) as text file
\newline \ 
\newline \abox{Responsible:} Bart Wakker
\newline \abox{Keywords:} utility, plotting
\newline{\tenpoint\newline
BOXSPEC is a MIRIAD task to save spectra from a vxy image in
text files.
\keyword{\bf IN}
The name of the input image. No default.
\keyword{\bf CHAN}
The channel range to plot, default is all channels.
\keyword{\bf AVEOP}
If 'a' then the pixels enclosed in the x-y area specified
are averaged.  If 's' they are just summed.  Default is 'a'
\keyword{\bf SMLEN}
Hanning smoothing length (an odd integer {\tt <} 15) Default is
no smoothing (smlen = 1).
\keyword{\bf INFILE}
List of locations and sizes for boxes in format
{\eightpoint\begintt
        NPOS
        IPOS  X  Y  XOFF  YOFF
\endtt}
where 
{\eightpoint\begintt
        NPOS   is the number of boxes to read
        IPOS   is a number identifying each box, used in
               creating each output file name
        X,Y    is the pixel location of each box
        X,YOFF are the half sizes of the box in pixels so that
               each box is of size 2XOFF+1 in x and 2YOFF+1 in y.  
        X,YOFF are optional and default to 2
\endtt}
\keyword{\bf LOG}
Save each spectrum in a file with root name LOG.  The
rest of the file name is formed from ``IPOS'' so that
for example, for LOG=SPEC and IPOS=32 the output file name
is SPEC32   
Each file contains  :   channel, velocity or freq, intensity
\par}
\module{calapply}%
\noindent Apply gain and passband corrections producing calibrated data
\newline \ 
\newline \abox{Responsible:} Lee Mundy
\newline \abox{Keywords:} calibration
\newline{\tenpoint\newline
CALAPPLY applies gain and passband corrections producing an output
calibrated data file. The input data file is untouched. Gains and
passband solutions can be applied jointly or independently. Data
outside the valid time range of gains fits may by flagged bad.
\keyword{\bf VIS}
Input UV data file for calibration. Works on only one file
at a time. No default.
\keyword{\bf GCAL}
Name of file with gain calibration fits. This must be a data file
produced by calmake. Default is blank and results in no gain
calibration.
\keyword{\bf PCAL}
Name of file with passband calibration fits. This must be a data
file produced by passmake. Default is blank and results in no passband 
calibration.
\keyword{\bf FLAGBAD}
Two element true/false parameter for controlling flagging of data
beyond range of validity of amplitude and phase gains fits. If 
flagbad=true,true, data outside valid time range of gains
fits will be flagged bad (although the data are still calibrated
using extrapolation). If flagbad=false,false, no data are flagged
bad. Default is flagbad=true,true.
\keyword{\bf SELECT}
Standard select keyword. See section ``uv selection''.
\keyword{\bf LINE}
Standard linetype keyword, see section ``uv Linetypes''.
\keyword{\bf OUT}
Output file name for calibrated data. No default. File must not
already exist.
\par}
\module{calfit}%
\noindent Fit polynomial to amp/phase data
\newline \ 
\newline \abox{Keywords:} calibration
\newline \abox{Responsible:} Peter Teuben
\newline{\tenpoint\newline
CALFIT is a MIRIAD task which fits polynomials to the wideband data.  
By default, Calfit uses 2nd order for phases and 0th order for 
amplitudes, without phase closure.  The resulting fit is stored in a
calibration dataset, and can be over-written.
\keyword{\bf GCAL}
The calibration data set.  
Default: none.
\keyword{\bf SOURCE}
Names of the sources to which the fit is applied. 
Default is to select all sources from calibration set.
\keyword{\bf ORDER}
The polynomial order to be fit to the amplitude and phase.  The
default is $\{$0,2$\}$, while the maximum is $\{$10,10$\}$.
Default: 0,2.
\keyword{\bf CLOSE}
If set to true, a baseline based phase closure will be used in the
gain fits.
For 'true', breakpoints are forced to be equal across baselines, 
taken from the first baseline found in the database.
Default: false
\keyword{\bf TAVER}
Taver consists of two numbers, ``tgap'' and ``ttot'', both in minutes,
which are used for vector averaging. If the time interval between
any two successive data points is greater than ``tgap``, or if the 
total time between the first data point in a vector average and any
succeeding data point exceeds ``ttot'', then a new vector average is
started.
Default: 0.0,1000.0 (no vector averaging)
\keyword{\bf MODE}
Work on 'phase' or 'amplitude'. If nothing specified, both
will be worked on. 
Default: both.
\keyword{\bf UNITS}
Units to work in. Options are: K, K/Jy, Jy/K.
See comments in ``calflag''
Default: Jy/K.
\par}
\module{calflag}%
\noindent Display raw calibration data with user interaction
\newline \ 
\newline \abox{Keywords:} calibration
\newline \abox{Responsible:} Peter Teuben
\newline{\tenpoint\newline
CALFLAG is a MIRIAD task which displays the raw calibration data and 
optionally a fit. CalFlag reads the data and fit from a cal file and 
displays the result visually.  The phases and amplitudes for both
sidebands and all baselines are simultaneously shown, but individual
plots or groups of plots can be zoomed.  The fit is displayed as a
line over the points, and all the axes are labeled, with ticks.  The
user may interactively pick different plots to zoom on, and can mark
individual points as good or bad. Breakpoints in time can be added
and deleted as well. To make new polynomial fits, the program CalFit
has to be rerun.

Commands in cursor mode are:

{\eightpoint\begintt
  d/D -- flag the nearest point as bad (uppercase only one band)
  a/A -- flag the nearest point as good (uppercase only one band)
  x   -- zoom/unzoom on the current column
  y   -- zoom/unzoom on the current row
  z   -- zoom/unzoom on the current plot
  b/B -- break polynomial here (uppercase only one band)
  c/C -- (re)connect polynomials here (uppercase only one band)
  q   -- quit (no save)
  e   -- exit (save flags and breakpoints if modifications done)
  ?   -- help and dedraw screen
\endtt}
\keyword{\bf GCAL}
The calibration data set.  
No default.
\keyword{\bf SOURCE}
Source names to be selected. 
Default is all sources selected.
\keyword{\bf FLAGS}
Pay attention to the flags? Default is true.
If set to false, all points are selected for viewing, including
the ones flagged bad.
\keyword{\bf CLOSE}
If set to true, will be true for phase closure, but false for 
baseline fit.  When set to true (an antenna based phase closure 
solution). When set to true, breakpoints are also forced to be at
the same time accross baselines.
Default: false.
\keyword{\bf TAVER}
Taver consists of two numbers, tgap and ttot, both in minutes,
which are used for vector averaging. If the time interval between
any two successive data points is greater than tgap,, or if the total
time between the first data point in a vector average and any
succeeding data point exceeds ttot, then a new vector average is
started.
Default: 0.0,1000.0 (no vector averaging)
\keyword{\bf DEVICE}
The PGPLOT graphics device.  
Default is ``?''
\keyword{\bf RESET}
Export mode: a logical, or can be left alone.
If set to 't' or 'f' all flags in the file are set to such.
The default is blank, meaning flags are not reset. If flags
are reset, there is not user interaction possible, and no display 
is started. This is a quick and dirty way to reset all flags
in the calibration dataset to some value. In this mode no graphics
interaction is used.
Default: not used.
\keyword{\bf UNITS}
Units to work in. Options are: K, K/Jy, Jy/K.
Note: since CALAPPLY assumes Jy/K any other units are nice to
look at, but produce badly calibrated files.
Default: Jy/K.
\keyword{\bf AMPMAX}
If supplied, the amplitude scale is fixed from 0 to ampmax,
otherwise autoscaling is done. Default is autoscaling.
\par}
\module{calflux}%
\noindent Print flux data for a calibrator source.
\newline \ 
\newline \abox{Responsible:} Jim Morgan
\newline \abox{Keywords:} calibration, flux
\newline{\tenpoint\newline
CALFLUX returns the flux (in Jy) of a calibrator source at a
given frequency (in GHz).  The source resides in a calibration
file that is formatted such that each record is composed of
white space separated fields ordered Source, Day (yymmmdd.d),
Freq (GHz), Flux (Jy), and rms (Jy).  Lines beginning with
a ``!'' are excluded.  Warnings are returned if there is no match
of the source in the calibrator file, if there is no matching
entry at the desired frequency, or the observation date is
greater than 4 years.  The Day field, presently, is formatted
as yymmmdd.d where yy is the year field 19yy, mmm is the three
character string of the month, and dd.d is the decimal value
of days.  None of the inputs are required, but they provide a
means of bracketing the desired source(s).
\keyword{\bf IN}
Name of the calibration data file (Default is the file
cals.fluxes in the directory MIRCAT).
\keyword{\bf SOURCE}
Name of the calibration source to list (default is all sources).
The source name is minimum match format.
\keyword{\bf FREQ}
Frequency that the source was observed at in GHz (default is
0.0, which implies all frequencies are valid matches).
\keyword{\bf DELFREQ}
A full width in GHz (default is 0.0) in which to accept
deviations from the value of ``freq'' as a match.  If ``freq''
is not given or is set to the default value, this input
is ignored.
\keyword{\bf DATE}
The cutoff date before or after which no observations are
listed (default is date=1.0, which implies all dates are valid
matches).  If ``date=0'', then only the most recent data is
listed.  If ``date{\tt >}0,'' then all data more recent than ``date''
are listed; ``date{\tt <}0,'' then all data prior to ``abs(date)''
are presented.  The format for ``date'' is the same as the DATE
field in the flux calibration file:  either ``yymmmdd.d'' or
``yymmmdd:hh:mm:ss.s'' with no internal spaces (the first 7
characters are required).
\keyword{\bf DELDATE}
A full width in Julian days (default is 0.0) in which to accept
deviations from the value of ``date'' as a match.  If ``date''
is not given or it is set to the default value, then this input
is ignored.
\keyword{\bf FLUX}
The lower limit flux value to consider as a match (default is
all matching fluxes).
\par}
\module{calib}%
\noindent Display raw calibration data with user interaction
\newline \ 
\newline \abox{Responsible:} Peter Teuben
\newline \abox{Keywords:} calibration
\newline{\tenpoint\newline
CALIB is a MIRIAD task which displays the raw calibration data and 
optionally a fit. Calib reads the data and fit from a cal file and 
displays the result visually.  The user has the option to view the
amplitudes, the phases, or the sums and difference of the phases in
the two sidebands.   The fit is displayed as a
line over the points, and all the axes are labeled, with ticks.  The
user may interactively pick different plots to zoom on, and can mark
individual points as good or bad. Breakpoints in time can be added
and deleted as well. A new polynomial fits can also be added 
interactively.

Commands in cursor mode are:

{\eightpoint\begintt
  d/D -- flag the nearest point as bad (uppercase only one band)
  a/A -- flag the nearest point as good (uppercase only one band)
  x   -- zoom/unzoom on the current column
  y   -- zoom/unzoom on the current row
  z   -- zoom/unzoom on the current plot
  b/B -- break polynomial here (uppercase only one band)
  c/C -- (re)connect polynomials here (uppercase only one band)
  0-9 -- order of new fit on current row
  ma  -- change mode to amplitudes
  mp  -- change mode to regular phases (U and L)
  md  -- change mode to phase difference and average
  q   -- quit (no save)
  e   -- exit (save flags, breakpoints and fits if modifications done)
  ?   -- help and dedraw screen
\endtt}
\keyword{\bf GCAL}
The calibration data set.  
No default.
\keyword{\bf SOURCE}
Source names to be selected. 
Default is all sources selected.
\keyword{\bf FLAGS}
Pay attention to the flags? Default is true.
If set to false, all points are selected for viewing, including
the ones flagged bad.
\keyword{\bf MODE}
Work on 'amplitude', 'phase', or 'difs'.
Default: amplitude
\keyword{\bf CLOSE}
If set to true, will be true for phase closure, but false for 
baseline fit.  When set to true (an antenna based phase closure 
solution). When set to true, breakpoints are also forced to be at
the same time accross baselines.
Default: false.
\keyword{\bf TAVER}
Taver consists of two numbers, tgap and ttot, both in minutes,
which are used for vector averaging. If the time interval between
any two successive data points is greater than tgap,, or if the total
time between the first data point in a vector average and any
succeeding data point exceeds ttot, then a new vector average is
started.
Default: 0.0,1000.0 (no vector averaging)
\keyword{\bf DEVICE}
The PGPLOT graphics device.  
Default is ``?''
\keyword{\bf RESET}
Export mode: a logical, or can be left alone.
If set to 't' or 'f' all flags in the file are set to such.
The default is blank, meaning flags are not reset. If flags
are reset, there is not user interaction possible, and no display 
is started. This is a quick and dirty way to reset all flags
in the calibration dataset to some value. In this mode no graphics
interaction is used.
Default: not used.
\keyword{\bf UNITS}
DANGEROUS: Units to work in. Options are: K, K/Jy, Jy/K.
Default: Jy/K.
\keyword{\bf AMPMAX}
If supplied, the amplitude scale is fixed from 0 to ampmax,
otherwise autoscaling is done. Default is autoscaling.
\par}
\module{callist}%
\noindent List items from calibration data set
\newline \ 
\newline \abox{Responsible:} Peter Teuben
\newline \abox{Keywords:} calibration
\newline{\tenpoint\newline
CALLIST is a MIRIAD task that lists some items in a calibration set.
\keyword{\bf GCAL}
Name of the calibration set
\keyword{\bf VERBOSE}
When set to 'true' is displays information per integration
time. The default is 'false', which only displays when source
name changes.
\keyword{\bf POLY}
When set to 'true' is displays information of all polynomials
\keyword{\bf BREAK}
When set to 'true' is displays information of all breakpoints
\keyword{\bf CODE}
Slot code for band/phase/amp to print. Must be 3 or 4
characters. If empty, correllations are not printed.
First character is A(mp) or P(phase).
Second character is U(pper) or L(ower) sideband.
Third character is W(ide) band or any of 1..8 for passband.
Default: Empty.
\keyword{\bf BASE}
Baseline, combined with slot code specified before, to print.
Valid entries must be 256*A1+A2, where A1 is antennae 1 and
A2 antennae 2. Note: A1 {\tt <} A2.
Default: 0
\keyword{\bf LOG}
Output device. (default is standard user output)
\par}
\module{calmake}%
\noindent Create cal file from observation of calibrator
\newline \ 
\newline \abox{Responsible:} Peter Teuben
\newline \abox{Keywords:} calibration
\newline{\tenpoint\newline
CALMAKE is a MIRIAD task which creates a calibration dataset.
Calmake reads in data set(s) of phase calibration source(s),
and creates a calibration data set, called a ``gcal'' file. 
The data within the gcal file is taken from either the wide 
band correlation channels or some averaged value from the narrow 
band channels.
At this time the data is converted to phase-amplitude, and
phase discontinuities are resolved.  If the user wishes, points
which cannot be phase-resolved adequately are marked as bad. 
\keyword{\bf VIS}
A list of input UV data sets to be used for calibration.  
No default.
\keyword{\bf GCAL}
The output gain calibration set.  
No default.
\keyword{\bf LINE}
Selection of the input data type used for output.
Line type of the data is in the format (see User's manual)
{\eightpoint\begintt
   type, nchan, start, width, step
\endtt}
Here ``type'' must be ``channel'', ``wide'', or ``velocity''.
The maximum number of channels must be 2 for this program.
The default is to select type ``wide'' with ``nchan'' set to 2,
``start'' set to 1, and ``width'' and ``step'' both set to 1.
\keyword{\bf FLAGBAD}
If ``flagbad'' is true and if any baseline does not close to
better than PI radians, then all baselines from that sideband
and integration time are flagged as bad data.
Default: false
NOTE: This option has been disabled since July 1990 when the 
new phase flipper was installed.
\keyword{\bf SELECT}
Selects which visibilities are to be used.  The default is to use
all visibilities.  See the Users Manual for information on how
to specify uv data selection.
\keyword{\bf AUTO}
Use auto breakpoints when the source name changes?
Default: True.
\keyword{\bf FLUXES}
List of sources and associated fluxes in Jy to be used for
calibration. The format is name of the source, followed by
its flux, and optionally more sources, e.g.
fluxes=3c273,10.3,3c84,12.3,bllac,2.2
Currently no frequency dependency can be specified.
\keyword{\bf FLUXTAB}
Name of the calibrators flux table file. If no name provided, the
system default (MIRCAT/calibrator.fluxes) is taken. The user
can also supply fluxes by hand (see keyword fluxes above) in
which case the flux table derived values are ignored.
\par}
\module{cgplot}%
\noindent Makes contour and grey scale plots
\newline \ 
\newline \abox{Responsible:} Neil Killeen
\newline \abox{Keywords:} plotting
\newline{\tenpoint\newline
CGPLOT makes contour and grey scale overlay plots of Miriad
images. Up to 2 contour plots and one grey scale plot may be
overlaid in multipanel plots of multichannel images. In addition
overlay locations (plotted as boxes or stars) may be specified
from an ascii file.   Substantial control of the plot parameters
allows publishable quality plots.  
\keyword{\bf CIN}
The input images (up to 2) to contour (optional).  If two
images are specified, they must have identical dimensions.
Default is no contour plot.
\keyword{\bf GIN}
The input file to make a grey scale of (optional). Can be the
same as either contour image.  Note that when CIN and GIN are
specified, the images must be the same size in all dimensions,
except the special case of one image with only two dimensions,
and the other with more than two (e.g. overlay multi-channel
images on a continuum image).
Default is no grey scale.
\keyword{\bf REGION}
Region of interest.  Use only hyper-rectangles please with
the BOX and IMAGE commands.  
Default is full image.
\keyword{\bf CHAN}
Two values. The first is the channel increment, the second is 
the number of planes to average, for each sub-plot.  Thus
CHAN=5,3  would average groups of 3 channels together, starting
5 channels apart such as: 1:3, 6:8, 11:13 ... 
In the case that CIN and GIN are both specified, and one of 
those images has two dimensions and the other three, then 
CHAN refers to that image with three dimensions. 
Defaults are 1,1
\keyword{\bf SLEV}
Two values for first contour image. First value is the type of 
contour level scale factor.  'p' for percentage and 'a' for 
absolute.   Second value is the level to scale LEVS by.  Thus   
SLEV=p,1  would contour levels at LEVS * 1\% of the image peak 
intensity.  Similarly, SLEV=a,1.4e-2   would contour levels at 
LEVS * 1.4E-2. 
Default is no additional scaling of LEVS
\keyword{\bf LEVS}
Levels to contour for first image, are LEVS times SLEV 
(either percentage of the image peak or absolute). 
Defaults try to choose something sensible.
\keyword{\bf SLEV2}
SLEV for second contour image.
\keyword{\bf LEVS2}
LEVS for second contour image.
\keyword{\bf RANGE}
Three values. The grey scale range (background to foreground),
and transfer function type ('lin' or 'log').
Default is linear between the image minimum and maximum.
\keyword{\bf DEVICE}
The PGPLOT plot device, such as plot.plt/ps 
Default is ? (prompt)
\keyword{\bf NXY}
Number of sub-plots in the x and y directions on the page. 
Defaults choose something sensible.
\keyword{\bf LABTYP}
Two values.  One for the spatial label type of each axis 
(x then y) as follows:
'rd'  the labels are RA and DEC, 
'p '  the labels are pixels
'po'  the labels are pixel offsets
'ao'  the labels are arcsecond offsets
'l '  the labels are linear coordinates as defined by the header
'lo'  the labels are offset linear coordinates

All offsets are from the reference pixel.  If you give only one 
value, the second (y) axis defaults to that type.
Defaults are 'p', 'p'
\keyword{\bf OPTIONS}
A list of plot annotation and format options. Minimum match 
of all keywords is active.
\newline\newline 'full' means do full plot annotation with contour levels, gray
  scale range, file names, reference values, etc.  Otherwise 
  more room for the plot is available. 
\newline\newline 'velocity' means label each sub-plot with the appropriate value 
  of the third axis (e.g. velocity or frequency).   
\newline\newline 'channel' means label each sub-plot with the channel number. 
  Both 'vel' and 'chan' can appear, and both will be written 
  on the plot
\newline\newline 'equal' means draw plots with equal scales in x and y. 
  Otherwise they will be unequal.
\newline\newline 'gaps' means leave gaps between sub-plots and label each 
  sub-plot, otherwise they will abut each other.
\newline\newline 'solneg1' means make negative contours solid and positive 
  contours dashed for the first contour image. The default, 
  and usual convention is solid positive and dashed 
  negative contours. 
\newline\newline 'solneg2' SOLNEG1 for the second contour image.
\keyword{\bf LINES}
Two values.  The first is the integer line width.  This line
width is used for the plot labels and annotation.
The second is the intensity level for the break between solid 
and dashed contours (see also SOLNEG in OPTIONS above). 
Defaults are +1, 0.0
\keyword{\bf LINES2}
LINES for the for second contour image.  This line width
is used ONLY for the contours, not any labelling.
\keyword{\bf CSIZE}
Three values.  Character sizes in units of the PGPLOT default
(which is \~ 1/40 of the view surface height) for the plot axis
labels, the velocity/channel label, and the overlay ID string
(if option `w' in OLAY used) label.
Defaults choose something sensible.  Use 0.0 to default the 
first or second, but not the second or third, e.g., 0.0, 1.5
\keyword{\bf SCALE}
Two values.  Scales in linear axis units/mm with which to plot
in the  x and y directions.  For example, if the increments 
per pixel are in radians, then this number would be radians/mm.
Although this choice of unit may be cumbersome, it makes no 
assumptions about the axis type, so is more flexible.   If you 
also chose OPTIONS=EQUAL then one of your scales, if you set 
both and differently will be over-ruled.  When plotting on hard
devices, the scales are printed on the terminal, so that you may
fine adjust them with ease.
If you give only one value, the second defaults to that. 
Defaults choose scales to fill the page optimally. To default 
the first but the second, use 0.0
\keyword{\bf OLAY}
Three values.   
\newline\newline The first value is the name of a file containing a list of
overlay (star or box) locations (NO TABS PLEASE!!). 
Details of the file format follow below.
Default is no overlays.
\newline\newline The second value is the type overlay (boxes or crosses)
\par 's'   for star overlays (i.e. crosses are plotted)
\par 'b'   for box overlays 
\par  If in addition, a 'w' follows the type (e.g., 'sw') then the
   overlay identification string (the first column in the overlay
   file, see below) is written in the corner of the overlay. 
   Default is 'b'
\newline\newline The third value (call it OTYPE) gives the units of the overlay 
   location in the file.   For 'p ', 'po', 'ao', 'l ', 'lo', the
   units of the overlay locations (and half-sizes) are pixels 
   (pixels), pixel offsets (pixels), arcsecond offsets
   (arcseconds), linear coordinates as defined by the header 
   (linear coords), and linear coordinate offsets (linear 
   coordinates).  All offsets are from the reference pixel. 
   Default is 'p '.
\newline\newline The format of the overlay file depends upon OTYPE.
First line must be the number of overlays.  
Successive lines as follows.

For OTYPE = 'rd'
{\eightpoint\begintt
  ID   HH MM SS.S   DD MM SS.S   XOFF   YOFF  CHAN
\endtt}
where ID is an identifying overlay string which can be 
optionally written in the overlay corner (see OTYPE above)
HH MM SS.S DD MM SS.S is the RA and DEC of the overlay center
XOFF is the RA half size of the overlay in pixels
YOFF is the DEC half size of the overlay in pixels
CHAN is the channel on which to overlay (0 means all channels)

For OTYPE = 'p ', 'po', 'ao', 'l ', 'lo'
{\eightpoint\begintt
  ID  X  Y  XOFF   YOFF  CHAN
\endtt}
where X,Y defines the center of the overlay in pixels ('p'),
offset pixels from the reference pixel ('po'),
offset arcseconds from the reference pixel ('ao'),
linear coordinates as defined by the header ('l')
or offset linear coordinates as defined by the header ('lo').
XOFF and YOFF are the overlay half-sizes in pixels ('p', 'po'),
arcseconds ('ao'), and linear coordinates ('l', 'lo').

In all cases, XOFF, YOFF and CHAN are optional.  The defaults
are 2, 2, and 0, respectively.  Obviously, you can't default YOFF
and specify CHAN.  The half-size defaults are probably only useful
for locations in pixels.
\par}
\module{clean}%
\noindent Apply Hogbom, Clark or Steer CLEAN algorithm to a map
\newline \ 
\newline \abox{Responsible:} Mel Wright
\newline \abox{Keywords:} deconvolution
\newline{\tenpoint\newline
CLEAN is a MIRIAD task, which performs a hybrid Hogbom/Clark/Steer Clean
algorithm, which takes a dirty map and beam, and produces an output
map which consists of the Clean components. This output can be
input to SELFCAL to self-calibrate visibilities, or input to RESTORE
to produce a ``clean'' image. Optionally CLEAN can take as one of
its inputs a model of the deconvolved image. This model could be
from a previous CLEAN run, or from other deconvolution tasks
(e.g. MAXEN). 
\keyword{\bf MAP}
The input dirty map, which should have units of Jy/beam. No
default. 
\keyword{\bf BEAM}
The input dirty beam. No default
\keyword{\bf MODEL}
An initial model of the deconvolved image. This could be the
output from a previous run of CLEAN, or the output of any of the
deconvolution tasks (e.g. MAXEN). It must have flux units of
Jy/pixel. The default is no model (i.e. a zero map). 
\keyword{\bf OUT}
The name of the output map. The units of the output will be
Jy/pixel. This file will contain the contribution of the input model.
It also should have a different name to the input model (if any).
It can be input to RESTORE, CLEAN (as a model, to do more cleaning),
or SELFCAL (for self-calibrating visibility data).
\keyword{\bf GAIN}
The minor iteration loop gain. Default is 0.1.
\keyword{\bf CUTOFF}
CLEAN finishes when the absolute maximum residual falls below
CUTOFF. Default is 0. 
\keyword{\bf NITERS}
The maximum number of minor iterations. Clean is finished when the
absolute value of (NITERS minor iterations) have been performed.
Clean may finish before this point, however, if NITERS is negative
and the absolute maximum residual becomes negative valued, or if
the cutoff level (as described above) is reached. 
\keyword{\bf REGION}
This specifies the region to be Cleaned. See the Users Manual for
instructions on how to specify this. The default is the inner
quarter of all planes.
\keyword{\bf PHAT}
The Cornwell's prussian hat parameter. When cleaning extended sources,
CLEAN may produce a badly corrugated image. This can be suppressed
to some extent by cleaning with a dirty beam which has had a spike
added at its center (i.e. a beam that looks like a prussian hat).
PHAT gives the value of this spike, with 0 to 0.5 being good
values. Default is 0 (but use a non-zero value for extended
sources). 
\keyword{\bf MINPATCH}
The minimum patch size when performing minor iterations. Default
is 51, but make this larger if you are having problems with
corrugations. You can make it smaller when cleaning images which
consist of a pretty good dirty beam. 
\keyword{\bf SPEED}
This is the same as the speed-up factor in the AIPS APCLN.
Negative values makes the rule used to end a major iteration more
conservative. This causes less components to be found during a
major iteration, and so should improve the quality of the Clean
algorithm Usually this will not be needed unless you are having
problems with corrugations. A positive value can be useful when
cleaning simple point-like sources. Default is 0. 
\keyword{\bf MODE}
This can be either ``hogbom'', ``clark'', ``steer'' or ``any'', and
determines the Clean algorithm used. If the mode is ``any'', then
CLEAN determines which is the best algorithm to use. The default
is ``any''. 
\keyword{\bf CLIP}
This sets the relative clip level in a Steer clean, values
typically being 0.75 to 0.9. The default is image dependent. 
\par}
\module{contsub}%
\noindent Subtract continuum map from vxy cube by spectral fitting
\newline \ 
\newline \abox{Responsible:} Bart Wakker
\newline \abox{Keywords:} map combination
\newline{\tenpoint\newline
CONTSUB will subtract off a continuum map from a VXY cube by
averaging two regions in the spectrum, passing a baseline through
the line defined by those two points, and subtracting that baseline
from the spectrum. This is done at each pixel. 
\keyword{\bf IN}
The name of the input image. No default.
\keyword{\bf OUT}
The name of the output image. No default.
\keyword{\bf LR}
Range of channels to be averaged at low end of spectrum. No default.
\keyword{\bf HR}
Range of channels to be averaged at high end of spectrum. No default.
\par}
\module{convol}%
\noindent Convolve a cube with a "beam function"
\newline \ 
\newline \abox{Responsible:} Mel Wright
\newline \abox{Keywords:} map manipulation, map analysis
\keyword{\bf MAP}
The input map. This can be two or three dimensional. It should have
units of Jy/beam. No default.
\keyword{\bf BEAM}
The input beam. This cannot be three dimensional. This must
be symmetrical about its reference pixel. No default.
\keyword{\bf REGION}
The region of the input map to convolve. See the Users Manual for
instructions on how to specify this. The default is the entire
input map.
\keyword{\bf OUT}
The output image. No default.
\module{copyhd}%
\noindent Copy items from one data-set to another.
\newline \ 
\newline \abox{Responsible:} Robert Sault
\newline \abox{Keywords:} utility
\newline{\tenpoint\newline
COPYHD is a Miriad task which copies items from one Miriad data-set
to another. There is no interpretation of the items at all.
\keyword{\bf IN}
Name of the input data set. No default.
\keyword{\bf OUT}
Name of the output data set. This must already exist. No default.
\keyword{\bf ITEMS}
A list of items to be copied across. At least one value must be
given.
\par}
\module{corset}%
\noindent Program that models correlator setups.
\newline \ 
\newline \abox{Responsible:} Jim Morgan
\newline \abox{Keywords:} utility
\newline{\tenpoint\newline
Program to model correlator setups.  It takes an observing frequency
and IF, then searches a line file for all spectral lines that would
be observable in the given setup.  The positions of the lines in the
correlator for various correlator parameters can be plotted. 
\keyword{\bf IN}
The name of the line file.
Default is ``MIRCAT:lovas3mm.dat.''
\keyword{\bf FREQ}
The rest frequency of the line in GHz.
Default is 110 GHz.
\keyword{\bf IF}
The intermediate frequency in MHz.
Default is 290 MHz.
\keyword{\bf OVERRIDE}
The override harmonic number.
Default is 0.
\keyword{\bf REFOSC}
The reference oscillator code (0-new, 1-VLBI, 2-old).
Default is new.
\keyword{\bf VLSR}
Vlsr of the source in km/s.
Default is 0 km/s.
\keyword{\bf DEVICE}
The device used for plotting.  See the User's Manual for
details on how to specify the device.
Default is a prompt asking for the device.
\keyword{\bf LOG}
The output log file.
The default is the terminal.
\keyword{\bf MODE}
Correlator mode (1-4).  See the User's Manual for an
explanation of the modes.
Default is 2.
\keyword{\bf CORFS}
Correlator frequency, in MHz, for the four mode settings.
Corfs are mode dependent; defaults are shown below:

{\eightpoint\begintt
     mode               corfs
      1                corf(1)=340
      2                corf(1)=340, corf(2)=460
     3,4               corf(1)=340, corf(2)=460
                       corf(3)=100, corf(4)=100
\endtt}
\keyword{\bf BW}
Correlator bandwidth, in MHz.
Bw(1) and bw(2) should take on values that are multiples of
2 starting at 1.25 MHz up to 40 MHz.
\keyword{\bf OPTIONS}
Type ``cross'' for cross-correlation or ``auto'' for
auto-correlation.       
Default is cross-correlation.
\par}
\module{delhd}%
\noindent Delete a data-set item.
\newline \ 
\newline \abox{Responsible:} Robert Sault
\newline \abox{Keywords:} utility
\newline{\tenpoint\newline
DELHD is a MIRIAD task to delete an item from a Miriad image or uv
data set.
\keyword{\bf IN}
The name of an item within a data set. This is given in the
form:
{\eightpoint\begintt
  % delhd in=dataset/item
\endtt}
\par}
\module{demos}%
\noindent Inverse mosaicing operation
\newline \ 
\newline \abox{Responsible:} Mel Wright
\newline \abox{Keywords:} map manipulation
\newline{\tenpoint\newline
DEMOS (de-mosaic) is a MIRIAD task which takes a primary beam
corrected cube, and forms output cubes. The output cubes are formed
by applying a primary beam at various pointing centres. Thus this task
performs the inverse operation of mosaicing. The input pointing
centres and the primary beam size are specified either directly, or
by giving a uv containing multiple pointings.

Because the output of DEMOS are not primary beam corrected, they can
be used for comparison with other uncorrected images and data. In
particular SELFCAL cannot handle a model which is primary beam
corrected, though it can handle a visibility data file containing
multiple pointings. Thus you could use DEMOS to break the model into
several models which are not primary beam corrected.
\keyword{\bf MAP}
This is the name of image, that is to be de-mosaiced. No default.
\keyword{\bf VIS}
This is an input uv file, whose pointing centres can act as
defaults (templates) for the ``center'' parameter. As this is only
used to determine defaults, it makes no sense to specify both ``vis''
and ``center''.
\keyword{\bf CENTER}
This gives a list of pairs of offset pointing positions, relative
to the reference pixel of ``map''. The offsets are measured in
arcseconds in the plane of the sky, in the directions of (ra,dec).
A pair of values must be given for each pointing, giving the offset
in x and y. If this is not given, the pointing centers are determined
from the ``vis'' file. Either ``vis'' or ``center'' must be present.
\keyword{\bf PBFWHM}
The Primary beam is modeled by circular gaussian. PBFWHM gives the
full width at half maximum of the primary beam, in arcseconds. One
value is given. If no value is given, DEMOS uses the default as
the primary beam size of the telescope used to observe the ``map''
file (if DEMOS knows the primary beam of the telescope).
\keyword{\bf OUT}
This gives a template name for the output images. The actual output
image names are formed by appending a number (starting at one) to
this output name. For example, if out=cygnus, then the output images
will be called cygnus1, cygnus2, etc.
\keyword{\bf IMSIZE}
This gives two values, being the output image size, in x and y.
If no value is given, then the outputs will be one primary beam
width in size. If one value is given, then this is used for both
x and y. Each output size might be smaller than this, to prevent
each output from extending beyond the edges of the input image.
\par}
\module{doc}%
\noindent MIRIAD documentation program
\newline \
\newline \abox{Responsible:} Bart Wakker
\newline \abox{Keywords:} tools
\newline{\tenpoint\newline
DOC extracts on-line documentation and presents the output in a
variety of formats. Non-programmers need only use DOC to display
formatted on-line documentation of MIRIAD tasks (eg, ``doc fits''). The
remainder of this documentation is intended only for MIRIAD
programmers.

Usage:
{\eightpoint\begintt
 doc [-f] [-ditruUpP] [-m module] [-x texformat] [-s type] [-o dir]
     [-a] [-w#] [-e] [-l listfile] [files]

\endtt}
Options:
{\eightpoint\begintt
    (none) show list of options
    -f     show the format for input files
    -d     type formatted document(s) (default option)
    -i     make an alphabetized list of routines in input file
    -t     make a list of routines by functional category
    -r     extract codename of programmer responsible for routines
    -U     make a single TeX/LaTeX output of all input files
    -u     make TeX/LaTeX files, one for each routine in inputs
    -P     make a single on-line-format output of routines in inputs
    -p     make on-line-format files, one for each routine in inputs
    -m     search for documentation of 'module' in list of input files
    -x     select TeX or LaTeX (default); choice case insensitive
    -s     select to make a LaTeX section/subsection for each task
    -o     put 'dir' as directory of source code in the output
    -a     ask user what to do if an output file already exists
    -w     stop printing after # lines and ask if user wants to go on
    -e     do not check for filename extensions of input files
    -l     process all files listed on file 'listfile'
    files input file(s) to be processed

\endtt}
Without the -e option, doc will work only on input files with
extensions .for, .f, and .c (source code) and with extensions .doc,
.sdoc, .tdoc, and .cdoc (on-line documentation).

Input files can be read from a file (option -l) or listed. These
options can be combined. If no filename extensions are given for
[files], they are searched for in \$MIRPDOC and \$MIRSDOC.

Option -u writes output in the user's current working directory, with
filename extension .tex. Option -p writes output in the user's current
working directory with the appropriate .doc, .cdoc, .sdoc, or .tdoc
filename  extension.

Options -i and -t are used to construct alphabetic and systematic
indices.

Below a short description of in-code format, where output files have
extensions .doc, .sdoc, .tdoc, and .cdoc:

{\eightpoint\begintt
 c=  [routine name] [one-line description] (for main programs/scripts)
 c*  [routine name] [one-line description] (for subroutines)
 c&  programmer responsible for the routine
 c:  comma-separated list of categories pertaining to the routine
 c+
 c   multi-line program description block
     [subroutine call and variable declarations]
 c@  keyword
 c   multi-line keyword description
 c<  standard-keyword
 c   multi-line description of non-standard features
 c-- end of multi-line documentation block

\endtt}
The flag character 'c' may also be 'C' or '/*' (the latter for .c
files). In the case where option -e was used, directives should start
with \# or \$! (for unix scripts and VMS command files, respectively).

The c{\tt <} directive may be used for some keywords which have a standard
description (in, out, vis, select, line, region, server, device).

The keyword directives c@ and c{\tt <} may be used more than once, unlike
the other directives. For tasks and scripts (i.e. not for subroutines)
the first character on the first non-empty line of the documentation
blocks determines an ``alignment column''. When converting to (La)TeX
format, lines that have a space in this column are typeset
``verbatim''. No line should start before the alignment column. The
documentation block between c+ and c-- may contain any character,
except that the tilde (\~) cannot be used inside a verbatim block.
Backslashes should be doubled.

Recognized task categories (: directive):
{\eightpoint\begintt
 General           Utility          Data Transfer   Visual Display
 Calibration       uv Analysis      Map Making      Deconvolution
 Plotting          Map Manipulation Map Combination Map Analysis
 Profile Analysis  Model Fitting    Tools           Other

\endtt}
Recognized subroutine categories (: directive):
{\eightpoint\begintt
 Baselines         Calibration      Convolution     Coordinates
 Display           Error-Handling   Files           Fits
 Fourier-Transform Gridding         Header-I/O      History
 Image-Analysis    Image-I/O        Interpolation   Least-Squares
 Log-File          Low-Level-I/O    Mathematics     Model
 PGPLOT            Plotting         Polynomials     Region-of-Interest
 SCILIB            Sorting          Strings         Terminal-I/O
 Text-I/O          Transpose        TV              User-Input
 User-Interaction  Utilities        uv-Data         uv-I/O
 Zeeman            Other

\endtt}
Recognized script/command file categories (: directive):
{\eightpoint\begintt
 System Operation  Programmer Tool  User Utility     Other

\endtt}
Examples of use:
doc fits                    - Print on-line doc on the screen
doc -p \$MIRPROG/* /fits.for  - Generate on-line doc
doc -u \$MIRPROG/* /fits.for  - Generate tex file
doc -e mir.bug.csh          - Print on-line doc of a script
doc -m xysetpl \$MIRSUBS/*   - Search for doc of xysetpl in MIRSUBS
doc -m xysetpl \$MIRSDOC/*   - Search for doc of xysetpl in MIRSDOC

\par}
\module{ellint}%
\noindent Integrate a Miriad image in elliptical annuli.
\newline \ 
\newline \abox{Responsible:} Mel Wright
\newline \abox{Keywords:} image analysis
\newline{\tenpoint\newline
ELLINT integrates a Miriad image in elliptical annuli in the first
two image planes, e.g. to find the radial brightness distribution,
or flux density as a function of distance in a galaxy. The
integration is done separately for each image plane in the region
included.
\keyword{\bf IN}
Input image name. xyz images only. No default.
\keyword{\bf REGION}
Region of image to be integrated. E.g.
{\eightpoint\begintt
  % ellint region=relpix,box(-4,-4,5,5)(1,2)
\endtt}
integrates the center 10 x 10 pixels for image planes 1 and 2.
The default region is the entire image.
\keyword{\bf CENTER}
The center of the annuli in arcsec from the center pixel, measured
in the directions of RA and DEC.
\keyword{\bf PA}
Position angle of ellipse in degrees. Default is 0 (north).
\keyword{\bf AR}
Axial ratio of ellipse. (0 to 1) default 1.
\keyword{\bf RADIUS}
Inner and outer radii and step size along major axis in arcsecs.
The default is the whole image in steps equal to the pixel size.
\keyword{\bf PBFWHM}
Primary beam fwhm in arcsecs (RA,DEC). If one value is given it is
used for both RA and DEC. pbfwhm=-1 uses the value from the image
header if present. The default is no correction.
\keyword{\bf LOG}
The output log file. The default is the terminal.
\par}
\module{fft}%
\noindent Fourier transform on image(s)
\newline \ 
\newline \abox{Responsible:} Mel Wright
\newline \abox{Keywords:} uv analysis, map making
\newline{\tenpoint\newline
FFT is a MIRIAD task which performs a fast Fourier transform on
an image. If the input is a cube, then each plane is FFT'ed
individually (i.e. this does not perform a 3D FFT).
\keyword{\bf RIN}
This gives the input real part image. No default.
\keyword{\bf IIN}
This gives the input imaginary part image. The default is a zero image.
\keyword{\bf SIGN}
This gives the sign of the exponent in the transform. -1 gives a
forward transform, +1 an inverse transform. An inverse transform
applies 1/N scaling. The default is a forward transform.
\keyword{\bf CENTER}
This gives the phase center of the transform. If two values are
given, then they are used as the phase center in the x and y axis
respectively. If one value is given, then this is used for the
transform center for both the x and y dimensions. The default is the
header value for CRPIX1 and CRPIX2 (if they are in the header) or
N/2+1 (if CRPIX is missing from the header).
\keyword{\bf ROUT}
The output real part image. The default is not to write this image out.
\keyword{\bf IOUT}
The output imaginary part image. The default is not to write this
image out.
\keyword{\bf MAG}
The output magnitude image. the default is not to write this image out.
\keyword{\bf PHASE}
The output phase image. The default is not to write this image out.
\par}
\module{fitgains}%
\noindent List gains for a uv dataset
\newline \ 
\newline \abox{Responsible:} Mel Wright
\newline \abox{Keywords:} uv analysis
\newline{\tenpoint\newline
FITGAINS lists the gains derived from a self-calibration 
for a MIRIAD UV data file. The mean and rms of the gains
are calculated.
\keyword{\bf VIS}
The input UV dataset name. Only the gains item needs to be
present in the uv dataset. No default.
\keyword{\bf LOG}
The list output file name.  The default is the terminal. The
gains are listed versus time, and can be plotted using Mongo.
\keyword{\bf REFANT}
The gain of this antenna is set to cmplx(1.,0.). The
other antenna gains are relative to the reference antenna.
The default is to use the original gains.
\par}
\module{fits}%
\noindent Conversion between MIRIAD and FITS image and uv formats
\newline \ 
\newline \abox{Responsible:} Robert Sault
\newline \abox{Keywords:} data transfer
\newline{\tenpoint\newline
FITS is a MIRIAD task, which converts image and uv files both from
FITS to Miriad format, and from Miriad to FITS format. Note that
because there is not a perfect correspondence between all information
in a FITS and Miriad file, some information may be lost in the
conversion step. This is particularly true for uv files.
\keyword{\bf IN}
Name of the input file (either a FITS or MIRIAD file name, depending
on OP). No default.
\keyword{\bf OP}
This can take the values ``uvin'', ``uvout'', ``xyin'', ``xyout'' or
``print''. These perform:
{\eightpoint\begintt
  uvin   Convert FITS uv file to Miriad uv file.
  uvout  Convert Miriad uv file to FITS uv file.
  xyin   Convert FITS image file to Miriad image file.
  xyout  Convert Miriad image file to FITS image file.
  print  Print out a FITS header.
\endtt}
\keyword{\bf OUT}
Name of the output file (either a MIRIAD or FITS file name, depending
on OP). If op=print, then this parameter is not required. Otherwise
there is no default.
\keyword{\bf LINE}
Line type of the output, when op=uvout. This is of the form:

{\eightpoint\begintt
  linetype,nchan,start,width,step

\endtt}
``Linetype'' is either ``channel'', ``wide'' or ``velocity''. ``Nchan'' is
the number of channels in the output.
\keyword{\bf ALTR}
Velocity information, used when op=uvin. If this is given, it
overrides any information that may be present in the FITS header.
It can be two values, namely the reference channel, and the
velocity at the reference channel (in km/sec).
\keyword{\bf STOKES}
MIRIAD cannot currently create a uv file with multiple Stokes
parameters. When reading in a uv file (op=uvin), FITS will convert
or extract a single polarization from the uv file. This parameter
can be ``i'', ``q'', ``u'', ``v'' or ``0''. Here ``0'' is like ``i'',
except that it assumes that the source is unpolarized.
\par}
\module{flint}%
\noindent Fortran source code verifier
\newline \ 
\newline \abox{Responsible:} Robert Sault
\newline \abox{Keywords:} tools
\newline{\tenpoint\newline
Flint is NOT intended to replace compiler checks.  Flint is blind
to much bad code that any compiler will pick up. Flint is intended 
to be used after a compiler accepts the source, but before you start 
debugging.

Checks it performs include:
{\eightpoint\begintt
  * undeclared variables.
  * variable declared but never used.
  * variables used before being assigned.
  * variables assigned to but not otherwise used.
  * names longer than 8 characters, or containing $ or _ characters.
  * lines longer than 72 characters, or containing an odd number of
    quote characters.
  * subroutine argument consistency: number, type and intent.

\endtt}
Areas for Improvement:
{\eightpoint\begintt
  * Improved error checking and handling.
  * Improve "initialisation checking" algorithm.
  * Consistency in common block definition. Checks could include 
    common block size, variable type matching, and even variable 
    name matching.
  * Better checks for illegal mixed expressions (i.e. mixing integers 
    and logicals).
  * Passing character expressions containing char*(*) variables.
  * Parsing of DATA statements is very crude.
  * SAVE statement usage

\endtt}
Things Flint does not understand:
{\eightpoint\begintt
  * ASSIGN
  * Assigned GOTO
  * PAUSE
  * ENTRY
  * alternate returns
  * many archaic i/o statements
  * much non-standard FORTRAN

\endtt}
Flint ignores:
{\eightpoint\begintt
  * IMPLICIT
  * EQUIVALENCE
  * FORMAT

\endtt}
Recognised FORTRAN extensions:
{\eightpoint\begintt
  * INTENT statement
  * DO/ENDDO, DOWHILE/ENDDO
  * # to start comments
  * tabs and full ascii character set.

\endtt}
There are some places where better use could have been made of stdio.
However Flint is intended to work on a number of systems, and so
it has been written to avoid some bugs in some non-UNIX stdio packages.
Systems on which Flint is believed to work are Berkeley UNIX (Sun, 
Alliant, Convex), UNICOS, VMS, Turbo-C (MS-DOS) and CTSS (hcc 
compiler).

\newline\newline ================
\newline Summary of Flags
\newline ================
\newline Flint takes a large number of flags, to attempt to keep the error 
messages that it generates down to manageable proportions. The 
command format is:

{\eightpoint\begintt
 flint [-acdfhjrsux2?] [-I dir] [-o file] [-l] file ...

  a    make crude list of all variables used
  c    Allow comments and continuation lines to be interwoven. 
       Normally flint flags this as an error.
  d    Do not insist that variables are always explicitly declared.
  f    Disable "line checks".
  h    Crude treatment of hollerith.
  j    Do not check if a variable has been initialised.
  r    Do not warn about seemingly redundant variables.
  s    Load the definitions of specific functions and FORTRAN-IV
       standard function.
  u    Do not worry about unused variables.
  x    Allow names longer than 8 characters.
  2    Flint performs two passes.
  ?    Print a message describing the flags.

  o    Generate output file giving subroutine definitions only. The 
       next command line argument gives the output file name.
  l    The following file is to be processed in "library mode". This
       means that the file is not echoed to flint.log, and that errors
       detected are to be ignored.
  I    add a directory to search for include files 

\endtt}
\newline\newline ================
\newline Basic Workings
\newline ==============
\newline Flint maintains two (hash) tables which contain all Flint info about
variables and subroutines. Almost always, these hash tables are 
accessed through the set\_variable and set\_routine subroutines. The 
caller passes in whatever it knows about the variable or routine, 
and the set\_? routine passes back the accumulated knowledge about 
the variable or routine. set\_variable and set\_routine are responsible 
for generating many warnings. 

\newline\newline ================
\newline Determining Intent
\newline\newline ==================
\newline Flint attempts to uncover the intent of subroutine arguments. It 
gleans information from both calls to the routine and the source of 
the routine itself (if available).

The following rules are used when analysing a call to a routine:
\newline\newline * Arguments which are passed in as constants, expressions or parameters
  are input! Arguments which are input dummy arguments to the current
  routine are also deemed to be inputs. These rules are reliable.
\newline\newline * Arguments which are variables which have not been initialised are
  assumed to be output. Also arguments which are output dummy arguments
  to the current routine, and have not yet been initialised, are also
  deemed to be outputs. These rules depend on the accuracy of the
  initialisation checking algorithm, which can be inaccurate. These
  rules are disabled if initialisation checking is disabled.
\newline\newline The following rules are used when analysing the source of the routine.
\newline\newline * Dummy arguments which are used as input only are clearly input. This rule is reliable.
\newline\newline * Dummy arguments which are assigned to before they are used, are 
  deemed as output. This rule depends on the initialisation checking 
  algorithm, and is turned off, if initialisation checking is turned 
  off.
\newline\newline * Dummy arguments which are used and then assigned to are deemed to
  be input/output. 

\newline\newline What goes on when dummy arguments are passed to subroutines produces 
much recursive thought. There are some instances where Flint has to 
discard some hard won information, because there are too few flags 
to describe all the contortions that can occur.

\newline\newline ================
\newline Flags and Hash Tables
\newline =====================
\newline EACH entry in the ``vhash'' hash table consists of a name and set of 
flags. These are mostly just variables, but they also contain entries 
for each EXTERNAL and each function or subroutine called (except the
FORTRAN intrinsics?).

Each entry in the ``rhash'' hash table is a subroutine or function 
definition. Each definition consists of the name, flags for the 
routine, the number of subroutine arguments, and an array of flags, 
one for each argument. Entries for FORTRAN intrinsics appear in this 
table.

The meaning of most flags (I hope) is fairly clear, but
F\_IN,F\_OUT,F\_PIN,F\_POUT require more attention. For a variable,
F\_IN and F\_OUT is set if the variable has been used as source or
destination of an operation, respectively. F\_PIN is set for subroutine
arguments, indicating that this variable may have been passed in by the
caller. If Flint determines that this is not so, then this flag is
turned off. F\_POUT indicates that this variable may have been passed
out by a subroutine. 

For a subroutine or function, these flags are used only in the
vhash table. F\_IN and F\_OUT indicate that the routine
has been called. F\_PIN is always set if it was a dummy argument.
F\_POUT is set if it was passed to a subroutine.

For a routine argument (in the ``rhash'' table), F\_IN and F\_OUT means
passed in and passed out. An absence of these indicates that Flint 
does not know.                                                       
\par}
\module{gainerr}%
\noindent Add dummy gains to a visibility file.
\newline \ 
\newline \abox{Responsible:} Robert Sault
\newline \abox{Keywords:} uv analysis
\keyword{\bf VIS}
The input visibility file. This is used to determine the number
of antennas and the time range. No default.
\keyword{\bf INTERVAL}
Averaging time interval, in minutes. Default is 5.
\keyword{\bf ANTPNOIS}
Rms phase noise, in degrees. Default is 0.
\module{hatfft}%
\noindent Timing and test of fft algorithms.
\newline \ 
\newline \abox{Responsible:} Mel Wright
\newline \abox{Keywords:} fft, timing
\newline{\tenpoint\newline
HATFFT is a Miriad task to get timing and test fft algorithms.
\keyword{\bf DATA}
Input data for the fft is generated internally. Possible values
are data=0 for random noise, and data='value'. The real and
imaginary parts are both set to random noise, or 'value'.
Default is 0.
\keyword{\bf MODE}
Mode can be 'mels' 'bobs' or 'diff'. Default is 'mels'
\keyword{\bf SIZE}
This is the size of the transform. Three numbers can be given
for the minimum, maximum, and step size. All numbers must be
powers of 2. Default is 128,2048,4 which does the transform for
sizes of 128, 512 and 2048.
\keyword{\bf NBASE}
Each transform is repeated nbase times. Maximum is 36. Default is 1.
\keyword{\bf SIGN}
The sign of the transform. Must be 1 or -1. Default is 1.
\par}
\module{hatmap}%
\noindent Convert old Hat Creek 2D images to MIRIAD cube
\newline \ 
\newline \abox{Responsible:} Mel Wright
\newline \abox{Keywords:} data transfer
\newline{\tenpoint\newline
HATMAP is a MIRIAD task which converts old Hat Creek 2D maps
into a MIRIAD cube
\keyword{\bf IN}
Filename of the first Hat Creek 2D map. Filenames must be of the
form nameNNN.ext where NNN is a 1 to 3 digit integer. No default.
\keyword{\bf OUT}
The output MIRIAD image. No default.
\keyword{\bf CHANINC}
The channel increment between Hat Creek maps. Increments the integer
NNN in the map filename. Default=1.
\keyword{\bf NCHAN}
The number of 2D Hat Creek maps to read. Default=1.
\keyword{\bf SCALE}
Scaling factor to normalize the maps to Jy/beam, and the beam to unit
amplitude. Enter the beam maximum. Default=1.
\par}
\module{hcconv}%
\noindent Convert Hat Creek uv format into MIRIAD uv format
\newline \ 
\newline \abox{Responsible:} Peter Teuben
\newline \abox{Keywords:} data transfer
\newline{\tenpoint\newline
Converts Hat Creek data on a Unix system to MIRIAD format.
The Hat Creek data are usually obtained by exploding a 
backup saveset. Giving no arguments at all will turn program
in auto scanning mode...                                  
As said, HCCONV only runs on SUN Unix (not CRAY); on VMS
use the program UVHAT.                                    
\keyword{\bf IN}
The Hat Creek data file(s). In no input given, autoscanning
for files of the form '*.ddmmm'. Formally in Unix wildcard
notation: *.[0-9][0-9][a-z][a-z][a-z] Default: empty.     
\keyword{\bf OUT}
Name of the MIRIAD data set to be created. If no name provided
hcconv chooses a name. See the ``mode'' keyword for how this 
can be done. If multiple input files, and one output file,
the input files are merged into one output file. Make sure the
type of the input files (auto/cross correlation) are all the 
same.
Default: empty, meaning auto-filename generation.         
\keyword{\bf MODE}
Option denoting how the output filename is formed from the
input filename. Options are: ``underscore'' (replaces the dot
by an underscore), ``strip'' (strips name from the dot onwards).
Default: ``underscore''                                   
\par}
\module{hermes}%
\noindent Make model images for Mercury.
\newline \ 
\newline \abox{Responsible:} Mel Wright
\newline \abox{Keywords:} image analysis, planets.
\newline{\tenpoint\newline
First calculates a Hermographic brightness temperature map,
then uses interpolation and coordinate rotations to find
the corresponding map in geocentric RA and DEC offsets from
the sub-earth point.  Next, the Fresnel reflection coef-
ficients are applied to obtain the model map.
\keyword{\bf IN}
Parameter file. Default is ``hermes.in''.
\keyword{\bf IMSIZE}
Image size. Default is 64.
\keyword{\bf LOG}
Output log file. Default is ``hermes.log''.
\keyword{\bf OUT}
Output image. Units JY/PIXEL. No default.
\par}
\module{histo}%
\noindent Find statistics of image and plot simple histogram
\newline \ 
\newline \abox{Responsible:} Mel Wright
\newline \abox{Keywords:} map analysis
\newline{\tenpoint\newline
HISTO is a Miriad task which finds the minimum, maximum, mean and
rms deviation from the mean for the selected region of the image.
It also displays a simple histogram which is useful for determining
the clip levels for clean and selfcal.
\keyword{\bf IN}
The input file name. No default.
\keyword{\bf REGION}
The region of interest of the input. Default is the entire input.
See the Users Manual for instructions on how to specify this.
\keyword{\bf RANGE}
The range in pixel values over which the histogram is calculated.
The default is the image minima and maxima.
\keyword{\bf NBIN}
The number of bins used in the histogram. Default is 16.
\par}
\module{imcat}%
\noindent Concatenate several images to one cube
\newline \ 
\newline \abox{Responsible:} Mel Wright
\newline \abox{Keywords:} map combination
\newline{\tenpoint\newline
IMCAT is a MIRIAD task to concatenate several images together,
along the third dimension (generally the frequency or velocity
dimension).
\keyword{\bf IN}
The input images. Several file names can be entered, separated
by commas. No default.
\keyword{\bf OUT}
The output image. No default.
\par}
\module{imdiff}%
\noindent Shift and expand an image to make it match another.
\newline \ 
\newline \abox{Responsible:} Robert Sault
\newline \abox{Keywords:} image-analysis
\keyword{\bf IN1}
The first input image. This is considered to be the ``reference''
image. No default.
\keyword{\bf IN2}
The second input image. This is considered to be the image that must be
adjusted to make it like IN1. No default.
\keyword{\bf RESID}
An output image consisting of the difference between IN1 and the
adjusted form of IN2. It is generally best to look at the residuals,
to check that they are noise-like. The residuals are defined as:
{\eightpoint\begintt
  
Resid(x,y) = In1(x,y) - Amp*In2(Expand*x+Xshift,Expand*y+YShift) - Offset
  
\endtt}
(Here X and Y are relative to IN2's reference pixel). The default is
not to create the residuals.
\keyword{\bf REGION}
The region of interest. Currently this must be a rectangular
region. The default is the entire image.
\keyword{\bf VARY}
Flags which indicate which parameters are to be optimised. VARY can
consist of the characters ``A'' (amplitude), ``O'' (offset), ``X''
(x-shift), ``Y'' (y-shift) or ``E'' (expansion). The order is not
important. For example VARY=AOXY tells IMDIFF to find optimum
amplitude, offset and shifts (x and y), but not vary the expansion.
The default is AOXYE, that is to allow everything to vary. 
\keyword{\bf GUARD}
The task cannot work right to the edge of the image, because the
shifts and expansions might mean that there is no corresponding point
between images near the image edges. Consequently a guard band is
needed. GUARD gives the width of this band, in pixels. In the output
residuals, this band will be set to zero. Default is 10. 

The following five parameters give the initial estimates for the
various parameters that IMDIFF is trying to find. If the parameters
are fixed, then IMDIFF will not change them from their initial
settings. 
\keyword{\bf XSHIFT}
This gives the initial value, in pixels, of the x-shift.
This should be only +/- a few pixels. Default is 0.
\keyword{\bf YSHIFT}
This gives the initial value, in pixels, of the y-shift. Default is 0.
\keyword{\bf EXPAND}
This gives the initial expansion. Only small expansions can be
handled (e.g. approximately in the range 0.95 to 1.05). This should
be adequate for proper motion studies. Default is 1.
\keyword{\bf AMP}
This gives the initial amplitude scale factor. Default is 1.
\keyword{\bf OFFSET}
This gives the initial dc offset to apply to IN2. The default
is 0.
\module{imgen}%
\noindent All-purpose image manipulator/creator
\newline \ 
\newline \abox{Responsible:} Mel Wright
\newline \abox{Keywords:} utility, map manipulation
\newline{\tenpoint\newline
IMGEN is a MIRIAD task which modifies an image, or creates a new
image. 
\keyword{\bf IN}
The input image, which is to be modified. The default is a map
with consists entirely of zeros. 
\keyword{\bf OUT}
The output image. No default. 
\keyword{\bf FACTOR}
Factor to multiply the input image by. This is meaningless if no
input image is given. The default is 1. 
\keyword{\bf IMSIZE}
If not input image is given, then this determines the size, in
pixels, of the output image. Either one or two numbers can be
given. If only one number is given, then the output is square.
The default is 256 pixels square. 
\keyword{\bf OBJECT}
This determines the type of object added to the input image.
Valid objects are ``gaussian'' (elliptical gaussians), ``disk''
(elliptical disks), ``j1x'', (a J1(x)/x function), ``point''
(a point source), ``noise'' (gaussian noise) and ``level'' (a dc
level). ``Gaussian'' is the default. 
\keyword{\bf SPAR}
Parameters which give the characteristics of the object. For
gaussian, j1x and disk, this consists of up to three numbers, being
the amplitude (default 1), the object full-width at half-max in
RA (default 1), and full-width half-max in DEC (default is width
in RA). Widths are given in arcseconds. For points and levels,
only one number, the amplitude, is needed, which has a default of
1. 
\keyword{\bf XY}
This is two numbers giving the relative offset of the center of
the object, in arcseconds. This is relative to the reference
point of the input, or the image center if there is no input.
This is ignored if the object is a level. The default is 0.
\keyword{\bf CELL}
The increment between pixels, in arcseconds. This is only used if
there is no input map. The default is 1 arcsec.
\par}
\module{imhead}%
\noindent List items and pixel values from an image.
\newline \ 
\newline \abox{Responsible:} Jim Morgan
\newline \abox{Keywords:} utility
\newline{\tenpoint\newline
Imhead lists a Miriad image header in detail.
Conversions of headers to astronomical coordinates are made.
\keyword{\bf IN}
Input image name.  There is no default file name.
\keyword{\bf LOG}
The output log file.  The default is to output to the terminal.
\par}
\module{imlist}%
\noindent List items and pixel values from an image
\newline \ 
\newline \abox{Responsible:} Mel Wright
\newline \abox{Keywords:} image analysis
\newline{\tenpoint\newline
IMLIST lists a Miriad image. It will list the header, statistics,
selected regions of the data, and the history. RA and DEC are converted
to astronomical coordinates. The default output to the terminal can be
terminated.
\keyword{\bf IN}
Input image name. No default.
\keyword{\bf OPTIONS}
The options are 'head' 'data' 'history' and 'stat'. Several options
can be given, separated by commas. For example, the default is:
{\eightpoint\begintt
  % imlist options=head
\endtt}
which lists the image header.
{\eightpoint\begintt
  % imlist options=head,data,history,stat
\endtt}
lists the image header, the data, history, and image statistics.
options=data lists the image with the specified format for each plane.
options=stat lists the total flux, max, min, and rms for each plane.
\keyword{\bf REGION}
Region of image to be listed. E.g.
{\eightpoint\begintt
  % imlist  options=data region=relpix,box(-4,-4,5,5)(1,2)
\endtt}
lists the center 10 x 10 pixels of image planes 1 and 2.
\keyword{\bf FORMAT}
Format for output, e.g., 1pe11.4, f5.2, 1pg12.5, default is 1pe10.3
The size of region possible to list depends on the format. 10 columns
of f5.1 fills an 80 character line.
\keyword{\bf LOG}
The output log file. The default is the terminal.
\par}
\module{implot}%
\noindent Multi-panel contour or grey scale plot of a Miriad Image.
\newline \ 
\newline \abox{Responsible:} Mel Wright
\newline \abox{Keywords:} image analysis, plotting
\newline{\tenpoint\newline
IMPLOT makes multi-panel plots of a Miriad Image. It works with
either contours or grey-scale plotting of individual maps. The plots
are annotated with various map-header variables on the right side
of the page.
\keyword{\bf IN}
Input image name. No default.
\keyword{\bf DEVICE}
Plot device/type (e.g. /retro, plot.plt/im for hardcopy on a Vax,
or  /sunview, plot/ps for harcopy on a Sun).  No default.
See the users guide for more details.
\keyword{\bf REGION}
Region of image to be plotted. E.g.
{\eightpoint\begintt
  % implot region=relpix,box(-4,-4,5,5)(1,2)
\endtt}
plots the center 10 x 10 pixels of image planes 1 and 2.
The default is the whole Image.
\keyword{\bf UNITS}
The units for the plot axes, labels and contour levels.
{\eightpoint\begintt
 p means pixels with respect to the center pixel (default)
 s means arcseconds with respect to the center pixel
 a means absolute coordinates (RA in hours, DEC in degrees)
\endtt}
note: RA increases to the left of the map; pixels to the right.
\keyword{\bf CONFLAG}
The contour flag is made up of letters denoting contour options:
{\eightpoint\begintt
 p means percent of max
 a means absolute steps
 i means conargs is list of absolute contours
 ip means conargs is list of percentage contours
 n means include negative contours
 g means gray scale (may be added to contours)
\endtt}
The default is conflag=p.
\keyword{\bf CONARGS}
Arguments for conflag; usually one number but may be multiple for
i option.  No default.
\keyword{\bf NBYN}
Sets the size of the square grid of panels.  Values are 1 to 4.
The default is 1.
\keyword{\bf BEAMQUAD}
Quadrant placement (1-4) for the Clean beam picture in the first
plot; 0 means no beam.  The default is 0.
\keyword{\bf LWIDTH}
Plotting line width, the default is 1.
\keyword{\bf LOG}
The output log file. The default is the terminal.
\keyword{\bf MODE}
Mode can be 'inter' or 'batch'. Default is batch. In mode=inter,
the user is prompted for options after each plot. The options are
apply to the current plot. Type the first character ( and {\tt <}cr{\tt >} )
to select an option. The cursor handling is device dependent, and
the character and {\tt <}cr{\tt >} may need to given several times.
Options:
{\eightpoint\begintt
  Help - decriptions of options.
  Box - draw a box to define a region of interest in the logfile.
  Comment - enter a comment into the logfile.
  Position - display cursor position. (if device supports a cursor)
  Quit - stop processing and exit from task.
  End - end of options, goto next plot.
\endtt}
\par}
\module{imspect}%
\noindent Make spectrum from a Miriad image.
\newline \ 
\newline \abox{Responsible:} Mel Wright
\newline \abox{Keywords:} image analysis and display.
\newline{\tenpoint\newline
IMSPECT makes an spectrum of the velocity or frequency axis of a
Miriad image. The spectrum is integrated, or averaged over the
region specified for the other image axes. The output spectrum
can be plotted and/or written out as an ascii file for further
analysis.
\keyword{\bf IN}
The input image. vxy and xyv images are acceptable inputs. 
No default.
\keyword{\bf REGION}
Region of image to be averaged. E.g.
{\eightpoint\begintt
  % imspect  region=relpix,box(-4,-4,5,5)(10,50) for an xyv
\endtt}
image, makes a spectrum of image planes 10 to 50 integrated
over the center 10 x 10 pixels in the x-y plane. For a vxy
image where the x and y size is 128x128, the corresponding
region=box(10,61,50,70)(61,70). Box refers image axes 1 and 2
for either vxy or xyv images. The default is the entire image.
\keyword{\bf XAXIS}
The x-axis can be plotted as 'channel' or the units in the image.
The default which is whatever units are in the header.
\keyword{\bf YAXIS}
If 'average' then the pixels enclosed in the x-y area specified
are averaged. If 'sum' they are just summed. Default is 'average'
\keyword{\bf SMLEN}
Hanning smoothing length (an odd integer {\tt <} 15) Default is
no smoothing (smlen = 1).
\keyword{\bf DEVICE}
Plot device/type (e.g. /retro, filename/im for hardcopy on a Vax,
/sunview, filename/ps for harcopy on a Sun). The default is no plot.
To print the hardcopy on the Vax use:
{\eightpoint\begintt
  $PGPLOT filename
\endtt}
To print the postscript file on the Sun:
{\eightpoint\begintt
  %lpr filename
\endtt}
\keyword{\bf LOG}
Write spectrum to this ascii file. Default is no output file.
\keyword{\bf COMMENT}
A one-line comment which is written into the logfile.
\par}
\module{imsub}%
\noindent Extract subportion of an image
\newline \ 
\newline \abox{Responsible:} Mel Wright
\newline \abox{Keywords:} map manipulation
\newline{\tenpoint\newline
IMSUB is a MIRIAD task which extracts a subportion of an image.
\keyword{\bf IN}
The input image. No default.
\keyword{\bf OUT}
The output image. No default.
\keyword{\bf REGION}
The region of interest. The default is the entire input image.
See the Users Manual for instructions on how to specify this.
\keyword{\bf INCR}
Increment to be used along each axis. Default is 1.
\par}
\module{imwrite}%
\noindent Make a Miriad Image from a foreign dataset.
\newline \
\newline \abox{Responsible:} Mel Wright
\newline \abox{Keywords:} image conversion, analysis.
\newline{\tenpoint\newline
Imwrite makes a Miriad image from data provided by the user.
This task provides a template with subroutines to read an image,
header, history and pixel values from ascii files. These items
can also be made up as subroutines provided by the user.
The task can readily be adapted to write the image using user
provided subroutines.
\keyword{\bf HEADER}
Parameter file containing image header items. The routine provided
here reads the header items from an ascii file. The default file is
no header file.
\keyword{\bf HISTORY}
File containing the image history. Default is no history file.
\keyword{\bf IMAGE}
File containing the image pixel values. Default is no image file.
\keyword{\bf IMSIZE}
Image size. 1 to 3 values nx,ny,nz. Default ny=nx, and nz=1.
\keyword{\bf LOG}
log file. Default is the terminal.
\keyword{\bf OUT}
Output image. No default.
\par}
\module{invert}%
\noindent Transform visibility data into a map
\newline \ 
\newline \abox{Responsible:} Robert Sault
\newline \abox{Keywords:} map making
\newline{\tenpoint\newline
INVERT is a MIRIAD task which forms a map from visibility data
via a convolutional gridding and FFT approach. Only one Stokes
map can be made at a time, though multiple frequency channels can
be made in one go.

By default the weight used for each visibility is proportional to
the integration time divided by the density of visibility points.
See the SUP parameter to adjust how the density function is calculated.
See the OPTIONS parameter to weight according to the nominal system
temperature.
\keyword{\bf VIS}
Input visibility data file. No default.
\keyword{\bf MAP}
Output map file name. Default is not to make the map. 
\keyword{\bf BEAM}
Output beam file name. Default is not to make the beam.
\keyword{\bf IMSIZE}
The size of the output image in pixels. Two values can be given,
giving the image size in RA and DEC. If only one value
is given, the output is a square image. At least one value must
be given. INVERT increases the sizes in RA and DEC to the next
power of two (if they were not initially a power of 2). 
\keyword{\bf SHIFT}
The shift (arcsec) to move the  map center from the phase center.
Positive values result in the map center being shifted to the
North and East of the phase center.  If one value is given, both
RA and DEC are shifted by this amount.  If two values are given,
then they are the RA and DEC shifts. 
\keyword{\bf CELL}
Image cell size, in arcsec. If two values are given, they give
the RA and DEC cell sizes. If only one value is given, the cells
are made square. 
\keyword{\bf FWHM}
Full width at half maximum, in arcsec, of a gaussian which
represents the typical resolution of interest in the source. If
two values are given, they are used as the fwhm in RA and DEC
respectively. If one value is given, it is used for both RA and
DEC. This parameter is used to determine the uv-taper to apply to
the data to optimize the signal to noise for sources of that
particular angular size. Set FWHM to zero if you want the full
resolution of the data. Default is 0. 
\keyword{\bf SUP}
Sidelobe suppression region, given in arcseconds. This parameter
determines the weighting scheme used. As uniform weighting
attempts to suppress sidelobes in the entire field, uniform
weighting corresponds to SUP=x, where x is the field size.
Natural weighting gives the best signal to noise ratio, at the
expense of no sidelobe suppression. Natural weighting corresponds
to SUP=0. Values between these extremes give a tradeoff between
signal to noise and sidelobe suppression. The default is to use
``uniform'' weighting. 
\keyword{\bf LINE}
Line type of the data. This is a string, followed by up to four
numbers, viz: 
{\eightpoint\begintt
  linetype,nchan,start,width,step
\endtt}
where ``linetype'' is one of ``channel'', ``wide'' and ``velocity''.
\keyword{\bf REF}
Line type of the reference channel, specified in a similar to the
``line'' parameter. Specifically, it is in the form:
{\eightpoint\begintt
  linetype,start,width
\endtt}
Default is no reference channel.
\keyword{\bf SELECT}
This allows a subset of the uv data to be used in the mapping
process. See the Users Manual for information on how to specify
this parameter. The default is to use all data.
\keyword{\bf OPTIONS}
This gives extra processing options. Currently there are two
possible values:
{\eightpoint\begintt
  systemp  Weight each visibility proportionally to 1/systemp**2.
  nocal    Do not apply gains table calibration to the data.
\endtt}
\par}
\module{itemize}%
\noindent List information about image dataset
\newline \ 
\newline \abox{Responsible:} Peter Teuben
\newline \abox{Keywords:} utility
\newline{\tenpoint\newline
ITEMIZE is a MIRIAD task which dumps a dataset or an item within a
dataset. If the input name is an item, then the contents of the
item (element for element) are written to the screen. If the input
name represents a dataset, then a summary of the items within the
dataset are given. If multiple input files are given, it acts as
a filescanner, and minimal information is given.
\keyword{\bf IN}
The name of either a dataset, or an item within a data set or a 
wildcard. For example:
{\eightpoint\begintt
  % itemize in=dataset
\endtt}
or
{\eightpoint\begintt
  % itemize in=dataset/item
\endtt}
or
{\eightpoint\begintt
  % itemize in='*'
\endtt}
For example, to print the history information of file ``cm'', use
{\eightpoint\begintt
  % itemize in=cm/history
\endtt}
When a dataset name is given, itemize summarizes the contents
of the entire dataset. When an item name is also given, then
itemize dumps the entire contents of the item (in accordance
to the index and format keywords).
When a wildcard expands in more than one file, itemize only 
checks to see if the specified file is a miriad dataset, and
attempts to say a few intelligent things about it.
\keyword{\bf LOG}
The name of the output listing file. The default is the users
terminal.
\keyword{\bf INDEX}
When dumping an entire item, ``index'' specifies the range of elements
to dump. The default is the entire item. For example, to print out
lines 10 through 20 of the history item, use:
{\eightpoint\begintt
  % itemize in=cm/history index=10,20
\endtt}
\keyword{\bf FORMAT}
When dumping an entire item, this gives the FORTRAN format specifier
to be used. For example, when dumping a real item, you may set:
{\eightpoint\begintt
  format=8e15.7
\endtt}
The default varies according to the data type.
\keyword{\bf SCAN}
Scan mode. If itemize is running in file scan mode, the default 
will scan for all files. If ``scan=miriad'' it will only report
on what it thinks are miriad data files. In regular item/dataset
reporting mode, scan has no meaning.
\par}
\module{linmos}%
\noindent Linear mosaicing of datacubes
\newline \ 
\newline \abox{Responsible:} Mel Wright
\newline \abox{Keywords:} map combination
\newline{\tenpoint\newline
LINMOS is a MIRIAD task which performs a simple linear mosaicing of
input cubes, to produce a single output cube. If only a single
input cube is given, LINMOS essentially does primary beam correction
on this input. When several, overlapping, inputs are given, then
LINMOS combines the overlapping regions in such a way as to minimize
the expected mean error in the output.
\keyword{\bf IN}
This gives the names of the input cubes. Many cubes can be given.
There is no default. Inputs do not need to be on the same grid system
on the x (RA) and y (DEC) dimensions. However, if they are not,
linear interpolation is performed to regrid some of the inputs. The
intensity units of all the inputs are assumed to be the same.
\keyword{\bf OUT}
The name of the output cube. No default. The grid system of the first
input is used as the grid system of the output.
\keyword{\bf PBFWHM}
Primary beams are modeled by circular gaussians. PBFWHM gives the
full width at half maximum of the primary beam, in arcseconds. There
should be as many values of this as there are input cubes. If a
value is not positive or not given, then LINMOS uses the pbfwhm from
the map header or calculates pbfwhm from the telescop name and
sky frequency for Hat Creek or VLA antennae (if the file header
indicates that this data was observed with one of these arrays), or
assumes a single dish telescope, with no primary beam correction.
\keyword{\bf RMS}
The rms noise levels in the input cubes. Only the relative sizes are
of importance. There should be as many values as there are input
cubes. The default is that all inputs have the same noise level.
\keyword{\bf SIGNAL}
An rms signal level. Only one value is required. Only the relative
size of this to the ``noise'' parameters is of importance. The
default value is infinity, which can allow the correction factor
to become excessively large. By setting a finite value, the
correction factor tends to taper off, and thus prevents the output
being swamped by noise amplification. 
\par}
\module{listobs}%
\noindent Makes a summary of a set of observations.
\newline \ 
\newline \abox{Responsible:} Lee Mundy
\newline \abox{Keywords:} utility
\newline{\tenpoint\newline
LISTOBS makes a summary of a set of observations.  Parameters of
interest to the observer are pulled from one or many files, and
printed to a log file.  A time ordered summary of the sources
observed is compiled.  The primary use of this program is to
create a summary of the instrument setup and all observations
made during a track.  Use wild cards or an include file to specify
all files relevent for your observations.
\keyword{\bf IN}
The name of the input image. No default.
\keyword{\bf LOG}
Output device. (default is standard user output)
\par}
\module{maths}%
\noindent Mathematical operations on images and image data
\newline \ 
\newline \abox{Responsible:} Peter Teuben
\newline \abox{Keywords:} utility, map combination, map manipulation
\newline{\tenpoint\newline
MATHS is a MIRIAD task which performs arithmetic expressions on a
number of images. The expression to be performed is given in a
FORTRAN-like syntax, and can consist of operators, real constants
and FORTRAN functions. Normal FORTRAN precedence applies.
Operators can be +, -, * and /, and all logical and relational
operators (e.g. .and. .or. .not. .gt. .ge. etc). In MATHS convention,
a positive value is considered TRUE, and a negative or zero value
is considered FALSE. Functions
appear only in the generic, rather than specific forms. For example
use ``log10'' rather than ``alog10'', and ``max'' rather than
``amax1''. Integers and double precision constants are converted to
reals. File names take the place of variables, and the expression
is evaluated on each pixel of the image. When there is more than
one file name in the input expression, the expression is evaluated
at corresponding pixels of the input images. For example to average
image ``fred'' with image ``bill'', use:

{\eightpoint\begintt
  exp=(fred+bill)/2

\endtt}
When a file name starts with a numeric character, or contains a
character which might be confused with an operator the file name
should be bracketed by angular ( {\tt <} and {\tt >} ). For example:

{\eightpoint\begintt
  exp=(<2ndtry>+bill.dat)/2

\endtt}
Files cannot take the name ``x,'' ``y'' or ``z''. MATHS interprets
these as being the 3 independent variables of an image, which
vary linearly between the limits set by the XRANGE, YRANGE and
ZRANGE parameters. The user chooses the meaning of these units.
For example, to create one cycle of a two dimensional sine wave
along the x and y coordinate axes use:

{\eightpoint\begintt
  exp=sin(x)*sin(y) xrange=-3.14,3.14 yrange=-3.14,3.14

\endtt}
In addition to the expression, MATHS also allows the user to specify
a ``mask expression''. MATHS main expression is only evaluated at
pixels where the ``mask expression'' is TRUE or positive valued. MATHS
does not check for divide by zero, logs of a negative number
or any similar problem. It will probably crash if this is attempted.
Consequently, when performing potentially dangerous operations, it
is best to guard the main expression by masking out dangerous
situations.  The mask expression can also be used to prevent the
calculation where doing so would be undesirable for other reasons
(e.g. where the signal is too weak to get meaningful results).

For example:

{\eightpoint\begintt
   exp=sqrt(fred) mask=fred.gt.0
\endtt}
\keyword{\bf EXP}
The expression to be evaluated.
\keyword{\bf MASK}
The mask expression. The expression given by ``exp'' is evaluated
only at those pixels where the mask expression is TRUE or
positive valued. Pixels, which fail this test, are marked as blank
in the output image.
\keyword{\bf REGION}
The region of interest in the input images.
\keyword{\bf OUT}
The name of the output image.
\keyword{\bf IMSIZE}
The output image size. This is used only if there is no input
images (i.e. the expression consists of a function of ``x'' and
``y'' only). No default.
\keyword{\bf XRANGE}
When ``x'' is present in the input expression, then the x variable
is varied linearly between the two limits set by XRANGE.
\keyword{\bf YRANGE}
When ``y'' is present in the input expression, then the y variable
is varied linearly between the two limits set by YRANGE.
\keyword{\bf ZRANGE}
When ``z'' is present in the input expression, then the z variable
is varied linearly between the two limits set br ZRANGE.
\par}
\module{maxen}%
\noindent Maximum Entropy deconvolution
\newline \ 
\newline \abox{Responsible:} Robert Sault
\newline \abox{Keywords:} deconvolution
\newline{\tenpoint\newline
MAXEN is a MIRIAD task which performs a maximum entropy deconvolution
algorithm on a cube.
\keyword{\bf MAP}
The input dirty map, which should have units of Jy/beam. No default.
\keyword{\bf BEAM}
The input dirty beam. No default
\keyword{\bf MODEL}
An initial model of the deconvolved image. This could be the output
from a previous run of MAXEN or any other deconvolution
task. It must have flux units of Jy/pixel. The default is a flat
estimate, with the correct flux.
\keyword{\bf DEFAULT}
The default image. This could be the output
from a previous run of MAXEN or any other deconvolution
task. It must have flux units of Jy/pixel. The default is a flat
estimate, with the correct flux.
\keyword{\bf OUT}
The name of the output map. The units of the output will be Jy/pixel.
It can be input to RESTORE, MAXEN (as a model, to continue the
deconvolution process), or to SELFCAL (for self-calibrating
visibility data).
\keyword{\bf NITERS}
The maximum number of iterations. The default is 20.
\keyword{\bf REGION}
This specifies the region to be deconvolved. See the Users Manual
for instructions on how to specify this. The default is the inner
quarter of all planes.
\keyword{\bf MEASURE}
The entropy measure to be used, either ``gull'' (-p*log(p/e)) or
``cornwell'' (-log(cosh(p))).
\keyword{\bf ALPHA}
An estimate of the alpha Lagrangian multiplier. Default is 0.
You would probably never set this except when restarting a MAXEN run.
In this case, set it to the value of ALPHA and BETA that the previous
run finished with.
\keyword{\bf BETA}
An estimate of the ``Beta'' Lagrangian multiplier. See ALPHA.
\keyword{\bf TOL}
Tolerance of solution. There is no need to change this from the
default of 0.01.
\keyword{\bf Q}
An estimate of the number of points per beam. MAXEN can usually
come up with a pretty good, image dependent estimate.
\keyword{\bf RMS}
The rms noise, in Jy/beam, in the dirty map. No default. The
convergence and behavior of MAXEN depends strongly on the value
of this parameter.
\keyword{\bf FLUX}
The flux of the source. If FLUX is given as a positive
number, then the resultant image is constrained to have this
flux. If FLUX is negative, then abs(FLUX) is used as an estimate
of the source flux only and the actual flux is allowed to vary.
If FLUX=0, then MAXEN makes a guess at the source flux. Default is
FLUX=0, though you should give MAXEN more information if at all
possible. On the other hand, a bad value here can cause MAXEN to
blow up.
\keyword{\bf MESSAGE}
Message level during execution. Message=0 turns off all output
during the iterations. Message=1 gives the Rms, Flux, and Gradient.
Message=2 also gives the Lagrange multipliers and the step size.
The default is message=1.
\par}
\module{miriad}%
\noindent General program to run miriad programs
\newline \ 
\newline \abox{Responsible:} Peter Teuben
\newline \abox{Keywords:} tools
\newline{\tenpoint\newline
Miriad is a command-line interface to run Miriad tasks.

Some important Miriad commands are:

{\eightpoint\begintt
    inp taskname ..... preview input parms to task taskname
    set parm=value ... set task parameter parm to value
    unset parm ....... unset task parameter parm
    help taskname .... show on-line docs for task taskname
    go taskname ...... run task taskname
    view taskname .... use editor to set parms for task taskname
    quit ............. exit immediately
    exit ............. exit, but save all parm settings
    ? ................ overview of all miriad commands

\endtt}
Any unrecognized commands are passed on to the system for execution.
Thus, the command ``ls -l \$HOME'' will return a listing of the files
in the user's home directory on Unix.  Unix aliases are not understood
though.
\par}
\module{mirtool}%
\noindent MIRIAD SunView interface (SUN only)
\newline \ 
\newline \abox{Responsible:} Peter Teuben
\newline \abox{Keywords:} tools
\newline{\tenpoint\newline
MIRTOOL (SUN only) is MIRIAD's SunView interface to
run MIRIAD tasks.  It is available only when running
SunView.

To invoke MIRTOOL, enter ``mirtool \&'' in any SunView
window that is on the machine's console (ie, a window
that is not being used to remotely log onto another
machine via ``rlogin'' or ``telnet'').
\par}
\module{moment}%
\noindent Calculate moments of a Miriad image.
\newline \ 
\newline \abox{Responsible:} Mel Wright
\newline \abox{Keywords:} image analysis
\newline{\tenpoint\newline
MOMENT calculates the nth moment of a Miriad image. Currently only
the moment of axis 1 or axis 3 is calculated for 3 dimensional images.
To obtain the moments for other axes use the task REORDER to reorder
the axes.
\keyword{\bf IN}
The input image. No default.
\keyword{\bf REGION}
The region of the input image to be used.
\keyword{\bf OUT}
The output image. No default.
\keyword{\bf MOM} \newline \ 
{\eightpoint\begintt
1      Average intensity. (units are same as individual channels)
0      Integrated intensity. (e.g. units Jy x km/s)
1      1st moment = sum(I*v)/sum(I)
2      dispersion = sqrt(sum(I*(v-M1)**2)/sum(I))
\endtt}
The v(i) are the axis values (e.g. velocities).  M1 is the first
moment.  The moment is calculated independently for each pixel
with absolute value greater than clip. Default = 0.
\keyword{\bf AXIS}
The axis for which the moment is calculated. Default = 3.
\keyword{\bf CLIP}
Pixels with absolute value .lt. clip are not included
in calculating the average. Default = 0.
\par}
\module{msss}%
\noindent MIRIAD version of the AIPS Sun Screen Server (SUN only)
\newline \ 
\newline \abox{Responsible:} Jim Morgan
\newline \abox{Keywords:} visual display
\newline{\tenpoint\newline
MSSS (SUN only) is MIRIAD's version of the AIPS Sun Screen
Server.  It is available only when running SunView, and
must be invoked separately before being used.  To invoke
MSSS, enter ``msss \&'' in any SunView window that is not
being used to remotely log onto another machine (ie, a
window not using either ``telnet'' or ``rlogin'').

To use MSSS with MIRIAD's TV routines, (1) invoke MSSS in
the manner described above, and (2) specify the server
correctly in the TV routine.  For example, if you are using
SunView on machine ``saturn'', you might do the following:
(1) enter ``msss \&'' to invoke MSSS; (2) remotely log onto
another machine (eg, the Cray) and run MIRIAD task TVDISP,
specifying ``server=msss@saturn''.
{\eightpoint\begintt
                                                          
\endtt}
\par}
\module{odnh3}%
\noindent Ammonia analysis program
\newline \ 
\newline \abox{Responsible:} Peter Teuben
\newline \abox{Keywords:} map combination, map manipulation
\newline{\tenpoint\newline
ODNH3 is a MIRIAD task which creates optical depth and rotational
temperatures maps from ammonia maps.  Ammonia J,K=(1,1) emission
has five hyperfine components, one main hyperfine component, two
inner hyperfine components, and two outer hyperfine components.

Odnh3 takes a main hyperfine component map and either an inner
or outer hyperfine component map and combines the two maps to
make a single main hyperfine component optical depth map. 

Odnh3 can also combine a J,K=(1,1) main hyperfine component
map, a J,K=(2,2) main hyperfine component map, and a J,K=(1,1)
main hyperfine component optical depth map to create a rotational
temperature map.

Any of the input maps can be masked to allow a signal-to-noise cutoff.
\keyword{\bf MAP1}
Main hyperfine component map of the J,K=(1,1) level.  No default.
\keyword{\bf MAP2}
For op=taui, map2 is the inner hyperfine component map of 
the J,K=(1,1) level. 
For op=tauo, map2 is the outer hyperfine component map of 
the J,K=(1,1) level.
For op=temp, map2 is the main  hyperfine component map of 
the J,K=(2,2) level.
No default.
\keyword{\bf MAPT}
For op=temp, mapt is the J,K=(1,1) main hyperfine component 
optical depth map. 
No default.
\keyword{\bf MASK}
A mask expression using FORTRAN syntax.  The optical depth or 
rotational temperature is only calculated at pixels where the 
mask is TRUE.  Thus, this keyword can be used to enforce a 
signal-to-noise cutoff on the input maps.  No default.
\keyword{\bf REGION}
The region of interest in the input images.  No default.
\keyword{\bf OP}
Which Odnh3 option is being used.  See map2 for details.  
No default.
\keyword{\bf OUT}
The name of the output image.  No default.
\par}
\module{panel}%
\noindent Control panel program
\newline \ 
\newline \abox{Responsible:} Jim Morgan
\newline \abox{Keywords:} tools
\newline{\tenpoint\newline
Panel is a TCP/IP server based on SunView, which listens for        
connections from the Miriad "ctrl" routines, constructs a           
control panel according to commands from the Ctrl routines,         
and allows the user and programmer to communicate through the       
control panel.                                                      
                                                                      
\newline See the Miriad Programmers manual, for documentation of the         
Ctrl routines.                                                      
\par}
\module{passfit}%
\noindent Fit polynomials to a passband calibration set
\newline \ 
\newline \abox{Responsible:} Lee Mundy
\newline \abox{Keywords:} calibration
\newline{\tenpoint\newline
PASSFIT is a MIRIAD task that fits polynomials to the passband 
calibration data.
The data and fits are displayed as amplitudes and phases with
all spectral windows for a baseline displayed at one time.
The amplitudes appear as red dots (open dots for flagged data);
the phases appear as blue pluses (open triangles for falgged data).
The amplitudes are autoscaled; the phases range from -180 to 180 
degrees from bottom-to-top of screen. The fits are shown as solid
colored lines.

Commands in cursor mode are:

{\eightpoint\begintt
   ?   This help (also redraws screen)
   q   quit (immediate quit without save)
   e   exit (go to next baseline or exit with save)
   d   delete point nearest cursor
   a   add back point nearest cursor
   1   numbers 1,2,3... zooms to that window
   x   Toggle zoom around in x
   y   Toggle zoom around in y
   z   Toggle zoom around in x and y
   u   unzooms in both x and y
\endtt}
\keyword{\bf PCAL}
The passband calibrator visibility data set.  No default.
\keyword{\bf DROP}
Number of channels to drop at the start and end of each window
before fitting. Fits will be extrapolated to cover these channels.
Default is 1/8th of window dropped from each side. For mode 4,
and additional 1/8th is dropped from the center.
\keyword{\bf ORDER}
The polynomial order for the amp and phase fit to each passband window.  
The default is order=1 for amplitude and 1 for phase. 
\keyword{\bf CHAVE}
Number of channels to average together before fitting polynomials
to amp and phase. Data for individual channels are still displayed
in plot; consequently, there may be an offset between the fits and
the points in the amplitude plots (amplitude bias). Default = 1.
\keyword{\bf DEVICE}
Standard device keyword. See section ``TV Devices''
\par}
\module{passmake}%
\noindent Create passband calibration set
\newline \ 
\newline \abox{Responsible:} Lee Mundy
\newline \abox{Keywords:} calibration
\newline{\tenpoint\newline
PASSMAKE is a MIRIAD task that makes a passband calibration data set
\keyword{\bf VIS}
The passband calibrator visibility data set.  No default.
\keyword{\bf GCAL}
The phase and amplitude gains solution to apply to data.
Default is to apply NO gains solution.
\keyword{\bf AVER}
Type of averaging to apply to passband data. Choices are vector,
scalar or self calibration.  Default is vector.
\keyword{\bf MODE}
Method for normalizing data. Choices are ``cross'' and ``self''. 
Cross mode use the broadband data obtained at the begining and 
ending of the passband observation for normalization. Self uses 
the average of the broadband data obtained by averaging all channels
in the passband data. 
Default is ``cross'' if a gcal file is specified. If no gcal 
file is specified, the default is ``self''.
\keyword{\bf PCAL}
Name for output passband calibration data file.
\par}
\module{prthis}%
\noindent List history item of a dataset
\newline \ 
\newline \abox{Responsible:} Bart Wakker
\newline \abox{Keywords:} utility
\newline{\tenpoint\newline
PRTHIS is a MIRIAD task to list the history item of a miriad
dataset.
\keyword{\bf IN}
The name of the input image. No default.
\keyword{\bf BCOUNT}
The first line to list. Default is 1.
\keyword{\bf LOG}
The name of the output listing file. The default is the terminal
\keyword{\bf TASK}
List history only of this list of tasks (max=5). Default is all tasks.
\par}
\module{puthd}%
\noindent Change the value of or add a header item
\newline \ 
\newline \abox{Responsible:} Peter Teuben
\newline \abox{Keywords:} utility
\newline{\tenpoint\newline
PUTHD is a MIRIAD task to add or modify an item in the ``header''
of an image or uv data set.
\keyword{\bf IN}
The name of an item within a data set. This is given in the
form:
{\eightpoint\begintt
  % puthd in=dataset/item
\endtt}
\keyword{\bf VALUE}
The value to be placed in the item.
\keyword{\bf TYPE}
The data type of the argument. Values can be 'logical', 'integer',
'real', 'double' and 'ascii'. The default is determined from the
form of the value parameter. Normally you can allow this parameter
to default.
\par}
\module{ratty}%
\noindent FORTRAN preprocessor, so that same code works on more machines
\newline \ 
\newline \abox{Responsible:} Bart Wakker
\newline \abox{Keywords:} tools
\newline{\tenpoint\newline
Ratty is a FORTRAN preprocessor for MIRIAD source code, intended to
make the same FORTRAN code compatible on VMS, SUN, Cray, Alliant,
and Convex. The output is suitable for use with the specified compilers
for these machines.

Usage:

{\eightpoint\begintt
    ratty [-s system] [-I incdir] [-D symbol] [-b] [infile] [outfile]

    system:  One of "f77" (generic unix compiler), "cft" (Cray FORTRAN
             compiler), "vms" (VMS FORTRAN), "fx" (alliant unix
             compiler), "convex" (convex unix compiler).

    incdir:  Directory to search for include files.  The -I option may
             be used repeatedly, but must list only one directory for
             each -I parameter (eg, list two directories as
             "-I dir1 -I dir2").

    symbol:  Define this symbol to the preprocessor. Multiple
             definitions are allowed, defining one symbol per -D entry
             (eg, define two symbols as "-D sym1 -D sym2").

    -b:      If specified, backslashes inside quoted textstrings are
             doubled. This allows for compilers which treat the
             backslash as an escape character.

    infile:  Input file name. If omitted, standard input is assumed
             and output must be the standard output.

    outfile: Output file name. If omitted, standard output is assumed.

\endtt}
Ratty recognizes the standard C preprocessor directives \#ifdef,
\#ifndef, \#else and \#endif.

The VAX do/dowhile/enddo extension for do loops is converted to
ANSI FORTRAN if the host machine does not support this extension.

'IMPLICIT NONE' and 'IMPLICIT UNDEFINED(A-Z)' are converted to
whatever the specified compiler supports.

Certain directives are recognized if the target machine has
vector processing capacities (compilers ``cft'', ``fx'' and ``convex''):

{\eightpoint\begintt
#maxloop, followed by a number, is converted to "cdir$ shortloop"
on "cft" and to "cvd$  shortloop" on "fx".
 
#ivdep is converted to "cdir$ ivdep" on "cft", to "cvd$  nodepchk"
on "fx", and to "c$dir no_recurrence" on"convex".
 
#nooptimize is converted to "cdir$ nextscalar" on "cft" and to
"cvd$ noconcur" followed by "cvd$ novector" on "fx".
\endtt}

\par}
\module{reorder}%
\noindent Interchange the axes of a cube
\newline \ 
\newline \abox{Responsible:} Mel Wright
\newline \abox{Keywords:} map manipulation
\newline{\tenpoint\newline
REORDER is a MIRIAD task to reorder and reverse axes within an image.
\keyword{\bf IN}
The input image. No default.
\keyword{\bf OUT}
The output image. No default.
\keyword{\bf MODE}
The new axis ordering in terms of the old axis ordering: i.e.
'312' make input axis 3 the first output axis, input axis 1
the second output axis, etc. Use a - to reverse the pixel
order on an axis.
\par}
\module{restore}%
\noindent Restore clean components to make the CLEAN map
\newline \ 
\newline \abox{Responsible:} Mel Wright
\newline \abox{Keywords:} deconvolution
\newline{\tenpoint\newline
RESTORE is a MIRIAD task which performs a number of functions
typically performed after the deconvolution step. The are
generating a ``CLEAN'' map, calculating residuals and reducing the
resolution of an image estimate to that of the dirty map.
\keyword{\bf MODEL}
The model of the deconvolved image. Usually this will be produced
by CLEAN or MAXEN. The units of this image should be Jy/pixel. No
default.
\keyword{\bf MAP}
The input dirty cube, which should have units of Jy/beam. The
default is a zero cube.
\keyword{\bf BEAM}
The input dirty beam. In some instances, this can be omitted.
\keyword{\bf MODE}
This can be one of the values ``convolve'', ``residual'', ``clean''. For
mode=``convolve'', the output is the model convolved with a gaussian
(the map parameter is ignored, and the beam is needed only if the
user does not give values for fwhm and pa). For mode=``residual'',
the output is the difference between the map and the model
convolved with the beam (the fwhm and pa parameters are ignored).
For mode=``clean'', the output is the map, less the model convolved
with the beam, plus the model convolved with a gaussian (i.e. the
normal restoration step in CLEAN). Mode=``add'' adds a sub-image into
the map. The default mode is ``clean''.
\keyword{\bf FWHM}
The size, in arcsec, of the gaussian beam to use in the
restoration. This will normally be two numbers, giving the
full-width at half-maxima of the major and minor axes of the
gaussian. If only one number is given, the gaussian will have
equal major and minor axes. If no values are given, they are
computed by fitting a gaussian to the given dirty beam.
\keyword{\bf PA}
The position angle, in degrees, of the gaussian restoring beam,
measured east from north. The default is determined from the dirty
beam fit (The value for PA is ignored, if not value is given for
FWHM).
\keyword{\bf OUT}
The output restored image. No default.
\par}
\module{selfcal}%
\noindent Determine self-calibration of calibration gains.
\newline \ 
\newline \abox{Responsible:} Mel Wright
\newline \abox{Keywords:} calibration, map making
\newline{\tenpoint\newline
SELFCAL is a MIRIAD task to perform self-calibration of visibility data.
Either phase only or amplitude and phase calibration can be performed.
The input to SELCAL are a visibility data file, and model images.
This program then calculates the visibilities corresponding to the
model, accumulates the statistics needed to determine the antennae
solutions, and then calculates the self-cal solutions.

The output is a calibration file, ready to be applied to the
visibility data.
\keyword{\bf VIS}
Name of input visibility data file. No default.
\keyword{\bf SELECT}
Standard uv data selection criteria. Generally this should not include
a ``dra'' and ``ddec'' selection, as SELFCAL automatically matches data
with the appropriate pointing center.
\keyword{\bf MODEL}
Name of the input models. Several models can be given, which can
cover different channel ranges of the input visibility data. Generally
the model should be derived (by mapping and deconvolution) from the
input visibility file, so that the channels in the model correspond
to channels in the visibility file. Though the maps can be made using
any linetype, generally ``channel'' linetype will give best results (??).
The units of the model MUST be Jy/pixel, rather than Jy/beam. If
no models are given, a point source model is assumed.
\keyword{\bf CLIP}
Clip level. Anything, in the input model, below the clip level is set
to zero. Default is 0.
\keyword{\bf OUT}
Name of output calibration file. The default is to write the gain
corrections into the input visibility file.
\keyword{\bf INTERVAL}
The length of time, in minutes, of a gain solution. Default is 5,
but use a larger value in cases of poor signal to noise, or
if the atmosphere and instrument is fairly stable.
\keyword{\bf OPTIONS}
This gives several processing options. Possible values are:
{\eightpoint\begintt
   amplitude     Perform amplitude and phase self-cal.
   phase                Perform phase only self-cal.
   smooth        Determine the solutions in such a way that they are
                 smooth with time.
   apriori       This is used if there is no input model, and the
                 source in the visibility data is either a planet,
                 or a standard calibrator. This causes the model data to
                 be scaled by the known flux of the source. For a planet,
                 this flux will be a function of baseline. If the
                 source is a point source, the ``apriori'' option is only
                 useful if the ``amplitude'' and ``noscale'' option are
                 being used. For a planet, this option should also be
                 used for a phase selfcal, to get the correct weighting
                 of the different baselines in the solution.
   noscale       Do not scale the model. Normally the model is scaled
                 so that the flux in the model visibilities and the
                 observed visibilities are the same. Generally this
                 option should be used with at least the apriori option.
                 It must be used if selfcal is being used to determine
                 Jy/K, and should also be used if the model is believed
                 to have the correct scale.
\endtt}
Note that ``amplitude'' and ``phase'' are mutually exclusive.
The default is options=phase.
\keyword{\bf MINANTS}
Data at a given solution interval is deleted  if there are fewer than
MinAnts antennae operative during the solution interval. The default
is 3 for options=phase and 4 for options=amplitude.
\keyword{\bf REFANT}
This sets the reference antenna, which is given a phase angle of zero.
The default, for a given solution interval, is the antennae with the
greatest weight.
\keyword{\bf OFFSET}
This gives the offset in arcseconds of a point source model (the
offset is positive to the north and to the east). This parameter is
used if the MODEL parameter is blank. The default is 0,0. The
amplitude of the point source is chosen so that flux in the model
is the same as the visibility flux.
\keyword{\bf LINE}
The visibility linetype to use, in the standard form, viz:
{\eightpoint\begintt
  type,nchan,start,width,step
\endtt}
Generally if there is an input model, this parameter defaults to the
linetype parameters used to construct the map. If you wish to override
this, or if the info is not in the header, or if you are using
a point source model, this parameter can be useful.
\par}
\module{shifty}%
\noindent Align two images.
\newline \ 
\newline \abox{Responsible:} Robert Sault
\newline \abox{Keywords:} image analysis
\newline{\tenpoint\newline
SHIFTY is a MIRIAD task which shifts one image, to that it best aligns
with another image. Only integral pixel shifts are performed.
\keyword{\bf IN1}
The first input image. This is considered the master image. No default
\keyword{\bf IN2}
The second input image. This is the image that is shifted to make it
align better with the master image. No default.
\keyword{\bf OUT}
The output image. No default.
\par}
\module{smooth}%
\noindent Convolve an image (in the image domain) with a gaussian.
\newline \ 
\newline \abox{Responsible:} Neil Killeen
\newline \abox{Keywords:} analysis
\newline{\tenpoint\newline
SMOOTH is a MIRIAD task which convolves an image by an elliptical
gaussian the hard way. 
\keyword{\bf IN}
The input image.
\keyword{\bf OUT}
The output image.
\keyword{\bf FWHM}
The Gaussian FWHM in arcseconds.
\keyword{\bf PA}
The Gaussian position angle in degrees CCW from North.
\keyword{\bf NORM}
If 0.0 the convolution integral is normalized by the volume of
the gaussian.  Else, it is normalized by norm.  Default is 1.0
\keyword{\bf BLANK}
If 'yes' then blanked input pixels do not contribute to the
convolution sum.
If 'no' then blanking is not checked for (but the output image
is blanked around the unconvoled edge, and wherever the
input image is blanked).
Default is 'yes'
\par}
\module{tpgains}%
\noindent Estimate antenna gains from total power measurements.
\newline \ 
\newline \abox{Responsible:} Mel Wright
\newline \abox{Keywords:} tpower, antenna gains
\keyword{\bf VIS}
Name of input visibility data file. No default.
\keyword{\bf TGAIN}
The total power is modeled to be correlated with the atmospheric
phase noise. tgain gives the total power in units of Kelvin/radian.
Default value for millimeter wavelengths is tgain=0.5 K/radian.
\keyword{\bf REFANT}
The gain of this antenna is set to cmplx(1.,0.).  Default is antenna 1.
\keyword{\bf OUT}
Name of output calibration file. The default is to write the gain
corrections into the input visibility file.
\module{tvall}%
\noindent Interact with a previously displayed image
\newline \ 
\newline \abox{Responsible:} Jim Morgan
\newline \abox{Keywords:} visual display
\newline{\tenpoint\newline
TVALL is a WERONG/MIRIAD task which allows interactive
modification of the TV lookup tables, etc. Usually this would be
used to manipulate an image previously loaded with TVDISP.
A menu for operations
to be performed will appear on the TV screen. The following are
possible: 

Choose the channel being displayed (Select the channel number
from the menu). 

Blink between two channels (The mouse y-coordinate determines
the blink rate. It blinks between the current and succeeding
image, mod 3. Hit any button to stop blinking). 

Fiddle the lookup tables (The transfer function is governed by
the mouse position. Hit any button to finish the command). 

Change the lookup tables between black and white, or colour. 

Zoom and pan (the left button zooms in, the middle button zooms
out, and the mouse position controls pan. The cursor drawn on
the IVAS gives the position, on the unzoomed screen, of the
central pixel currently being displayed. To finish, hit the
right button.). 

Switch the menu off (by `selecting' the asterisk). This is
useful if you want to take a photograph of the screen. 

Return cursor position. The coordinates given are coordinates
(not image coordinates). 
\keyword{\bf SERVER}
The TV device. No default. See the Users Manual for information
on how to specify this.
\par}
\module{tvcln}%
\noindent Clark/Steer CLEAN with display of intermediates
\newline \ 
\newline \abox{Responsible:} Mel Wright
\newline \abox{Keywords:} deconvolution
\newline{\tenpoint\newline
TVCLN is a MIRIAD task, which performs a hybrid Clark/Steer Clean
algorithm, which takes a dirty map and beam, and produces an output
map which consists of the Clean components. This output can be
input to ASCAL to self-calibrate visibilities, or input to RESTORE
to produce a ``clean'' image. Optionally CLEAN can take as one of
its inputs a model of the deconvolved image. This model could be
from a previous CLEAN run, or from any of the other deconvolution
tasks (e.g. MAXEN). 
\keyword{\bf MAP}
The input dirty map, which should have units of Jy/beam. No
default. 
\keyword{\bf BEAM}
The input dirty beam. No default
\keyword{\bf MODEL}
An initial model of the deconvolved image. This could be the
output from a previous run of CLEAN, or the output of any of the
deconvolution tasks (e.g. MAXEN). It must have flux units of
Jy/pixel. The default is no model (i.e. a zero map). 
\keyword{\bf OUT}
The name of the output map. The units of the output will be
Jy/pixel. It can be input to RESTORE, CLEAN (as a model, to do
more cleaning), or tor ASCAL(??) (for self-calibrating visibility
data). 
\keyword{\bf GAIN}
The minor iteration loop gain. Default is 0.1.
\keyword{\bf CUTOFF}
CLEAN finishes when the absolute maximum residual falls below
CUTOFF. Default is 0. 
\keyword{\bf NITERS}
The maximum number of minor iterations. Clean is finished when the
absolute value of NITERS minor iterations have been performed. Clean
may finish before this point, however, if NITERS is negative and the
absolute maximum residual becomes negative valued, or if the cutoff
level (as described above) is reached. 
\keyword{\bf REGION}
This specifies the region to be Cleaned.
\keyword{\bf PHAT}
The Cornwells prussian hat parameter. When cleaning extended sources,
CLEAN may produce a badly corrugated image. This can be suppressed
to some extent by cleaning with a dirty beam which has had a spike
added at its center (i.e. a beam that looks like a prussian hat).
PHAT gives the value of this spike, with 0 to 0.5 being good
values. Default is 0 (but use a non-zero value for extended
sources). 
\keyword{\bf MINPATCH}
The minimum patch size when performing minor iterations. Default
is 51, but make this larger if you are having problems with
corrugations. You can make it smaller when cleaning images which
consist of a pretty good dirty beam. 
\keyword{\bf SPEED}
This is the same as the speed-up factor in the AIPS APCLN.
Negative values makes the rule used to end a major iteration more
conservative. This causes less components to be found during a
major iteration, and so should improve the quality of the Clean
algorithm Usually this will not be needed unless you are having
problems with corrugations. A positive value can be useful when
cleaning simple point-like sources. Default is 0. 
\keyword{\bf MODE}
This can be either ``hogbom'', ``clark'', ``steer'' or ``any'', and
determines the Clean algorithm used. If the mode is ``any'', then
CLEAN determines which is the best algorithm to use. The default
is ``any''. 
\keyword{\bf CLIP}
This sets the relative clip level in a Steer clean, values
typically being 0.75 to 0.9. The default is image dependent. 
\keyword{\bf SERVER}
If this is set to the name of a TV device, then at the end of
each major cycle, the restored image is displayed on the TV.
\par}
\module{tvdisp}%
\noindent Display an image on a TV device
\newline \ 
\newline \abox{Responsible:} Jim Morgan
\newline \abox{Keywords:} visual display
\newline{\tenpoint\newline
TVDISP is a MIRIAD task to display an image on a TV device. TVDISP
does not clear the screen before it initializes the TV. See TVINIT
for this function.
\keyword{\bf IN}
Input file name. No default.
\keyword{\bf REGION}
The region of the image to be displayed. Default is the entire
image. See the Users Manual for information on how to specify
this.
\keyword{\bf RANGE}
The minimum and maximum range used in scaling. Default is the
image minimum and maximum. 
\keyword{\bf TVCHAN}
The TV channel to display the image on. This can be 1 to 3 on the
IVAS, or 1 or 2 on a Sun. The default is 1. 
\keyword{\bf INCR}
Only every INCR'th pixel is displayed. This can take 1 to 3
values, giving the increment in x, y and z respectively. The
default such than the images will fit on the screen. 
\keyword{\bf TVCORN}
This gives the coordinate where the lower left corner of the image
is displayed. The coordinate system runs from 1 to 1024 in x and
y, the origin being in the lower left corner. The default centers
the image on the screen. 
\keyword{\bf SERVER}
The TV device where the image is to be displayed. No default. 
See the Users Manual for information on how to specify this.
\keyword{\bf OPTIONS}
These indicate extra operations to be performed. One or more can
be given. Possible values are: 

{\eightpoint\begintt
  fiddle  Allow interactive modification of colour tables, etc.
  movie   Run the sequence as a movie.
\endtt}
\par}
\module{tvflag}%
\noindent Interactive editing of a UV data set on a TV device.
\newline \ 
\newline \abox{Responsible:} Jim Morgan
\newline \abox{Keywords:} calibration, uv-data, tv, plotting, display
\newline{\tenpoint\newline
TVFLAG is a MIRIAD task which allows interactive baseline
editing of a UV data set.  The user should make sure,
if the display device is capable of creating menus that
the menu program should be loaded ``before'' running this
routine (eg. panel for the SSS) - see the User's Guide or
Cookbook under TV Devices for details.

NOTE:  This application will not work if it is run from
MIRTOOL on the SUN without PANEL running.

When this program is run, each baseline is displayed,
one at a time, with the x-axis representing channel
number (increasing to the right), the y-axis representing
the change in time (increasing upward), and the color
intensity representing the amplitude of the visibility.
The current antenna that constitute the baseline are
labeled at the bottom of the plot (if room permits).
In addition, if room permits, a color wedge is displayed
to the right side of the visibility data and is the sum
of the data over all displayed channels.  Also, a wedge
of the sum of the data over all displayed times will
appear above (or below) the data as room permits.

The user may edit the data in one of two ways: 1) By
entering commands at the keyboard; or 2) by selecting
commands from the listed menu items.  Which method is
determined by the presence of a menu program (such as
``panel'' for the Sun systems).

If no menu program is currently active, then a warning
is issued to the user and all commands are prompted from
the keyboard.  Commands are entered followed by a carriage
return ({\tt <}CR{\tt >}).  Single letter abbreviations are used for
all commands.  The current command list, each corresponding
abbreviation, and a brief description of each command is
available by entering a question mark (``?'') as a command.
Commands requiring further interaction (eg. ``select'' and
``value'') will prompt the user when they are invoked.

If a menu program is active, then TVFLAG will attempt
to construct an assortment of menu buttons and other
items that will perform the equivalent of entering commands
at the keyboard.  Certain commands (eg. ``select'' and
``value'') require further cursor input and will prompt
the user for a particular action (for example: ``select''
will request the user to select the region to box for
further flagging commands).

The current technique of editing is to load the baseline
to be edited, zoom and pan to the desired location, and
then, select the region to be edited followed by an
editing command.  To identify a region to edit, chose
the ``select'' command and then use the cursor to box
the desired points.  This is done by moving to a corner
of the region to select, identifying the program of the
corner (usually by pushing and holding the left mouse
button), moving the pointer to the other corner (by
dragging the mouse), and then identifying the program of
the other corner (releasing the left mouse button).  After
identifying the region to edit, the selected region may be
flagged as good or bad using the appropriate command (or
button).  If data is flagged improperly, the ``undo''
command will reverse the last editing command.

To save the editing done to a particular baseline, enter
the ``exit'' command (or select the ``exit'' button).
The flagging changes will be immediately applied to the
data.  To NOT save the changes made to a particular
baseline, use the ``quit'' command (or button).  After
a ``quit'' or ``exit'' operation, the data for the next
baseline is loaded onto the display the user may continue
editing.  Selecting the ``abort'' command (button) will
perform the same operation as ``quit'', but will also
terminate the program.
\keyword{\bf VIS}
The name of the input UV data set.  A visibility file name
must be supplied.  Only one file may be edited at a time.
\keyword{\bf SOURCE}
The name of the source to apply corrections to if more than
one source is present in a UV data set.  All UV data that
does not correspond to the input source name is ignored
(ie. not edited).  If this keyword is not set (the default),
then every display will contain all sources at that
baseline.  Only one source name may be input.
\keyword{\bf SERVER}
The TV device where the data is to be displayed.  There
is no default for this keyword.  See the section on TV
Devices in the Miriad Users Guide for further information
on how to specify this keyword.
\keyword{\bf TVCHAN}
An integer specifying which TV channel the data is
displayed.  The default is channel 1.
\keyword{\bf RANGE}
The minimum and maximum range used in scaling the data
on the TV.  Default is to autoscale to the first image.
If this keyword is used, TWO parameters must be input.
\keyword{\bf TVCORN}
The integer device coordinate of the lower left corner
of the image.  The default is to center the image on the
display.  If this keyword is used and only the x coordinate
value is input, the y coordinate value is set to the x value.
\keyword{\bf LINE}
Line type of the data in the format
{\eightpoint\begintt
    type,nchan,start,width,step
\endtt}
Here ``type'' must be `channel' and the maximum of
both ``width'' and ``step'' must be 1.  The default is
to display all channels.
\keyword{\bf SELECT}
Selects which visibilities are to be used.  The default is
to use all visibilities.  See the section on UV Data
Selection in the Miriad Users Guide for information on how
to specify this keyword.
\par}
\module{tvinit}%
\noindent Initialize a TV device
\newline \ 
\newline \abox{Responsible:} Robert Sault
\newline \abox{Keywords:} visual display
\newline{\tenpoint\newline
TVINIT is a MIRIAD/Werong task which initializes a TV device. These
initializations clear the screen, set pan, zoom and colour lookup
table to there defaults.
\keyword{\bf SERVER}
The TV device. No default. See the Users Manual for information
on how to specify this.
\par}
\module{uvaflag}%
\noindent Use flags in one uv database to set flags in another
\newline \ 
\newline \abox{Responsible:} Bart Wakker
\newline \abox{Keywords:} calibration
\newline{\tenpoint\newline
UVAFLAG is a MIRIAD task which flags correlations in one 
database when they are flagged in a template data base.
Data bases must be identical except for the values of the
correlations and the initial flag mask. 
\keyword{\bf TVIS}
The template input visibility file. No default.
\keyword{\bf VIS}
The name of the input visibility file. No default.
\par}
\module{uvcat}%
\noindent Concatenate and copy uv datasets into one; apply gains file, select windows.
\newline \ 
\newline \abox{Responsible:} Robert Sault
\newline \abox{Keywords:} uv analysis
\newline{\tenpoint\newline
UVCAT is a MIRIAD task which copies and catenates multiple MIRIAD
uv data sets. By default, UVCAT applies the gains file in copying the
data. The spectral windows copied to the output file can be selected.
\keyword{\bf VIS}
The names of the input uv data sets. Multiple names can be given,
separated by commas. At least one name must be given.
\keyword{\bf SELECT}
The normal uv selection commands. One unusual aspect of this is that
the ``window'' subcommand can be used to select which windows are
copied to the output file (normally the ``window'' only has an
effect for velocity line type). The default is to copy everything.
\keyword{\bf OPTIONS}
This gives extra processing options. Several options can be given,
each separated by commas. They may be abbreivated to the minimum
needed to avoid ambiguity. Possible options are:
{\eightpoint\begintt
   'nocal'       Do not apply the gains file. By default, UVCAT
                 applies the gains file in copying the data.
   'nowide'      Do not copy across wide-band channels.
   'nochannel'   Do not copy across spectral channels.
   'unflagged'   Copy only those records where there are some
                 unflagged visibilites.
\endtt}
\keyword{\bf OUT}
The name of the output uv data set. No default.
\par}
\module{uvcheck}%
\noindent Check uvdata selection and uv-variable values.
\newline \ 
\newline \abox{Responsible:} Mel Wright
\newline \abox{Keywords:} uv analysis, checking
\newline{\tenpoint\newline
UVCHECK is a Miriad program to check the uvdata.
The default is to read through the selected uvdata and report the
times when the source, freq, or number of channels change. The range
of values in other variables can also be checked. The number of
records read and the uv and date range are also reported.
\keyword{\bf VIS}
The input visibility file. No default.
\keyword{\bf SELECT}
This selects which visibilities to be used. Default is
all visibilities. See the Users Guide for information about
how to specify uv data selection.
\keyword{\bf LINE}
Linetype of the data in the format line,nchan,start,width,step
``line'' can be `channel', `wide' or `velocity'.
Default is channel,0,1,1,1, which returns actual number of spectral
channels. Use line=wide to check number of wideband correlations.
If the linetype requested is not present it will NOT be listed.
\keyword{\bf VAR}
Name of uv-variable to check the range of values. Default is none.
\keyword{\bf RANGE}
The range of values for var. Times and values when var is outside
this range are listed in the output. Default range=0,0 (min,max)
\keyword{\bf LOG}
The output log file. Default is the terminal.
\par}
\module{uvcover}%
\noindent Display uv coverage
\newline \ 
\newline \abox{Responsible:} Lee Mundy
\newline \abox{Keywords:} uv analysis
\newline{\tenpoint\newline
UVCOVER displays UV coverage.
\keyword{\bf VIS}
The name of the input visibility file. No default.
\keyword{\bf DEVICE}
Standard device keyword. See section ``TV Devices''
\par}
\module{uvdisplay}%
\noindent Suntools display header information about uv dataset
\newline \ 
\newline \abox{Responsible:} Lee Mundy
\newline \abox{Keywords:} uv analysis
\newline{\tenpoint\newline
UVDISPLAY displays values of all uv header variables.
``coord'', ``wcorr'', ``bl'', and ``corr'', are not displayed,
since they are baseline based variables, not header variables. 
\keyword{\bf VIS}
The uv dataset.                                                
\par}
\module{uvedit}%
\noindent Editing of the baseline of a UV data set.
\newline \ 
\newline \abox{Responsible:} Jim Morgan
\newline \abox{Keywords:} calibration, uv-data
\newline{\tenpoint\newline
UVEDIT is a MIRIAD task which allows baseline editing of a UV
data set.  As a result of the editing, certain header variables
of the data set are changed.  The headers ``corr'' and ``wcorr''
are always changed since they are the data themselves.  The
headers ``coord(2)'' are the baseline coordinates and, as a
result, are also always changed.  The headers ``lst'', ``ut'',
and ``time'' are updated whenever a time offset is entered.
The headers ``ra'', ``dec'', ``obsra'', and ``obsdec'' are
changed whenever a positional correction is entered.  Finally,
antenna coordinate corrections will cause the header ``antpos''
to be corrected.

NOTE: There can be NO select keyword for this routine!  If
one includes the select option, then data that is not selected
will not be copied!
\keyword{\bf VIS}
The name of the input UV data set.  At least one file name must
be supplied.  Up to 10 visibility files are currently allowed.
\keyword{\bf SOURCE}
The name of the source to apply corrections to if more than
one source is present in a UV data set.  All UV data that does
not correspond to the input source name is copied without being
edited.  If this keyword is not set (the default), then all
sources are edited.  Only one source name may be input.
\keyword{\bf ANTPOS}
Inputs are the absolute equatorial coordinates entered in the
following order (NO checking is done for consistency):
{\eightpoint\begintt
     value = A1,X1,Y1,Z1,A2,X2,Y2,Z2,A3,X3,Y3,Z3,....
\endtt}
The input values are the antenna number and the three equatorial
coordinates (entered in units of nanoseconds).  Note that A1 does
not necesarily have to correspond to Antenna 1; it is used to 
represent the variable containing the antenna number.  Antenna
(and the corresponding coordinates) not included in the input
listing do not have their coordinates changed.  You may specify
either ``antpos'' or ``dantpos'', but not both.
\keyword{\bf DANTPOS}
Inputs are the equatorial coordinate offsets entered in the
following order (NO checking is done for consistency):
{\eightpoint\begintt
     value = A1,X1,Y1,Z1,A2,X2,Y2,Z2,A3,X3,Y3,Z3,....
\endtt}
The input values are the antenna number and the three equatorial
coordinate offsets (entered in units of nanoseconds).  These input
values are added to the absolute coordinates read from the data.
Note that A1 does not necesarily have to correspond to Antenna 1;
it is used to represent the variable containing the antenna
number.  Antenna present in the data but not included in the
input value list are treated as having a zero coordinate offset.
You may specify either ``antpos'' or ``dantpos'', but not both.
\keyword{\bf RA}
Input is either an absolute or delta right ascension of the
phase tracking center.  If one value is present, it is considered
as a offset position and is to be entered as time seconds.
Otherwise, three values are expected and are to be entered in
the following order:
{\eightpoint\begintt
     value = HH,MM,SS.S
\endtt}
The right ascension (offset) is relative to the epoch coordinates.
The default value is 0 seconds offset (no change).
\keyword{\bf DEC}
Input is either an absolute or delta declination of the phase
tracking center.  If only one value is present, it is considered
as a offset position and is to be entered in arcseconds.
Otherwise, three values are expected and are to be entered in
the following order:
{\eightpoint\begintt
     value = DD,MM,SS.S
\endtt}
The declination (offset) is relative to the epoch coordinates.
The default value is 0 arcseconds offset (no change).
\keyword{\bf TIME}
Input is a time offset (in seconds) to be added to the clock time.
The default value is 0 seconds offset (no change).
\keyword{\bf OUT}
The name of the output visibility file.  This parameter is
ignored when more than one visibility file is given.  If no value
for ``out'' is given or more than one visibility file is input,
then the output file name(s) will be the same as the input file
name(s) but with an ``\_c'' appended to the file name
(ie. ``Vis = saturn,jupiter'' will result in output files ``saturn\_c''
and ``jupiter\_c'').
\par}
\module{uvflag}%
\noindent Flags or unflags uv data
\newline \ 
\newline \abox{Responsible:} Bart Wakker
\newline \abox{Keywords:} calibration, uv analysis
\newline{\tenpoint\newline
UVFLAG is used to change flags corresponding to visibility data.
Using the keywords select, shadow, line and edge one selects a
portion of a uv-data file. For all selected correlations the flags
are then set to the value specified with the keyword flagval. If
flagval equals 'flag', the uv data are flagged as bad; for flagval
equal to 'unflag', the uv data are flagged as good.
The user can control the amount of output produced using the options
keyword. Only records that fulfill all selection criteria as
specified by select, shadow, line and edge are shown and counted,
not the complete datafile.
Because the physical writing of the flags is done using a buffering
approach, the flags are only actually changed in the datafile if
one lets uvflag finish normally. Especially for options 'indicative'
and 'full' it is necessary to know this. I.e., if the 'q' option is
used to stop printing, the flags have not yet been changed.
UVFLAG can also be used to inspect the uv-data. If option 'noapply'
is used, everything works as it would do normally, except that the
flags are not actually changed in the datafile. This is particularly
useful in combination with option 'full'.
\keyword{\bf VIS}
The name of the input visibility file. No default.
\keyword{\bf SELECT}
Standard select keyword. See section ``uv selection''.
\keyword{\bf SHADOW}
Extra selection on top of 'select', to select shadowed data
only. There are two values; the first gives whether or not
shadowed data should be selected ('y', 't' or 1 to select,
'n', 'f' or 0 to ignore this selection, which is the
default); the second, optional, value gives the minimum
projected baseline (in meter), below which data is
considered shadowed. The default overlap is 3m.
[n]
\keyword{\bf LINE}
Standard linetype keyword, see section ``uv Linetypes''.
Uvflag supports only types 'channel and wide'. Also, since
averaging of data is an undefined operation when setting
flags, width is forced to be equal to 1. If you want to flag
averaged data you first have to make a new dataset.
['channel',0,1,1,1]
\keyword{\bf EDGE}
This keyword allows uvflag to work on only a subset of
the channels selected by 'line'. Flags will be changed
only for the first n and last m channels that were selected
using 'line'.
If 'edge' is zero, this extra selection is ignored.
If one value is given, the number of selected channels at
the start and end of the band is assumed to be equal.
If two values are given the first gives the number of
selected channels at the start, the second the number at
the end. In this case either one may also be 0.
The most common use of 'edge' is with a 'line' keyword that
selects all correlator channels (the default).
[0,0]
\keyword{\bf FLAGVAL}
Either 'flag' or 'unflag', which tells whether the flags for
the correlations selected by 'select', 'shadow', 'line' and
'edge' have to be set or unset. May be abbreviated to 'f' or
'u'. Exactly one of the options 'flag' or 'unflag' must be
present.
\keyword{\bf OPTIONS}
One or more of
1) 'noapply',
2) 'none', 'brief', 'indicative', 'full', 'noquery',
3) 'hms', 'decimal'.
These options can be abbreviated to uniqueness.

\newline 1) 'noapply' will go through the whole process without
actually changing the flags. Useful for checking what will
happen or for inspecting the flags.
No history comments are written.

\newline 2) 'none', 'brief', 'indicative' and 'full' control the
amount of information returned to the user. The default is
'none', i.e. no output; except when a logfile is given (see
keyword log), then it becomes 'indicative'.
{\eightpoint\begintt
- 'brief' gives an overview when UVFLAG finishes.
- 'indicative' lists the number of good, bad and changed
   flags for each selected uv-record.
- 'full' lists the data, the old flag and the new flag for
  each channel separately and also an overview of each record.
- If more than 1 verbosity level is given, the lowest is taken.
- Option 'noquery' will turn off the feature that printing
  is halted every 22 lines.
\endtt}
\newline 3) for verbosity levels 'indicative' and 'full' the format
of the time that is written is determined by 'hms' (hours,
minutes and seconds) or 'decimal' (decimal parts of a day).
['none,hms' or 'indicative,hms']
\keyword{\bf LOG}
Name of a file where reported information is written.
If empty this implies the terminal screen, else the
named file. Giving a filename also sets option 'indicative'.
[terminal]
\par}
\module{uvgen}%
\noindent Compute visibilities for a model source.
\newline \ 
\newline \abox{Responsible:} Mel Wright
\newline \abox{Keywords:} uv analysis, map making
\keyword{\bf SOURCE}
The name of a text file containing the source components. The
default is ``uvgen.source''. If the specified model components file
does not exist, UVGEN interactively prompts the user for information,
and then generates the file. The source components are elliptical
Gaussian components described by a peak flux density (Jy) and
position offsets (arcsecs) from the phase center in the directions
of ra and dec. The sources are specified by the full width to half
maximum of the major and minor axes; the position angle of the
major axis measured from north to the east. The default half width
for a ``point'' source is 0.``0001 .
\keyword{\bf ANT}
The name of a text file containing the position of the antennae.
The default is ''uvgen.ant``. If the specified antenna file does
not exist, UVGEN interactively prompts the user for the coordinates
of the antennas, and then generates the antenna file.
The antenna positions can be given in either a right handed
equatorial system or as a local ground based coordinates measured to the
north, east and in elevation. See the ''baseunit`` parameter to
specify the coordinate system.
\keyword{\bf BASEUNIT}
This specifies the coordinate system used in the antenna file.
A positive value for ''baseunit`` indicates an equatorial system,
whereas a negative value indicates a local system. The magnitude of
''baseunit`` gives the conversion factor between the baseline units
used in the antenna file, and nanoseconds. The default value is +1,
which means that the antenna file gives the antenna position in an
equatorial system measured in nanoseconds. Remember 1 ns is equivalent
to 0.3 meters.
\keyword{\bf SITE}
This can take on the value of ''hatcreek`` or ''other``. This determines
the interpretation of the correlator setup parameters (see below).
The default is ''hatcreek``.
\keyword{\bf CORR}
This gives the name of a text file specifying the correlator
setup, and a spectral line model. The default name is ''uvgen.corr``.
If it does not exist, UVGEN prompts interactively, and then creates
the file.
The values are:
{\eightpoint\begintt
Correlator mode:    an integer between 1 and 4. For Hat Creek, this gives the
                    correlator setup. Otherwise it is ignored.
Number of spectra:  For Hat Ck, this can be from 0 to 8. 0 would be for wide
                    band only. Otherwise it should be 0 or 1.
Number of channels: This is ignored for Hat Ck (it is determined from the
                    number of spectra and correlator mode).
Four correlator LO frequencies and two correlator bandwidths: 
                    These are specified in MHz. For other than Hat Ck, only
                    the first correlator bandwidth is used. For Hat Ck, the
                    LOs would normally be in the range 80 -- 520, and the
                    bandwidths 5 -- 40.
\endtt}
Three parameters, famp, fcen and fwid, giving line to continuum
ratio, freq and width (GHz). This gives a simple spectral line.
In particular, the visibility value for a channel is scaled by
a factor:
{\eightpoint\begintt
    1 + famp*( 1-min(1,abs((f-fcen)/fwid)) )
\endtt}
\keyword{\bf TIME}
The start time of the observation. This is in the form
{\eightpoint\begintt
  yymmmdd.ddd or yymmmdd:hh:mm:ss.s
\endtt}
The default is 80JAN01.0. A function of this is also used
as a seed to start the random number generator off.
\keyword{\bf FREQ}
Frequency for the model in GHz. For Hat Ck, this is assumed to
be in the upper sideband. Default is 100 GHz.
\keyword{\bf DECL}
Declination for the model in degrees. Default is 0 deg.
\keyword{\bf STOKES}
Polarization selection. This can be one of 'ii' (default), 'lr',
'rl', 'rr' or 'll'.
\keyword{\bf LAT}
Latitude of observatory, in degrees. Default is 40 degrees.
\keyword{\bf HARANGE}
Hour Angle range (start,stop,step) in hours. Default is
harange=-6,6,0.1 (6 minute integration)
\keyword{\bf GNOISE}
Antenna based gain noise, given as a percentage. This gives the
multiplicative gain variations, specified by the rms amplitude to be
added to the gain of each antenna at each sample interval. The gain
noise can also be used to mimic random pointing errors on the same
time scale as the sample interval for the data.
The default is 0 (i.e. no gain noise).
\keyword{\bf PNOISE}
Antenna based phase noise, in degrees. This gives the phase
noise, specified by the rms phase noise to be added to each
antenna at each sample interval. The default is 0 (i.e. no
phase noise).
\keyword{\bf SYSTEMP}
System temperature for additive noise, in Kelvin. This is used
to generate random Gaussian noise to add to each data point. The
default is 0 K (i.e. no noise).
\keyword{\bf TPOWER}
Three numbers can be given to represent the total power variations
due to receiver instability, telescope elevation dependence, and
atmospheric noise. The total power is computed as:
{\eightpoint\begintt
  tpower = systemp + trms + telev * cos(el) + tatm * antpnoise
\endtt}
The receiver instablity is modeled as additive Gaussian noise.
The atmospheric noise is modeled to be correlated with the antenna
phase noise. Units of trms, telev in Kelvin and tatm in Kelvin/radian.
The systemp is not changed. Typical values for millimeter wavelengths
are trms=1 K (10-4 * systemp) telev=100 K and tatm=0.5 K/radian.
Default is tpower=0,0,0
\keyword{\bf JYPERK}
The system sensitivity, in Jy/K. Its value is given by
{\eightpoint\begintt
  2500 / (eta * D**2)
\endtt}
where eta is the combined aperture and correlator efficiency
(typical correlator efficiency is 0.9, and aperture is 0.6),
and D is the dish diameter in meters. The default is 200, which
is typical for Hat Ck. For the Australia Telescope, a value of
9 is expected.
\keyword{\bf OUT}
This gives the name of the output Miriad data file. The default
it ''uvgen``.
\module{uvhat}%
\noindent Convert Hat Creek uv format into MIRIAD uv format
\newline \ 
\newline \abox{Responsible:} Wilson Hoffman
\newline \abox{Keywords:} data transfer
\newline{\tenpoint\newline
UVHAT is a MIRIAD task which converts a file from RALINT UV format 
into MIRIAD UV format.
On VMS it automatically detects so called ``oldhat'' and ``newhat''
formatted RALINT files.
On SUN Unix the ``oldhat'' data can only be converted with UVHAT.
On SUN Unix the ``newhat'' data can only be converted with HCCONV.
\keyword{\bf IN}
The input file name. No default.
Visibility data may be accessed either directly by giving
the filename, or indirectly via an index file which is
identified by name.INC. The index file can also select data
by time range and antennas, and the data can be scaled by
parameters in the index file. The index file can be used to
directly convert RALINT uvdata sets into a MIRIAD uv file.
The structure of the index file is as follows: 

filename.ext    (name of the data file to include)
first, last     (selects day.ut range)
first, last     (selects day.ut range; up to 20 time ranges)
ANT=num         (select antennas, e.g. 123 for all three)
SCA=s1,s2,s3    (scales the data for baselines 12, 23, 31)
WEI=w1,w2,w3    (weights - not used in UVHAT)
DPH=p1,p2,p3    (phase correction added to data for baselines 12,23,31)
{\eightpoint\begintt
                [units: radians]
\endtt}
next {\tt <}filename.ext{\tt >}
etc.

The scale factors can be used to convert the visibility data to Jy
for mapping. This is normally done by the calibration. The data for
the preceeding filename is multipled by sca(i)*expi(dph(i)) for
baselines i=1,2,3.

Bad channels can be flagged using the following two parameters:
\newline \ 
CHN= number of ranges followed by list up to 10 pairs of numbers
to specify ranges of GOOD channels.
\newline \
BAD= number of ranges of bad channels followed by list of up to
      10 pairs of numbers to specify ranges of BAD channels.
e.g.
{\eightpoint\begintt
        CHN=2,1,64,129,256
        BAD=3,32,33,135,137,19,191
\endtt}
uses channels 1 tho 64 and 129 thro 256, and flags channels (32,33)
(135,137), and (191,193). Channels .gt.256 are flagged BAD.

The initial default is CHN=1,1,512  i.e. no bad channels.
The channel selection is not reset for each file and need only be
reset if channel selection changes.

Comment lines must start with ``!''.
\keyword{\bf OUT}
The output dataset name (looks like 1-30 alphas, ends up as a
subdirectory of the current default directory.
\keyword{\bf LOGRANGE}
The beginning and end Hat Creek Log number, to be included in
the output. Default is all data.
\par}
\module{uvindex}%
\noindent Scan a uvdata file, and note when uvvariables change.
\newline \ 
\newline \abox{Responsible:} Mel Wright
\newline \abox{Keywords:} uv analysis, checking
\newline{\tenpoint\newline
UVINDEX is a Miriad program which scans a uvdata file.
Changes in source name, pointing center, number of wideband
and spectral line channels, and the observing frequency are
listed in the output.
\keyword{\bf VIS}
The input visibility file. No default.
\keyword{\bf LOG}
The output log file. Default is the terminal.
\par}
\module{uvlist}%
\noindent Print data values from uv dataset
\newline \ 
\newline \abox{Responsible:} Mel Wright
\newline \abox{Keywords:} uv analysis
\newline{\tenpoint\newline
UVLIST lists a MIRIAD UV data file. It can list either the
variables or the correlation data.

Generally the data are presented in their raw units, but some
variables and data can undergo some massaging. In particular, the
time is given as a standard calendar date, except for
options=brief,data (see below), where the time is given as days
(and fractions of a day) since 1980.0. 
\keyword{\bf VIS}
The input UV dataset name. No default.
\keyword{\bf OPTIONS}
This is composed of ``brief'' (default), ``full'' (the opposite of
brief), ``data'' (default -- this lists correlation data),
``variables'', ``spectra'' and ``history''. Several of these can
be given, separated by commas. For example, the default is:
{\eightpoint\begintt
  % uvlist options=brief,data
\endtt}
The most comprehensive listing is:
{\eightpoint\begintt
  % uvlist options=full,data,variables,history,spectra
\endtt}
These suboptions can be abbreviated to uniqueness.
\keyword{\bf SELECT}
This selects the data to be processed, using the standard uvselect
format. Default is all data.
\keyword{\bf LINE}
For options=data, this gives the linetype that is printed, in the
form:
{\eightpoint\begintt
  type,nchan,start,width,step
\endtt}
where type can be `channel' (default), `wide' or `velocity'.
The default is to print all the raw channel data (no averaging,
etc). If options=brief, a maximum of only 6 channels will be printed.
\keyword{\bf RECNUM}
The number of output records. This is used to cut off long outputs.
The default is 1.
\keyword{\bf LOG}
The list output file name. The default is the terminal.
\par}
\module{uvmodel}%
\noindent Add, subtract, etc, a model from a uv data set.
\newline \ 
\newline \abox{Responsible:} Mel Wright
\newline \abox{Keywords:} uv analysis
\newline{\tenpoint\newline
UVMODEL is a MIRIAD task which modifies a visibility dataset by a model.
Allowed operations are adding, subtracting, multiplying, dividing and
replacing. The model is specified in the image domain, so that its
Fourier transform is first computed before application to the
visibilities. The model may be either an image (e.g., a CLEAN
component image) or a point source.

An example is as follows. UVMODEL could be used to remove CLEAN
components from  a visibility data file.  The residual data base could
then be examined for anomalous points, which could in turn be clipped.
UVMODEL could then be reapplied to add the CLEAN components back into
the visibility data base for re-imaging. 
\keyword{\bf VIS}
Input visibility data file. No default
\keyword{\bf MODEL}
Input model cube. If left blank, then a point source model is used.
This will generally be a deconvolved map, formed from the visibility
data being modified. It should be made with ``channel'' linetype.
\keyword{\bf SELECT}
The standard uv selection subcommands.
\keyword{\bf OPTIONS}
This gives extra processing options. Several values can be given
(though many values are mutually exclusive), separated by commas.
Option values can be abbreviated to uniqueness.
Possible options are:
{\eightpoint\begintt
  add       Form: out = vis + model (default).
  subtract  Form: out = vis - model
  multiply  Form: out = vis * model
  divide    Form: out = vis / model
  replace   Form: out = model
  flag      Form: out = vis, but flag data where the difference
            between vis and model is greater than "sigma" sigmas.
  unflag    Unflag any flagged data in the output.
  autoscale Adjust the scale of the model to minimise the difference
            between the model and the visibility.
  apriori   Use flux from flux table or data file planet info.
  imhead    Copy image header items to uvvariables.
\endtt}
The operations add, subtract, multiply, divide, replace and flag are
mutually exclusive. The operations flag and unflag are also mutually
exclusive.

The unflag option should be used with caution. Data in the output
may still be flagged, if it was not possible to calculate the
model.
\keyword{\bf CLIP}
Clip level. Pixels in the model below this level are set to zero.
The default is 0.
\keyword{\bf FLUX}
If MODEL is blank, then the flux (Jy) of a point source model should
be specified here. The default is 1 (assuming the model parameter
is not given).
\keyword{\bf OFFSET}
The RA and DEC offsets (arcseconds) of the point source from the phase
centre. A point source to the north and east has positive offsets.  
Defaults are zero.
\keyword{\bf SIGMA}
For options=flag, UVMODEL flags those points in the output that
differ by more than ``sigma'' sigmas. The default is 100.
\keyword{\bf OUT}
Output visibility data file name. No default. The output file will
contain only as many channels as there are planes in the model
cube. The various uv variables which describe the windows are
adjusted accordingly.
\par}
\module{uvplot}%
\noindent Plot selected uvdata variables and linetypes versus time.
\newline \
\newline \abox{Responsible:} Mel Wright
\newline \abox{Keywords:} uv analysis and display.
\newline{\tenpoint\newline
UVPLOT makes a table of uvdata variables and linetypes versus time.
The table can be plotted or written into a logfile for further
analysis or for input to MONGO for plotting.
Column one is always the Julian day. Only 10 columns can be written
into the logfile and extra inputs are discarded. However, up to 30
items can be plotted with amplitude and phase for each linetype being
plotted in the same plot.
\keyword{\bf VIS}
The input uv dataset name. No default.
\keyword{\bf DEVICE}
Plot device/type (e.g. /retro, filename/im for hardcopy on a Vax,
/sunview, filename/ps for harcopy on a Sun). The default is no plot.
To print the hardcopy on the Vax use:
{\eightpoint\begintt
  $PGPLOT filename
\endtt}
To print the postscript file on the Sun:
{\eightpoint\begintt
  %lpr filename
\endtt}
\keyword{\bf LOG}
The output logfile name. The default is no output logfile.
\keyword{\bf TITLE}
A title which is put on the plot and written into the logfile.
\keyword{\bf AVERAGE}
The averaging time interval in seconds. The default is 1s. For the
rms to work the averaging time must be more than the integration time.
A new average is started if the source name changes.
\keyword{\bf VAR}
List of variables to be plotted. The default is no variables. \newline
E.g.    var=systemp(1),systemp(2),windmph,axisrms(1) \newline
The default subscript is 1. For a list of variables, use UVLIST.
\keyword{\bf ANTENNAS}
Antenna select of vla form for use with linetypes:
antennas  WITH  antennas  (if WITH is missing, both sides equal)
where antennas is an antenna list of the form: ant1\_ant2\_antn
and means all baselines that can be made from the named antennas
antennas can also be  *   which means all possible antennas
or antennas can be  *-antn\_antm  which means all ants - antn and antm.
\keyword{\bf LINE}
Line type of the data in the format
{\eightpoint\begintt
  linetype,nchan,start,width,step
\endtt}
Here ``linetype'' can be `channel', `wide' or `velocity'.
The default is channel,1,1,1,1.
The linetype is qualified by the keywords, 'antennas', and 'type'
and should not include the modifiers -amp, -phase etc. since these
are described by the keyword 'type' below.
\keyword{\bf TYPE}
Select 'amp' 'phase' or 'rms' to be plotted. These can be combined.
Select 'amp' 'phase' or 'rms' to be plotted. These can be combined.
{\eightpoint\begintt
  E.g. 'amphaserms' will produce 4 columns for each line and baseline
\endtt}
selected. The default is 'amp'.
\keyword{\bf REF}
Line type of the reference channel, specified in a similar to the
``line'' parameter. Specifically, it is in the form:
{\eightpoint\begintt
  linetype,start,width
\endtt}
The default is no reference channel.
\par}
\module{uvplt}%
\noindent Plot visibility amplitude or phase versus time or uv-distance.
\newline \ 
\newline \abox{Responsible:} Mel Wright
\newline \abox{Keywords:} uv analysis, plotting
\newline{\tenpoint\newline
UVPLT is a Miriad task which plots amplitude or phase
versus time or uv-distance.
\keyword{\bf VIS}
The input visibility file. No default.
\keyword{\bf LINE}
Line type of the data in the format
{\eightpoint\begintt
  type,nchan,start,width,step
\endtt}
Here ``type'' can be `channel', `wide' or `velocity'.
Default is channel,1,1,1,1
\keyword{\bf SELECT}
This selects which visibilities to be used. Default is
all visibilities. See the Users Guide for information about
how to specify uv data selection.
\keyword{\bf FMODE}
Flagging type to plot.  'a' to plot flagged and unflagged
'u'  to plot unflagged data, 'f' to plot flagged data.
Default is to plot unflagged data
\keyword{\bf XAXIS}
Type of x-axis, either 'uvdist' or 'time', Default is 'uvdist'
\keyword{\bf YAXIS}
Type of y-axis, either 'amp' or 'phase'. Default is 'amp'
\keyword{\bf XRANGE}
Plot range in the x-direction (in wavelengths or days). Default 
is to self-scale.
\keyword{\bf YRANGE}
Plot range in the y-direction (in intrinsic units).  Default is
to self-scale.
\keyword{\bf DEVICE}
Plot device/type (e.g. /retro, plot.plt/im for hardcopy on a Vax,
or  /sunview, plot/ps for harcopy on a Sun).  No default.
If device is unset, the user is prompted interactively for the
plot device, with the option to redefine the plotting window.
In this case points outside of x,y-range are included in the
plot buffer.  After each plot is drawn, the user should hit
{\tt <}cr{\tt >} when ready, and the program will then prompt for window
redefinition.   
\keyword{\bf LOG}
The output logfile name. The default is the terminal.
\keyword{\bf COMMENT}
A one line comment which is written into the logfile.
\par}
\module{uvpoint}%
\noindent Generate artifial mosaiced observation
\newline \ 
\newline \abox{Responsible:} Mel Wright
\newline \abox{Keywords:} uv analysis, map making
\newline{\tenpoint\newline
UVPOINT generates a uv data file of a mosaiced observation of point
sources. A random uv coverage is generated, with a distribution
function roughly that of the VLA. The observation is for a three
element array, with 60 sec integration time, starting on New Years
1980. Source velocity is 0, no doppler velocity, RA and DEC of 45 deg. 
\keyword{\bf OFFSET}
Offset of the point sources, in arcsec, from the observation center.
Two values are given per source. Several pairs of values can be given,
corresponding to different sources. Default is 0,0.
\keyword{\bf FLUX}
Source flux value. A value can be given for each source. Default is 1.
\keyword{\bf RMS}
RMS of the noise to add to the data. Default is 0.
\keyword{\bf PBFWHM}
The primary beam fwhm, in arcseconds. Default is to use the Hat
Creek primary beam size.
\keyword{\bf CENTER}
Offset pointing centers for the mosaiked observation, in arcseconds.
Two values (x and y offset) are required per pointing center. Several
values can be given. Default is 0,0.
\keyword{\bf UVRANGE}
UV range of the observation, in kilowavelengths. Default is 0.25,10.
\keyword{\bf FREQ}
Observing frequency, in GHz. Default is 100.
\keyword{\bf NVIS}
Number of visibilities to generate. Default is 10000.
\keyword{\bf OUT}
Name of the output visibility file. No default.
\par}
\module{uvputhd}%
\noindent Allows user to alter values of header variables in uv dataset
\newline \ 
\newline \abox{Responsible:} Lee Mundy
\newline \abox{Keywords:} vis, uv, header
\newline{\tenpoint\newline
UVPUTHD allows the user to change the values of variables in a
uv dataset. All occurances of the variable are changed to the
new value. If the variable is an array, all new values must be
entered in sequential order. If the user desires to set all members 
of an array to a single value, only one value need be entered.
\keyword{\bf VIS}
Name of the input MIRIAD data file. Only one file at a time!!
\keyword{\bf HDVAR}
Name of header variable to be changed. Refer to user manual or
run VARLIST to see the selection of allowed variable names.
\keyword{\bf TYPE}
Type of variable, either integer (i),real (r),double precision (d), 
or ascii (a). Unused if variable already exists in the data file.
\keyword{\bf LENGTH}
Length of array of variable values. Unused if variable already 
exists in data file.
\keyword{\bf NVALS}
Number of new values for varibles that are being entered in varval.
\keyword{\bf VARVAL}
New values of header variable - if the variable is an array
all values must be specified or will assume one value for all.
\keyword{\bf OUT}
Name of the output data set.
\par}
\module{uvspect}%
\noindent Plot visibility amplitude/phase vs velocity
\newline \ 
\newline \abox{Responsible:} Mel Wright
\newline \abox{Keywords:} uv analysis, plotting
\newline{\tenpoint\newline
UVSPECT is a Miriad task which plots visibility spectra for
uvdata as Amplitude and Phase versus channels or velocity.
\keyword{\bf VIS}
The input visibility file. No default.
\keyword{\bf DEVICE}
Plot device/type (e.g. /retro, plot.plt/im for hardcopy on a Vax,
or  /sunview, plot/ps for harcopy on a Sun).  No default.
\keyword{\bf AVER}
Averaging time in hours. Default is 1 hour.
\keyword{\bf SELECT}
This selects which visibilities to be used. Default is
all visibilities. See the Users Guide for information about
how to specify uv data selection.
\keyword{\bf LINE}
Line type of the data in the format
{\eightpoint\begintt
  type,nchan,start,width,step
\endtt}
Here ``type'' can be `channel', `wide' or `velocity'.
Default is all channels.
\keyword{\bf FMODE}
Flagging type to plot. Set fmode=a to plot flagged and unflagged
data. fmode=u to plot unflagged data, fmode=f to plot flagged data.
Default is to plot all data.
\keyword{\bf XRANGE}
Plot range in the x-direction (in channels or km/s according to
the linetype selected). Default is to self-scale.
\keyword{\bf AMPRANGE}
Plot range for amplitude in the y-direction (in intrinsic units).
Default is to self-scale.
\keyword{\bf PHIRANGE}
Plot range for phase in the y-direction (in degrees).
Default is (-180,180) degrees.
\keyword{\bf MODE}
Mode can be 'inter' or 'batch'. Default is batch. In either mode,
the user is prompted for options after each plot on a non-hardcopy
device. The options are applied to the current spectrum, which can
then be replotted. mode=inter displays the cursor. Type the first
character and {\tt <}cr{\tt >} to select option. The cursor handling is device
dependent, and the character and {\tt <}cr{\tt >} may need to given several times.
Options:
{\eightpoint\begintt
  Device - enter new plot device.
  Limits - change plot limits.
  Velocity - change x-axis to velocity.
  Replot - replot current spectrum.
  Position - display cursor position. (if device supports a cursor)
  Write - write out current spectrum to an ascii file.
  Quit - stop processing and exit from task.
  End - end of options, get next average.
  <cr> - go to next time average.
\endtt}
The original inputs are restored for the next time average.
\par}
\module{uvtrack}%
\noindent Plot u-v tracks for n configurations of m antennas
\newline \ 
\newline \abox{Responsible:} Mel Wright
\newline \abox{Keywords:} uv analysis
\newline{\tenpoint\newline
Plot theoretical u-v tracks for n configurations of m antennas 
and create model uvdata to make a beam.
\keyword{\bf MODE}
Mode can be 'inter' or 'batch'. Batch mode generates uvdata
and the u-v tracks are plotted if a plot device is given.
In interactive mode the user can read and write antenna arrays,
add and delete antenna configurations, generate model uvdata,
and plot u-v tracks for existing or model uvdata. Interactive
mode also supports MONGO plot options on the Vax.
Default mode is interactive and no inputs need be given.
\keyword{\bf SOURCE}
Source name. Default=DECxxxx
\keyword{\bf DEC}
Source declination in degrees. Default=30 degrees.
\keyword{\bf HARANGE}
Hour Angle range (start,stop,step) in hours. Default is
to use the elevation limit, with a step=0.2 (12 minute)
\keyword{\bf ELEV}
Elevation limit in degrees. Default=10 degrees.
\keyword{\bf FREQ}
observing frequency in GHz. Default=100 Ghz.
\keyword{\bf LAT}
Latitude of observatory, in degrees. Default=40 degrees.
\keyword{\bf ANT}
The name of a text file containing the antenna positions.
No default. For mode=inter, the task prompts the user for
the antenna positions, and then generates the antenna file.
\keyword{\bf DEVICE}
PGPLOT device: plot showing uv tracks. No default.
\keyword{\bf PLIM}
Plot limit in nanosecs. Default is to autoscale.
\keyword{\bf OUT}
The name of the output uv data set. No default.
\par}
\module{varlist}%
\noindent List all variables in dataset
\newline \ 
\newline \abox{Responsible:} Lee Mundy
\newline \abox{Keywords:} utility, plotting
\newline{\tenpoint\newline
VARLIST lists all variable names or names, types and length
in a data set.
\keyword{\bf VIS}
The name of the input visibility file. No default.
\keyword{\bf LOG}
Name of output file
\keyword{\bf OPTION}
Option for printout (names or all).  Default is ``all.''
\par}
\module{varplot}%
\noindent Plot uv variables
\newline \ 
\newline \abox{Responsible:} Lee Mundy
\newline \abox{Keywords:} uv analysis, plotting
\newline{\tenpoint\newline
VARPLOT makes X,Y plots selected variables from a uv data set.
Only integer, real, and double precision variables maybe plotted.
When curser is in the plot window, the following keys are active:
{\eightpoint\begintt
   X - expand window in X to give one column of plots
   Y - expand window in Y to give one row of plots
   Z - expand window in both X and Y to show only one plot
   N - step to "next" plot in x or y or both depending on expansion
\endtt}
\keyword{\bf VIS}
The name of the input visibility file. No default.
\keyword{\bf DEVICE}
Standard device keyword. See section ``TV Devices''
\keyword{\bf XAXIS}
Name of variable to be plotted along x-axis (default = ut time).
\keyword{\bf YAXIS}
Name of variable to be plotted along y-axis (no default)
\keyword{\bf MULTI}
Make multiple plots or a single plot? Yes yields multiple plots
No yields a single plot with multiple lines as needed
(default = yes)
\keyword{\bf COMPR}
Compress number of x or y variables to be plotted by averaging
over spectral windows. Currently only SYSTEMP can be averaged.
\par}
\module{velplot}%
\noindent Interactive task to Average, Graph, and Extract Miriad Images.
\newline \ 
\newline \abox{Responsible:} Mel Wright
\newline \abox{Keywords:} analysis, plotting
\newline{\tenpoint\newline
VELPLOT is an interactive task to Average, Graph, and Extract
regions from Miriad Images. There are three basic options:

{\eightpoint\begintt
 - Velocity-averaged x-y images over selected velocity intervals.
   Can also make mean velocity and velocity dispersion images.
 
 - Spectra at selected x-y positions. Can be convolved by a beam
   in the x-y plane, and smoothed in velocity.
 
 - Position-velocity maps along user selected position angles
   in the x-y plane. Can be convolved in the x-y plane.

\endtt}
The output from each option can be plotted or listed, or written
out as Miriad images of velocity-averaged maps, position-velocity maps,
or ascii spectra respectively. \newline \ 

The task saves user-defined lists of velocity-averaged maps, position
-velocity maps, and spectra positions. These lists can be created and 
edited within the task, and can be passed between the three options. 
For example, a list of spectra or position-velocity cuts can be created 
with the cursor whilst displaying velocity-averaged maps. These lists
can then be used to display the spectra and position-velocity cuts.

Each option provides choice of contour levels, map units, and
interactive help if requested.
\keyword{\bf IN}
Input image name. No default.
\keyword{\bf DEVICE}
The MONGO plotting device. Default is 0 for Tectronics terminal.
\keyword{\bf REGION}
Region of image to be plotted. E.g.
{\eightpoint\begintt
  % image region=relpix,box(-30,-30,20,90)(16,57)
\endtt}
reads in 50 x 120 pixels of image planes 16 to 57.
The default is the whole Image. The task issues a warning to exit,
or will run much more slowly, if too large a region is requested.
\keyword{\bf LOG}
The output log file. The default is standard output.
\par}
\module{views}%
\noindent Generate a projection of the datacube
\newline \ 
\newline \abox{Responsible:} Bart Wakker
\newline \abox{Keywords:} visual display
\newline{\tenpoint\newline
VIEWS is a MIRIAD task which generates projections of a cube, viewed
as if the cube was rotating around its y axis. The view is also
from an angle ``phi'' to the x-z plane. The output is intended to be
fed into a tvmovie program, to view a rotating cube. The projection is
a thresholding operation -- pixels values less than the threshold are
transparent, a pixel greater than the threshold is opaque, with a
brightness given by the pixel value.
\keyword{\bf IN}
The name of the input image. No default.
\keyword{\bf REGION}
Standard region keyword. See section ``Image region of interest''.
\keyword{\bf OUT}
The name of the output image. No default.
\keyword{\bf PHI}
The elevation angle, with respect to the x-z plane, of the observer
(degrees). This must be in the range [0,90]. Default is 30.
\keyword{\bf ASPECT}
Aspect ratio of the different axes. Two values can be given, giving
the y/x aspect ratio, and the z/x aspect ratio. Default should be
fine.
\keyword{\bf THRESH}
Threshold used in determining which details appear in the output.
Default is the average of the image minima and maxima.
\keyword{\bf IMSIZE}
Size of the output cube. Default is 128x128x64 (should be adequate).
\par}
\module{zeeeta}%
\noindent Compute Zeeman parameter eta
\newline \ 
\newline \abox{Responsible:} Neil Killeen
\newline \abox{Keywords:} profile analysis
\newline{\tenpoint\newline
ZEEETA is a MIRIAD task to compute the Zeeman S/N parameter 
{\eightpoint\begintt
eta = (dI/dnu)\_rms / sigma
\endtt}
\keyword{\bf IN}
The input Stokes I image in vxy order. No default.
\keyword{\bf OUT}
The output eta image, if blc,trc is not specified
\keyword{\bf SIGMA}
r.m.s. noise of signal free I spectrum
\keyword{\bf CHAN}
The channel range. Default is all channels.
\keyword{\bf BLC}
Specify the blc of the spatial window in pixels.  If unset,
the whole image is done, with no spatial summing, and
an output eta image made.  If blc and trc are set, then
eta is computed for this window, and the results output to
the terminal.
\keyword{\bf TRC}
Specify the trc of the spatial window in pixels
\keyword{\bf DER}
1 or 2 for one or two sided derivative.  If you would
like to see a plot of the last derivative spectrum, include
a `p' as well.  E.g., 2p or 1p
\keyword{\bf AVEOP}
a' to average spectra before computing eta. Only if blc,trc set
\par}
\module{zeefake}%
\noindent Generate fake Zeeman data.
\newline \ 
\newline \abox{Responsible:} Neil Killeen
\newline \abox{Keywords:} profile analysis
\keyword{\bf IOUT}
The output Stokes I cube. No default.
\keyword{\bf IUOUT}
The output Stokes I cube in the case of no splitting.  Default
is not to write this cube out.
\keyword{\bf VOUT}
The output Stokes V cube. No default.
\keyword{\bf IMSIZE}
The output cube sizes, all three dimensions required (VXY)
\keyword{\bf VINC}
The velocity increment along the cubes in Km/s. No default.
\keyword{\bf DELV}
The Zeeman splitting (separation of split lines) in Km/s. No default.
\keyword{\bf FWHM}
The FWHM of the Gaussian line profile in Km/s. No default.
\keyword{\bf GRVC}
The velocity increment of the line centre in the x-direction 
per pixel (km/s).  Makes a linear ramp across the source.
\keyword{\bf GRINT}
Intensity increment in the x direction per pixel. Makes
a triangular weighting function.  Peak response is 1.0
\keyword{\bf GRFWHM}
FWHM increment in the x direction per pixel.  Makes
a linear ramp across the source.
\keyword{\bf GRSPLIT}
Splitting increment in the x direction.  Makes a linear ramp
across the source.
\keyword{\bf THETA}
The angle of the magnetic field to the l-o-s in degrees. No default.
\keyword{\bf NOISE}
The r.m.s. noise to be added to the RR and LL responses. Note
that the peak RR or LL response in this program is 1.0 for theta=0
\keyword{\bf RESTFREQ}
The rest frequency of your line in GHz.
\keyword{\bf TYPE}
The line type; ``e'' for emission (default) ``a'' for absorption
\module{zeemap}%
\noindent Map the magnetic field from I and V cubes
\newline \ 
\newline \abox{Responsible:} Neil Killeen
\newline \abox{Keywords:} profile analysis
\newline{\tenpoint\newline
ZEEMAP tries to find the line-of-sight magnetic field splitting 
a transition by measuring the shift in the line owing to the Zeeman 
effect.
\keyword{\bf IIN}
Stokes I spectral cube, with axes in order Channel, X, Y.
\keyword{\bf VIN}
Stokes V spectral cube, with axes in order Channel, X, Y.
\keyword{\bf B}
Output file name for Line-of-sight magnetic field.
\keyword{\bf G}
Output file name for amount of residual I found in V (FIT only)
\keyword{\bf D}
Output file name for the derivative of I/2 spectrum (DIV only)
\keyword{\bf BE}
Output file name for the error in B (DIV only)
\keyword{\bf BLC}
Bottom left corner of box to be fit
\keyword{\bf TRC}
Top right corner of box to be fit
\keyword{\bf FREQ}
Frequency of line in GHz. This is used to determine amount of
splitting per Gauss.
\keyword{\bf OP}
Operation code. Possible values are ``div'' and ``fit''. For
``div'', the line shift is determined by dividing V by the
derivative of I/2. For ``fit'', the line shift is determined
by fitting of the derivative of I and I to V.
\keyword{\bf CUTOFF}
For ``div'', the division will not be performed unless the
derivative of I/2 is greater than cutoff.
For ``fit'', the fit will not be performed unless one or more
channels in the I spectrum is larger in absolute
value than cutoff.
\keyword{\bf VRMS}
RMS deviation in V spectrum in a signal free area.  This can easily
be gotten using histo.  Used for calculating error in B for op=div.
\keyword{\bf MODE}
This determines the algorithms used, if operation=``fit''. It
can consist of several flags:
{\eightpoint\begintt
 l  Include leakage term.
 2  Use 2 sided derivative approximation.
 m  Use maximum likelihood technique.
 x  Perform extra checks for better solutions, when
    using the maximum likelihood technique.
\endtt}
Default mode=2m
\keyword{\bf Y CORLEN}
Correlation length.  Selected window must be divisible by this
number, which is the number of pixels over which you think the
magnetic field might be constant.  Default is 1, try maybe 2 or 4.
\par}
\module{zeesim}%
\noindent Test reliability of Zeeman splitting errors
\newline \ 
\newline \abox{Responsible:} Neil Killeen
\newline \abox{Keywords:} profile analysis
\newline{\tenpoint\newline
Estimate the reliability of Zeeman error estimates for spatial 
summing or averaging.    This is generally essential for spatial
summing, and possibly needed for spatial averaging when eta is 
of the order of unity.

Program in four logical sections
1)  Read in data
2)  Fit data
3)  Find the FFT of the beam
4)  Simulate 
\keyword{\bf IIN}
Input I cube. No default.
\keyword{\bf VIN}
Input V cube. No default.
\keyword{\bf BEAM}
The beam of the observation.  Not needed for AVEOP=`a'
\keyword{\bf MODE}
This is a character string that determines the algorithm used 
in the fitting process. It consists of several flags, which 
can be:
{\eightpoint\begintt
  m Use maximum likelihood technique.
  l Include a leakage term in the fitting.
  2 Use a two sided derivative estimate.
  x Perform extra checks for better solutions when using the 
    maximum likelihood technique.
  d Debiased least squares estimate.
\endtt}
The default is ' ' i.e., least squares and 1.
\keyword{\bf AVEOP}
a' for averaged spectrum in window, `s' for summed spectra
ummed computational load is orders of magnitude greater than for 
veraged spectra.  'h' for hybrid of averaging and summing.
efault is `a'.  The hybrid is the same as `s', except when 
he simulated spectra, whereupon the spectra are first
veraged for each window.
\keyword{\bf CHAN}
Channel range. Default is all channels.
\keyword{\bf FREQ}
Frequency (GHz) for conversion of channel splitting to B field.
\keyword{\bf BLC}
Bottom left corner of spatial region to examine. Default is (1,1).
\keyword{\bf TRC}
Top right corner of spatial region to examine. Default is all
of image.
\keyword{\bf BIN}
Binning widths for all thre (v,x,y) dimensions. Default = 1. 
\keyword{\bf SPLIT}
Splitting (split to unsplit in channels) to use for calculation
of hatV and all simulation (i.e., don't use actual splitting
as predicted by fitting algorithm when generating V spectra.
\keyword{\bf NRUNS}
The number of simulation runs to undertake.   If 0, then just
the initial fits are done.  Default is 0.
\keyword{\bf INFILE}
File containing a list of windows.  If this file specified,
blc and trc are ignored.  Should be in format:
{\eightpoint\begintt
  NWIN
  I XCEN YCEN XOFF YOFF
\endtt}
And there are nwin of these lines.  XOFF and YOFF define the
box half-sizes from their specified centres, XCEN and YCEN.
You can leave off XOFF and YOFF and they will default to 2
All units are pixels.
\keyword{\bf LOG} \newline \ 
If NRUNS = 0
{\eightpoint\begintt
  One value which is a root file name, appended to which is
  the box number as specified in the INFILE (or = 1 for
  a box specified with BLC and TRC).   These files contain
  the fitted results for each window.   The files are opened
  in APPEND mode.
\endtt}
If NRUNS {\tt >} 0
{\eightpoint\begintt
  Two values.  The first is a root file name, appended to which 
  is the box number as specified in the INFILE (or = 1 for
  a box specified with BLC and TRC).   These files contain the
  results for each simulation with a cumulative fiddle factor
  worked out.  The first line is the initially fit results
  before simulation.
  The second value is a file containing a statistical summary
  of the final results of the simulations from all windows, 
  including the initially fit results before simulation began.
\endtt}
\keyword{\bf NRAN}
Throw away NRAN random numbers before starting.  Use this to 
continue a set of simulations with different random numbers.
\par}
\module{zeestat}%
\noindent Fits Zeeman parameters to I and V cubes
\newline \ 
\newline \abox{Responsible:} Neil Killeen
\newline \abox{Keywords:} model fitting
\newline{\tenpoint\newline
ZEESTAT is a Miriad task that fits Zeeman parameters to a region of
input I and V cubes. Zeestat also analyses the noise correlations
to determine a factor used in correcting the sigma estimates.
\keyword{\bf IIN}
Input I cube. No default.
\keyword{\bf VIN}
Input V cube. No default.
\keyword{\bf BEAM}
The beam of the observation. This is used in determining the sigma
correction factor. The default is a delta function. You should
give the beam if you want an accurate estimate of sigma.
\keyword{\bf RHO}
The channel-to-channel correlation coefficient. This is used in
determining the sigma correction factor. The default is 0.
\keyword{\bf FREQ}
Frequency (GHz) of splitting. No default.
\keyword{\bf MODE}
This is a character string that determines the algorithm used in the
fitting process. It consists of several flags, which can be:
{\eightpoint\begintt
  m Use maximum likelihood technique.
  l Include a leakage term in the fitting.
  2 Use a two sided derivative estimate.
  x Perform extra checks for better solutions when using the maximum
    likelihood technique.
\endtt}
The default is ' ' i.e., least squares and 1.
\keyword{\bf AVEOP}
If 'a' then the spectra specified by the spatial window are
averaged before being passed to the fitting routine.  Otherwise,
and by default, all spectra are passed to the fitting routine
and ``spatial summing'' (as defined by SKLZ) is performed.
\keyword{\bf CHAN}
Channel range. Default is all channels.
\keyword{\bf BLC}
Bottom left corner of spatial region to examine. Default is (1,1).
\keyword{\bf TRC}
Top right corner of spatial region to examine. Default is all
of image.
\keyword{\bf LOG}
This is the name of an output logfile, which shows how the error
estimate varies with the alpha parameter.
\par}
