%------------------------------------------------------------------------
% Appendix - Image Items
%------------------------------------------------------------------------
%  History
%
%  --mar91 mjs  Original version.
%  23jun91 mjs  cdelt/crpix/crval are type "double", not "real"
%
%------------------------------------------------------------------------

\beginsection Discussion of MIRIAD Image Items

This list describes items that may be present in MIRIAD image data
sets.  A number of MIRIAD image items have their origins in FITS,
and consulting that standard will provide additional discussion on the
items.  These include (where {\bf N} is the axis number):
{\ninepoint\begintt
-------- -------- -------- -------- -------- -------- -------- --------
crpixN   crvalN   ctypeN   datamin  datamax  date-obs epoch    history
instrume naxis    naxisN   object   observer telescop
-------- -------- -------- -------- -------- -------- -------- --------
\endtt}
Data storage types have the following meanings:
\vskip 0.2in
{\ninepoint\parskip=0.0in
{\tt char ......} A null-terminated character string, 8 characters or
less. \newline
{\tt double ....} IEEE 64-bit floating point. \newline
{\tt integer ...} A 32-bit signed integer. \newline
{\tt real ......} IEEE 32-bit floating point. \newline
{\tt text ......} A string of characters, not null-terminated. \newline
}
\vskip 0.4in
{\ninepoint\raggedright
\tabskip=0em
\halign {#\tabskip=2em&\tabskip=2em&#\tabskip=2em&#\hfil\cr

{\underbar{Item}}&{\underbar{Type}}&{\underbar{Units}}&{\underbar{Description}}\cr
\cr
{\bf bmaj}      &real   &radians    &Beam major axis, full width at\cr
                &       &           &half maximum.\cr
\cr
{\bf bmin}      &real   &radians    &Beam minor axis, full width at\cr
                &       &           &half maximum.\cr
\cr
{\bf bpa}       &real   &degrees    &Beam position angle.\cr
\cr
{\bf bunit}     &char   &-          &The units of the pixels, either\cr
                &       &           &{\tt JY/BEAM} or {\tt JY/PIXEL}\cr
                &       &           &(same as FITS BUNIT keyword).\cr
\cr
{\bf cdelt1},   &double &(see Desc.)&The increment between pixels of the\cr
{\bf cdelt2},   &       &           &axes.  Units are radians for RA or\cr
...             &       &           &DEC, GHz for frequency, and \cr
                &       &           &km/s for velocity.\cr
\cr
{\bf crpix1},   &double &-          &The pixel coordinate of the reference\cr
{\bf crpix2},   &       &           &pixel of the axes.\cr
...             &       &           &\cr
\cr
{\bf crval1},   &double &(see Desc.)&The coordinate value at the reference\cr
{\bf crval2},   &       &           &pixel for the axes.  Units are radians\cr
...             &       &           &for RA and DEC, GHz for frequency, and\cr
                &       &           &km/s for velocity.\cr
\cr
{\bf ctype1},   &char   &-          &The type of the axes.  Values are the\cr
{\bf ctype2},   &       &           &same as FITS CTYPE keywords:\cr
...             &       &           &{\tt RA---SIN} for RA,\cr
                &       &           &{\tt DEC--SIN} for DEC;\cr
                &       &           &coordinate as velocity ({\tt VELO-LSR})\cr
                &       &           &or frequency ({\tt FREQ}).\cr
\cr
{\bf datamax}   &real   &-          &The maximum pixel value.\cr
\cr
{\bf datamin}   &real   &-          &The minimum pixel value.\cr
\cr
{\bf date-obs}  &char   &-          &Format is 'dd/mm/yy'.\cr
\cr
{\bf epoch}     &real   &years      &The epoch of the coordinate system.\cr
                &       &           &(FITS EQUINOX)\cr
\cr
{\bf history}   &text   &-          &The history of processing performed on\cr
                &       &           &the data set.\cr
\cr
{\bf image}     &real   &-          &The pixel data.\cr
\cr
{\bf instrume}  &char   &-          &The data acquisition instrument.\cr
\cr
{\bf lstart}    &real   &-          &Linetype start channel or velocity used\cr
                &       &           &in making image.\cr
\cr
{\bf lstep}     &real   &-          &Linetype step size used in making image.\cr
\cr
{\bf ltype}     &char   &-          &Linetype used in making image.\cr
\cr
{\bf lwidth}    &real   &-          &Linetype width used in making image.\cr
\cr
{\bf mask}      &-      &-          &A bitmap used to determine which pixels\cr
                &       &           &in the image have been blanked.\cr
\cr
{\bf naxis}     &integer&-          &The number of dimensions.\cr
\cr
{\bf naxis1},   &integer&-          &The number of pixels along the dimension.\cr
{\bf naxis2},   &       &           &\cr
...             &       &           &\cr
\cr
{\bf niters}    &integer&-          &The total number of deconvolution\cr
                &       &           &iterations performed on the image.\cr
\cr
{\bf object}    &char   &-          &Source name (e.g., {\tt CHICYG}).\cr
\cr
{\bf observer}  &char   &-          &Observer name, identification.\cr
\cr
{\bf obsdec}    &double &radians    &The apparent DEC of the phase center of\cr
                &       &           &the observation.\cr
\cr
{\bf obsra}     &double &radians    &The apparent RA of the phase center of\cr
                &       &           &the observation.\cr
\cr
{\bf pbfwhm}    &integer&-          &The primary beam, full width at half\cr
                &       &           &maximum.  Its value is {\tt 1} if the\cr
                &       &           &primary beam is unknown.  Its value is\cr
                &       &           &{\tt 0} if there is no primary beam\cr
                &       &           &apodization (e.g., primary beam\cr
                &       &           &correction has been applied, or single\cr
                &       &           &dish antenna).\cr
\cr
{\bf restfreq}  &double &GHz        &The rest frequency of the observed data.\cr
\cr
{\bf telescop}  &char   &-          &Data acquisition telescope (e.g.,\cr
                &       &           &{\tt HATCREEK}).\cr
\cr
{\bf vobs}      &real   &km/s       &Velocity of the observatory, with\cr
                &       &           &respect to the rest frame, during the\cr
                &       &           &observation.\cr
\cr
{\bf xshift}    &real   &radians    &Shift in RA of the map center from the\cr
                &       &           &phase center, when mapping.\cr
\cr
{\bf yshift}    &real   &radians    &Shift in DEC of the map center from the\cr
                &       &           &phase center, when mapping.\cr
}
}

\beginsection RALINT header and Corresponding MIRIAD Image Items

{\parskip=0.0cm
\def\bbox#1{\hbox to 2.2cm{#1\hfil}}
\bbox{xy}pixel size (") \newline
\bbox{ras decs}coordinate at the reference pixel (radians) \newline
\bbox{dras ddecs}pointing center offset from (ras,decs) (radians) \newline
\bbox{vel,delv}lsr velocity and width of this map (km/s) \newline
\bbox{rfrq}restfreq (GHz) \newline
\bbox{freqs}sky frequency at time of first observation (GHz) \newline
\bbox{ }freqs = restfreq(1-(vobs-vsource)/ckms) \newline
\bbox{cbmaj,} \newline
\bbox{cbmin,} \newline
\bbox{cbpa}beam size (") and position angle (degrees) \newline
\bbox{PLMAJS,} \newline
\bbox{PLMINS,} \newline
\bbox{PLANGS}reference planet axes (arcsecs, arcsecs, degrees) \newline
\bbox{ } \newline
}
linetype, svel, sk are the line type, 1st and last channel, velocity and
width used by subroutine ALINE when gridding the data, see aline.hlp;
position velocity plots change linetype(1) to 'LV' \newline
maptype is a coded combination of bunit and ctype1, ctype2, and ctype3

\$MIR/src/prog/vis/hatmap.h defines the Ralint 2-d map header.  There is also
a 3-d Ralint format, with Image of size (nx,ny,nz) with velocities given by
the array vlsr. \newline
The coordinates and reference pixel are give w.r.t. the original Ralint
2d map.
