%------------------------------------------------------------------------
% Chapter 8 - uv Datasets
%------------------------------------------------------------------------
% History:
%   mjs  23mar91  Adapted from the old Programmer's Guide.
%------------------------------------------------------------------------
\beginsection{uv Datasets}

\beginsub{General}

At the very least, a uv dataset can be viewed as a sequence of 
correlation
records, with associated u and v coordinates, time and baseline number.
Associated with each correlation is a flag, indicating whether the
correlation is believed to be good or bad.

The MIRIAD uv data structure required a more general structure. Unfortunately
this is more complicated and somewhat cumbersome for simple cases.
A uv dataset can be viewed as an ordered (generally time ordered) stream of
named records or ``variables''. There are markers in this data stream, to
indicate when
several variables change ``simultaneously'' (i.e. they correspond to the
same time). Each variable
consists of an array of values, the type of which can be either integer, real
or double precision, etc. Correlation data, u and v coordinates,
time and baseline numbers are specific examples of variables.
Because of the special nature of these variables, special routines are
available to simplify accessing them. A list of the variables
that may be present in a uv dataset is given in the appendices.

In addition to this variable stream, a uv dataset will contain a file giving
flagging information.

It should be noted that ``variables'' and ``items''
are quite distinct. For a particular dataset, variables vary, or at least may
vary,
whereas data items are fixed. The notion of variables is unique to uv datasets,
whereas all MIRIAD datasets are composed of data items. The stream of
uv variables is implemented as three data items, called {\tt visdata},
{\tt vartable} and {\tt flags}.

There are many uv routines.  The routines used to access and manipulate a
uv dataset are given below.
{\ninepoint\begintt

      subroutine uvopen(tno,dataname,status)
      subroutine uvread(tno,preamble,data,flags,n,nread)
      subroutine uvflgwr(tno,flags)
      subroutine uvwflgwr(tno,flags)
      subroutine uvwrite(tno,preamble,data,flags,n)
      subroutine uvclose(tno)
      subroutine uvrewind(tno)

      subroutine uvwread(tno,data,flags,n,nread)
      subroutine uvwwrite(tno,data,flags,n)

      subroutine uvgetvra(tno,varname,data)
      subroutine uvgetvri(tno,varname,data,n)
      subroutine uvgetvrr(tno,varname,data,n)
      subroutine uvgetvrd(tno,varname,data,n)
      subroutine uvgetvrc(tno,varname,data,n)

      subroutine uvrdvra(tno,varname,data,default)
      subroutine uvrdvri(tno,varname,data,default)
      subroutine uvrdvrr(tno,varname,data,default)
      subroutine uvrdvrd(tno,varname,data,default)
      subroutine uvrdvrc(tno,varname,data,default)

      subroutine uvprobvr(tno,varname,type,length,update)

      subroutine uvputvra(tno,varname,data)
      subroutine uvputvri(tno,varname,data,n)
      subroutine uvputvrr(tno,varname,data,n)
      subroutine uvputvrd(tno,varname,data,n)
      subroutine uvputvrc(tno,varname,data,n)

      subroutine uvtrack(tno,varname,switches)
      integer function uvscan(tno,varname)
      logical function uvupdate(tno)
      subroutine uvmark(tno,onoff)
      subroutine uvcopyvr(tin,tout)
      subroutine uvnext(tno)

      subroutine uvset(tno,object,type,n,p1,p2,p3)
      subroutine uvselect(tno,object,p1,p2,flag)
      subroutine uvinfo(tno,object,data)
\endtt}

\beginsub{OPEN, CLOSE and REWIND}

The routine {\tt uvopen} opens a uv dataset and readies it for access.
Here {\tt dataname} is a string giving the dataset's name. {\tt status} can
be either {\tt 'old'} or {\tt 'new'}, depending on whether an old dataset is
being read, or a new dataset is being created. {\tt Tno} is an integer
handle passed back by the open routine (and is used for all future access
to the dataset). The routine {\tt uvclose} closes the dataset.
The routine {\tt uvrewind} positions a uv file at its beginning, and
allows it to be read again.

\beginsub{Reading and Writing Visibilities}

The routines {\tt uvread} and {\tt uvwrite} are routines used to read and
write the correlation data (and the associated flagging information).
Here {\tt preamble} is an array of four double precision elements, {\tt data}
is an array of {\tt n} 
complex elements, whereas {\tt flags} is an array of {\tt n} logical values.
{\tt preamble}, {\tt data} and {\tt flags} are output from {\tt uvread},
whereas they are input to {\tt uvwrite}. The four elements of
{\tt preamble} are the u coordinate, v coordinate, time (Julian date) and
baseline number. The baseline number, $Bl$, is calculated  as:
$$\eqalign{Bl &= 256 A_1 + A_2\cr}$$
where $A_1$ and $A_2$ are the numbers of the first and second antennae
respectively (antenna numbers vary from 1 to $N_{antenna}$). The array
{\tt data} is used to store the complex correlation data, whereas the logical
values of the array
{\tt flags} indicate whether the corresponding correlation is
deemed good or bad (true or false, respectively). For {\tt uvread}, {\tt n}
limits the number of correlations that can be read; the actual number of
correlations read is passed back as {\tt nread}. {\tt uvread} can perform
a number of additional processing steps -- see the description of
{\tt uvset} elsewhere in this manual.

The flags of a data file can be modified using the {\tt uvflgwr} subroutine.
When called, the flags associated with the previous call to
{\tt uvread} and {\tt uvwrite} are changed to those values given in the
{\tt flags} array. Using {\tt uvflgwr}, when reading a visibility file,
is the method used to develop flagging tasks.  Currently {\tt uvflgwr}
has the limitation that the linetype is either `channel' or `wide', and that
the `start' and `width' linetype parameters are 1.  Also, {\tt uvflgwr}
aborts if no flagging file exists.

\beginsub{Reading and Writing Continuum Visibilities}

MIRIAD uv datasets can contain both spectral and continuum visibility
data simultaneously. When both are present, the user/programmer will
normally select which of these data to read, using the {\tt uvset} routine.
However, this allows only one ``linetype'' to be
read and written at a time. The routines {\tt uvwread} and {\tt uvwwrite}
allow the programmer to read and write the continuum data independently
of the ``linetype''. These routines completely bypass linetype processing.
They should be used only when both a particular linetype, and the continuum
data are required. {\tt uvwread} should be called after the call to
{\tt uvread}, whereas {\tt uvwwrite} should be called before the call to
{\tt uvwrite}.

The routine {\tt uvwflgwr} is the ``wide'' equivalent of {\tt uvflgwr}.
That is, by calling {\tt uvwflgwr}, you can overwrite the flags associated
with the previous call to {\tt uvwread} and {\tt uvwwrite}.  Note that
{\tt uvwflgwr} aborts if no flagging file exists.

\beginsub{Direct Access to uv Variables}

The main routines to access the variables are the {\tt uvgetvr} and
{\tt uvputvr} routines. These read or write (respectively) an array of
variables of given name ({\tt varname}). The data read or written are
in the array {\tt data} (which can be a character string or an integer, real
or double precision
array, depending on the routine called). Exactly {\tt n} values are
read or written. Except for character strings, it is a fatal error,
when reading, if {\tt n} does not
agree with the actual number of values for the variable. For a character
string, the string is blank padded on a read (no {\tt n} parameter is
needed).

When reading an old dataset, the {\tt uvgetvr} routines should not be called
before the first call to {\tt uvread}.
When creating a dataset, the initial values for all variables should be
written, by calls to the {\tt uvputvr} routines, before the first call
to the {\tt uvwrite} routine.

Often it occurs that we are interested in a variable which has a single
value, but we are not sure if the variable is present in the dataset.
It would be possible to handle this with {\tt uvprobvr} and {\tt uvgetvr}.
If the variable is not present, then we would want to use a default value.
The {\tt uvrdvr} routines package these three steps. It returns the
value of the variable, {\tt data}. If the variable is missing from the
data stream, the default value, {\tt default}, is returned. One disadvantage
is that the {\tt uvrdvr} routines only return a single value (the first
value in a multi-element variable).

The routine {\tt uvread} functions by scanning through the variable streams
and returns with its results when the ``correlation data'' (``corr'' or
``wcorr'') is encountered. If you are not interested in reading the
correlation data (i.e. if you do not intend calling {\tt uvread}),
then the {\tt uvscan} routine can be used to scan through the variable stream
until another variable is encountered. Actually {\tt uvscan} may well read
somewhat past the desired variable, until it has read
all variables which changed simultaneously with the desired one.
{\tt uvscan} returns 0 if the variable was successfully found, -1 if an end
of file was encountered, or 1 if the variable was not found. Note that it
may not make a great deal of sense to intermix calls to {\tt uvscan}
and {\tt uvread}.

\beginsub{Variable Override}

Occassionally it is useful for the user to override the value of a uv variable.
This is especially useful if the value of the variable is both wrong and
important! When uvio opens an old file, for each variable name that is
present in the uv file, it checks if there is a corresponding item with
the same name. If there is, then the value of the item is used to override
the value of the variable. Unfortunately the item must consist of a single
number, and this single number will be copied into each value of the variable
(if the variable consists of several elements).

\beginsub{UVNEXT}

The uv i/o routines need to know what variables change simultaneously.
Routines {\tt uvread} and {\tt uvscan} scan until they have read
all variables that have changes simultaneously. Conversely routine
{\tt uvwrite} assumes that all variables that change simultaneously with
the variables that it writes, have already been written with the routine
{\tt uvputvr}. Hence, if the programmer is using {\tt uvscan}, {\tt uvread} or
{\tt uvwrite}, then simultaneity is not a concern, as long as variables are
read after {\tt uvscan/uvread}, and written before {\tt uvwrite}. The routine
{\tt uvnext} provides better control, where this is needed. For an input
file, a call to {\tt uvnext} causes the next set of variables (which
change simultaneously) to be read. For an output file, {\tt uvnext} causes
a marker to be written into the data, to indicate that variables written
after the call did not change simultaneously with variables written before
the call.

\beginsub{Determining uv Variables and Their Characteristics}

Routine {\tt uvprobvr} checks for the existence of a variable, and
returns information about it. The string {\tt varname}, the variable name,
is input. The single character {\tt type} is output, being either
`a', `r', `d', `c', `i', `j', or ` ' (a blank), which indicates (respectively)
that the variable is of type string (ascii), real, double precision, complex,
integer, short integer or (in the case of the blank) that the variable is not
present in the data-set.  The output integer {\tt length} gives the number of
elements in the variable, and the output logical {\tt update} indicates
whether the variable has been updated `recently'.  Both {\tt length} and
{\tt update} have no meaning if the variable is not present in the dataset.
Before a variable is first read, {\tt length} will be zero, and {\tt update}
will be {\tt .FALSE.}.

There is no special routine to return a complete list of the variables
present in a uv dataset, however this information is present in the 
item ``{\tt vartable}''. This is a text file with each line
consisting of two fields separated by a blank. The first is the ``type''
(either a, r, d, c, i, or j) of the variable, the second is the variables
name.  The following section of FORTRAN lists the variables present in a
uv dataset.
{\ninepoint\begintt

       character var*12,name*(?)
       integer iostat,tno,item

       call uvopen(tno,name,'old')
       call haccess(tno,item,'vartable','read',iostat)
       call hreada(item,var,iostat)
       dowhile(iostat.eq.0)
         call output(var(3:10))
         call hreada(item,var,iostat)
       enddo
       call hdaccess(item,iostat)
       call uvclose(tno)

\endtt}

\beginsub{Keeping Track of uv Variables}

With many variables streaming past, there is a need to keep track of some
particular variables. It would be rather inefficient and laborious to need
to continually call {\tt uvprobvr}, to check on particular variables. The
{\tt uvtrack} routine is used to instruct the uv routines to keep track
of when a certain variable changes its value, and to perform special
processing on these variables at a later stage. Typically {\tt uvtrack}
would be called soon after {\tt uvopen}, marking all the variables of
particular interest. The special processing that the uv routines perform
is dictated by the {\tt switches} argument. This is a string,
consisting of several characters, each character representing a particular
processing step to be taken. Currently there are two switches --
{\tt u} and {\tt c}. The {\tt u} switch is used by {\tt uvupdate}, whereas
the {\tt c} switch is used by {\tt uvcopyvr}.

The routine {\tt uvupdate} returns a {\tt .TRUE.} value if one of the
variables, marked with the {\tt u} switch, has been updated ``recently''.

The routine {\tt uvcopyvr} copies variables marked with the {\tt c}
switch, from the input dataset (given by {\tt tin}) to the output
dataset (given by {\tt tout}) if they have changed ``recently''.  You need
only mark the variables in the input dataset.

\beginsub{When Do UV Variables Change?}

A uv variable can change its value in any part of the uv variable stream.
So it can change its value after each call to {\tt uvread}, {\tt uvnext},
 or {\tt uvscan}.  The uv routines which need to know if a uv variable has
changed ({\tt uvprobvr}, {\tt uvcopyvr} and {\tt uvupdated}) normally (i.e.,
 by default) work on whether the particular variable(s) of interest has
changed since the last ``mark'' in the uv stream.  By default, any routine
which causes more of the uv stream to be read ({\tt uvread}, {\tt uvscan} and
{\tt uvnext}) move this marker to the current point in the uv stream, before
reading more.  The {\tt uvmark} routine provides greater control at marking
the position in the stream. Calling
{\ninepoint\begintt
      call uvmark(tno,.TRUE.)
\endtt}
sets the marker at the current position in the uv file, and disables
{\tt uvread}, etc, from resetting the marker. Calling
{\ninepoint\begintt
      call uvmark(tno,.FALSE.)
\endtt}
also sets the marker at the current position, and enables {\tt uvread}, etc,
to reset the marker on each call.

\beginsub{Massaging Steps Performed by UVREAD -- UVSET}

Below are the arguments to {\tt UVSET} for {\tt STATUS=OLD}.

\vskip 0.2in
{\raggedright
\def\bbox#1{$\underline{\smash{\hbox{#1}}}$}
\ninepoint
\tabskip=0em
\halign {#\tabskip=2em&#\tabskip=1em&#\tabskip=.25em&#\tabskip=.25em&#\tabskip=.25em&#\hfil\cr

\bbox{\bf Object}&\bbox{\bf Type}&\bbox{\bf N}&\bbox{\bf P1}&\bbox{\bf P2}&\bbox{\bf P3}\cr
\cr
data      & channel	& nchan, & start, & width, & step\cr
          & wide	& nchan, & start, & width, & step\cr
          & velocity	& nchan, & start, & width, & step\cr
reference & channel	&  ---,  & start, & width, & --- \cr
          & wide	&  ---,  & start, & width, & --- \cr
          & velocity	&  ---,  & start, & width, & --- \cr
coord	  & wavelength	&  ---,  & ---,   & ---,   & --- \cr
          & nanosec	&  ---,  & ---,   & ---,   & ---\cr
planet	  & ---		&  ---,  & plmaj, & plmin, & plangle\cr
selection & amplitude	&  n,    & ---,   & ---,   & --- \cr
}
}

As mentioned above, the {\tt uvread} routine can perform, at the programmers
request, extra processing steps on the visibility data. These steps
consist of averaging and resampling frequency channels, uv coordinate
conversion and some corrections for planet observations. The steps are
requested by calls to {\tt uvset}. In the call to {\tt uvset}, the argument
{\tt object} (a string) gives the general processing step that is being
requested. The {\tt type} argument (another string) gives more specific
details, and the arguments {\tt n} (integer) and {\tt p1}, {\tt p2} and
{\tt p3} (reals) give any numerical values needed.

Note that the set-up given by {\tt uvset} only becomes correctly
activated during the next call to {\tt uvread}. Before this next call,
the setup is in a somewhat nebulous state. So you should not expect
various other routines associated with {\tt uvread} to work
as expected until after the next call to {\tt uvread}. Associated
routines include {\tt uvflgwr} and {\tt uvinfo}.

The table summarizes the possible values of the arguments to {\tt uvset}.
Here the column titled ``Object'' and ``Type'' are the possible string
values that {\tt object} and {\tt type} can take on. The sixth column
gives the meaning for the parameters {\tt n,p1,p2,p3}.
Dashes in the sixth column indicate that the argument's value is ignored in
this particular call. While several processing steps can be performed
simultaneously (several calls to {\tt uvset} will be needed to specify them
all), others are mutually inconsistent. When mutually inconsistent
steps are requested, the last requested step is honored. Each processing
step requires further explanation. 
\item{$\bullet$} {\tt object='data'} This gives operations on the spectral
data. Type {\tt 'channel'} selects the channels to be returned, and possible
averaging together of the channel data. If the original channels are
numbers from 1 to $N$, then by using {\tt type='channel'}, {\tt uvread} will
return $nchan$ massaged channels, where channel $i$ of the massaged channels
is formed by averaging $width$ channels of the original data, starting at
channel $(i-1)\cdot step + start$.  If {\tt uvset} is called with $nchan$
being zero, all channels are selected (note that this only makes sense if
$start$, $step$ and $width$ are all 1).  {\tt type='wide'} is similar, but
uses the continuum data rather than the spectral data.  {\tt type='velocity'}
is also similar, returning a weighted sum of the spectral data. However, in
this case $start$, $width$ and $step$ are given in units of km/s (rather
than channels). This is particularly useful if the spectrometer setup is not
constant throughout the data or there is no Doppler tracking, and so the
velocity of a given channel changes.  Note that {\tt 'channel'},
{\tt 'wide'} and {\tt 'velocity'} are mutually exclusive. The default is
{\tt 'channel'} (or {\tt wide} if there is no spectral data in the file),
with $start$, $step$, and $width$ set to 1.

If there are fewer than $nchan$ channels in the dataset, then dummy channels,
which are flagged as bad, are added. If $nchan$ is specified as 0, then
{\tt uvread} will return as many channels as possible.

\item{$\bullet$} {\tt object='reference'} The ``reference line'' is a
spectral channel, or an average of spectral channels, which the main data
is divided by. Typically, the reference line would be a strong point source
(e.g. a maser). The resultant data is essentially normalized and shifted,
but it also has atmospheric-based and instrument-based calibration problems
removed. The extra parameters needed to describe the reference line are the
same as for {\tt object='data'}, except that the number of channels, and the
increment are ignored (there can be only one reference line). The default is
not to have a reference line.  \item{$\bullet$} {\tt object='coord'} This
sets the units of the $u$ and $v$ coordinates returned in the preamble. Using
{\tt 'wavelength'} or {\tt 'nanosec'} sets the units of the returned $u$
and $v$. For {\tt 'wavelength'}, the sky frequency used is that of the first
channel returned. The default value is {\tt 'nanosec'}.
\item{$\bullet$} {\tt object='planet'} This causes the $u$ and $v$
coordinates to be scaled and rotated, and the correlation values to be scaled,
to adjust for changes when observing planets.  The parameters {\tt plmaj}
and {\tt plmin} give the reference size (arcseconds) and parameter
{\tt plangle} gives and position angle (degrees) of the planet. If the
reference size is 0, then the size of the first selected data record is used.
\item{$\bullet$} {\tt object='selection'} This gives extra control over the
uv selection process.  Currently there is only one possible type,
{\tt 'amplitude'}, which enables or disables the amplitude selection process.
If the argument {\tt n} is positive, then amplitude selection is applied
(i.e., the normal action), otherwise amplitude selection is not applied.

\beginsub{Setting Up UVWRITE -- UVSET}

Below are the arguments to {\tt UVSET} for {\tt STATUS=NEW}.
\vskip 0.2in
{\raggedright
\def\bbox#1{$\underline{\smash{\hbox{#1}}}$}
\ninepoint
\tabskip=0em
\halign {#\tabskip=2em&#\tabskip=1em&#\tabskip=.25em&#\tabskip=.25em&#\tabskip=.25em&#\hfil\cr

\bbox{\bf Object}&\bbox{\bf Type}&\bbox{\bf N}&\bbox{\bf P1}&\bbox{\bf P2}&\bbox{\bf P3}\cr
\cr
corr & c        & ---, & ---, & ---, & ---\cr
     & j        & ---, & ---, & ---, & ---\cr
data & channel  & ---, & ---, & ---, & ---\cr
     & wide     & ---, & ---, & ---, & ---\cr
}
}

\item{$\bullet$} {\tt object='corr'}
The uv routines allow the correlation data to be stored on disk, either
as floating point numbers, or as 16 bit integers with an associated scale
factor. The 16 bit format roughly halves the disk space required, but slows
the read and write operation, and can cause precision problems. On the
first call to {\tt uvwrite}, the uv routines decide on the format to
use, using a simple rule. The {\tt uvset} call
can be used to override this rule. To be of use, it must be called before
the first call to {\tt uvwrite}. The {\tt type} argument is a single
character, being {\tt `r'} or {\tt `j'}, which instructs floating
point or scaled integers, respectively, to be used.
\item{$\bullet$} {\tt object='data'} This controls whether {\tt UVWRITE}
writes the data to the {\tt corr} or {\tt wcorr} variable. The default is
to write it to the {\tt corr} item (i.e. it assumes that the data is
spectral, rather than continuum data).

\beginsub{Selection Steps Performed by UVREAD -- UVSELECT}

The arguments to {\tt UVSELECT} are given below:
\vskip 0.2in
{\raggedright
\def\bbox#1{$\underline{\smash{\hbox{#1}}}$}
\ninepoint
\tabskip=0em
\halign {#\tabskip=2em&#\tabskip=2em&#\tabskip=1em&#\hfil\cr

\bbox{\bf Object}&\bbox{\bf Units}&\bbox{\bf P1}&\bbox{\bf P2}\cr
\cr
time         & Julian date & tmin,    & tmax \cr
dra          & radians     & dramin,  & dramax \cr
ddec         & radians     & ddecmin, & ddecmax \cr
ra           & radians     & ramin,   & ramax \cr
dec          & radians     & decmin,  & decmax \cr
uvrange      & wavelengths & uvmin,   & uvmax \cr
uvnrange     & nanoseconds & uvmin,   & uvmax \cr
pointing     & arcseconds  & pntmin,  & pntmax \cr
visibility   &             & vismin,  & vismax \cr
increment    &             & incr,    & -- \cr
on           &             & state,   & -- \cr
polarization & FITS code   & polval,  & -- \cr
amplitude    &             & ampmin,  & ampmax \cr
window       &             & win,     & -- \cr
antennae     &             & ant1,    & ant2 \cr
or           &             & --,      & -- \cr
and          &             & --,      & -- \cr
clear        &             & --,      & -- \cr
}
}

Another function performed by {\tt uvread} is to skip or to flag data that
is not required. The routine {\tt uvselect} is used to instruct {\tt uvread}
on which data are to be selected and rejected. Normally the programmer will
not call {\tt uvselect} directly, but will use the {\tt SelInput} and
{\tt SelApply} routines. The Users Manual gives a description
about the way the user normally interacts with these routines.

Generally, {\tt uvselect} will be called many times, each call giving
different selection or discard criteria.  The routines {\tt SelInput}
and {\tt SelApply} merely sequentially parse and pass the user-given
criteria to {\tt uvselect}. Hence the `grammar' of the sequence of calls to
{\tt uvselect} (i.e. use of ``or'', or multiple occurrences of
criteria based on the same parameter) is the same as the `grammar' of the
user-specified task parameter. The `grammar' will not be repeated here.

The argument {\tt object} is a string giving the parameter on which a
select/discard criteria is based. The
arguments {\tt p1} and {\tt p2} (both double precision) give
added numerical parameters. Generally {\tt p1,p2} give a range of
parameter values to select or discard. Note that the units of the values are
consistent with the units
of the underlying uv variables. These are not
necessarily the most convenient units for the user, and so the user
interface (given by {\tt selinput} and {\tt selapply}) performs some
conversion between user-units and program-units.

There are a few additional objects, when compared with the {\tt SelInput}
and {\tt SelApply}
routines. These include the {\tt ra} and {\tt dec} objects, which give the
pointing centre RA and DEC (after {\tt dra} and {\tt ddec} have been taken
into account). Another is the {\tt and} operator. {\tt uvselect} treates
{\tt and} and {\tt or} as having identical precedence, and handles these
operators in the order in which they are given. Beware of this lack of
precedence.

The {\tt clear} object causes the selection criteria to be reset to its
default of selecting everything.

The argument {\tt flag} determines whether data which matches the
associated criteria is to be selected (in which case {\tt flag=.TRUE.})
or discarded (in which case {\tt flag=.FALSE.}).

For example, to select data with Julian days 2444239.5 to 2444240.5 (i.e.
data for 1 January, 1980), use:
{\ninepoint\begintt
     call uvselect(tno,'time',2444239.5d0,2444240.5d0,.TRUE.)
\endtt}
To select all data, except for 1 January, 1980, use:
{\ninepoint\begintt
     call uvselect(tno,'time',2444239.5d0,2444240.5d0,.FALSE.)
\endtt}
Note: In the {\tt 'antennae'} criterion, an antennae number of 0 is treated
as a ``match-all'' number.

\beginsub{Getting Information After UVREAD}

Below are the arguments to {\tt uvinfo}.
\vskip 0.2in
{\raggedright
\def\bbox#1{$\underline{\smash{\hbox{#1}}}$}
\ninepoint
\tabskip=0em
\halign {#\tabskip=2em&#\tabskip=2em&#\hfil\cr

\bbox{Object}&\bbox{Units}&\bbox{No. Values Returned}\cr
\cr
restfreq  & GHz          & nread \cr
velocity  & km/s         & nread \cr
bandwidth & GHz          & nread \cr
frequency & GHz          & nread \cr
sfreq     & GHz          & nread \cr
visno     &              & 1\cr
amprange  & (flux units) & 3\cr
}
}

The routine {\tt uvinfo} returns information about the data returned by
the last call to {\tt uvread}. The argument {\tt object} is a character
string indicating the information that is desired. The argument {\tt data}
is a double precision array, containing the returned information.
Possible values for {\tt object} are:

\item{$\bullet$} {\tt 'restfreq'}:  {\tt data} contains the rest frequency
(GHz) for each channel returned by {\tt uvread}.
\item{$\bullet$} {\tt 'velocity'}:  {\tt data} contains the velocity (km/s)
for each channel returned by {\tt uvread}.
\item{$\bullet$} {\tt 'frequency'}:  {\tt data} contains the frequency
(GHz) of the channel returned by {\tt uvread}, after removing the
doppler contribution.
\item{$\bullet$} {\tt 'bandwidth'}:  {\tt data} contains the bandwidth
(GHz) of each channel returned by {\tt uvread}.
\item{$\bullet$} {\tt 'sfreq'}:  {\tt data} contains the sky frequency (GHz)
of each channel returned by {\tt uvread}.
\item{$\bullet$} {\tt 'visno'}:  {\tt data} contains a single number, which
is the visibility number (running from 1 upwards) of the last channel read.
\item{$\bullet$} {\tt 'amprange'}:  {\tt data} contains three values. The
first value indicate the type of amplitude selection that was requested for
this record, and the second and third values give a flux range.
Possible value of data(1) are -1 (data outside the range [data(2),data(3)]
were rejected), 0 (no amplitude selection was active) or +1 (data inside the
range [data(2),data(3)] were rejected).

For example, consider the following code fragment:
{\ninepoint\begintt
      integer maxchan
      parameter(maxchan=512)
      integer tno,nread
      complex data(maxchan)
      logical flags(maxchan)
      double precision preamble(4),velocity(maxchan)
           .
           .
           .
      call uvread(tno,preamble,data,flags,maxchan,nread)
      call uvinfo(tno,'velocity',velocity)
\endtt}
After the call to {\tt uvinfo}, the array velocity will contain the velocity
of each channel read by {\tt uvread}.

\beginsection{uv Selection -- SELINPUT and SELAPPLY}

{\ninepoint\begintt
     subroutine SelInput(key,sels,maxsels)
     logical function SelProbe(sels,object,value)
     subroutine SelApply(tno,sels,flag)
\endtt}
These routines are the usual programmer interface to the uv selection
routines. They perform the parsing and checking of the user input, and
the calling of the {\tt uvselect} routine to actually implement the
selection process. For more information see uv selection in the User's
Guide, and the {\tt uvselect} routine in this Programmer's Guide.

{\tt selinput} calls the {\tt keya} routine to get the user-specified
selection criteria. These criteria is then broken into an intermediate form.
The argument {\tt key} is the keyword to be used. Generally it should be
{\tt 'select'}. The real array {\tt sels} (of size {\tt maxsels} elements)
is used to hold the intermediate form of the selection.

{\tt selapply} takes selection criteria, in its intermediate form, and
calls the {\tt uvselect} routine to apply it. The argument {\tt flag}
determines whether criteria are actually to be used for selection
({\tt flag=.TRUE.}), or rejection ({\tt flag=.FALSE.}).

{\tt selprobe} returns information about whether uv data with a particular
parameter value may have been selected.
It does not guarantee that such data might exist
in any particular data file. It also has the limitation that
information is not present to convert ``uvrange'' and ``uvnrange''
calls into each other. These should be treated with caution. The {\tt sels}
array is the intermediate form returned by {\tt selinput}, and {\tt value}
is a double precision value, giving the parameter value that is of interest.
The {\tt object} argument determines the meaning (and the units) of this
value. Possible values are as below:

{\raggedright
\def\bbox#1{$\underline{\smash{\hbox{#1}}}$}
\ninepoint
\tabskip=0em
\halign {#\tabskip=2em&#\hfil\cr

\bbox{Object}&\bbox{Units of Value}\cr
\cr
'time'       & Julian day.\cr
'antennae'   & Baseline number = 256*ant1 + ant2.\cr
             & One of ant1 or ant2 can be zero.\cr
'uvrange'    & Wavelengths.\cr
'uvnrange'   & Nanoseconds.\cr
'visibility' & Visibility number (1 relative).\cr
'dra'        & Radians.\cr
'ddec'       & Radians.\cr
'pointing'   & Arcseconds.\cr
'amplitude'  & Same as correlation data.\cr
'window'     & Window number.\cr
}
}

Note that this does not support all objects to {\tt uvselect}.
The name must be given in full (no abbreviations and case is significant).
