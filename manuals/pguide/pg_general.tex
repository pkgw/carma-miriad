%------------------------------------------------------------------------
% Chapter 10 - General Discussions
%------------------------------------------------------------------------
%
%  History:
%
%   rjs  28mar90  First version for System Guide
%   mjs  22mar91  Adapted into Programmer's Guide
%------------------------------------------------------------------------
\beginsection{Numeric Routines}

The following numeric routines should be used in preference to others that do
the same function, as these are reasonably optimized.  Generally, these are
tailored to the target machine:  on the Cray these are based on Cray SCILIB
routines; on the Convex, they are partially based on VECLIB routines.

\beginsub{FFT Routines}

There are three FFT routines, namely:
{\ninepoint\begintt
      subroutine fftrc(in,out,sign,n)
      subroutine fftcr(in,out,sign,n)
      subroutine fftcc(in,out,sign,n)
\endtt}
These perform one-dimensional FFTs. In all cases, {\tt sign} is the sign
of the exponent in the transform (i.e. a {\tt sign} of -1 is conventionally
viewed as a forward transform), and {\tt n} is a power of 2 giving the
length of the (full) sequence. ${1\over N}$ scaling is never performed
(it is up to you to scale at the best time). {\it In} is the input
array, and {\it Out} is the output array. These routines
evaluate:
$$\eqalign{Out(l) &= \sum_{k=1}^N In(k)\exp(\pm j {2\pi\over N} (k-1)(l-1))\cr}$$
where $k$ and $l$ vary from $1$ to $N$.
{\tt fftrc} transforms a real
sequence (i.e. {\tt in} is a real array) and outputs only the first $N/2 + 1$
complex values. No information is lost because of the conjugate
symmetry of FFTs of real sequences. Conversely {\tt fftcr} takes a
complex sequence of length $N/2+1$ and produces a real sequence of length $N$.
Finally, {\tt fftcc} performs a complex to complex transform, with
both input and output being of length $N$.

\beginsub {Min and Max Value Routines}

There are two routines to find the index of the minimum and maximum values
of an array, namely:
{\ninepoint\begintt
      integer function ismin(n,data,step)
      integer function ismax(n,data,step)
\endtt}

\beginsub{WHEN and ISRCH Routines}

{\ninepoint\begintt
      subroutine whenxxx(n,array,step,target,index,nindex)
      integer function isrchxxx(n,array,step,target)
\endtt}
The {\tt when} routines return the indices of all locations in an array that
have a ``true relational value'' to the target. See page 4-64 of the
Cray ``Library Reference Manual'' for more information. The {\tt isrch}
routines return the first location in an array that has a ``true
relational value'' to the target. See page 4-59 of the Cray ``Library
Reference Manual''. For both routines, ``{\tt xxx}'' can be one of
{\tt ieq}, {\tt ine}, {\tt ilt}, {\tt ile}, {\tt igt}, {\tt ige}, {\tt feq},
{\tt fne}, {\tt flt}, {\tt fle}, {\tt fgt} or {\tt fge}.
These give the type of {\tt array} (Integer or Floating), and the relation
(equal, not equal, less than, etc).

Because these are well optimized routines, these routines should be used in
preference to straightforward FORTRAN code that performs an equivalent
function.

\beginsub{BLAS and LINPACK Routines}

The BLAS routines are ``Basic Linear Algebra Subprograms'', while the
LINPACK routines are a suite of linear equation solving routines. It is
common to find these routines, in an optimized form, on vector computers.
Additionally they are public domain routines in standard FORTRAN, making them
easy to port. The MIRIAD library (or a system library) has these routines in
their single precision real and complex forms.
There are a number of good references on BLAS and LINPACK routines
available (e.g. 4-1 and 4-38 in the Cray ``Library Reference Manual''),
or the Convex ``VECLIB Users Manual'').
