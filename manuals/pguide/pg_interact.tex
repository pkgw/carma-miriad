\beginsection Task Parameters - key routines

A MIRIAD task is invoked by entering
{\ninepoint\begintt
taskname keyword1=value1 keyword2=value2 keyword3=value3 ...
\endtt}
The {\tt key} routines are used to pass parameters to MIRIAD programs.
They parse the command-line parameters, translating the values to their
appropriate internal representation.  Typically, the {\tt key} routines will
be called in the first few lines of the task, and all checking for the
validity of parameters should be carried out at that time.

{\ninepoint\begintt
keyini ..... Initialize the key routines |newline
   |newline 
keya ....... Keyword value is a character string.
keyd ....... Keyword value is a double-precision number.
keyf ....... Keyword value is a character string (see below).
keyi ....... Keyword value is an integer.
keyl ....... Keyword value is a boolean (logical).
keyr ....... Keyword value is a real number. |newline
 |newline
mkeya ...... Retrieve multiple character values.
mkeyf ...... Retrieve multiple character values (see below).
mkeyi ...... Retrieve multiple integer values.
mkeyr ...... Retrieve multiple real values. |newline 
   |newline
keyfin ..... Indicates that key routine processing is done |newline 
   |newline 
keyprsnt ... Determines if a given keyword was actually set by the user
\endtt}
\beginsub{key Calling Sequence}

Calls to the {\tt key} routines are sandwiched between calls to {\tt keyini}
and {\tt keyfin}.  Their calling sequence is:

{\ninepoint\begintt
subroutine keyini |newline
   |newline
subroutine keya(keyword,value,default)
subroutine keyd(keyword,value,default)
subroutine keyf(keyword,value,default)
subroutine keyi(keyword,value,default)
subroutine keyl(keyword,value,default)
subroutine keyr(keyword,value,default) |newline
   |newline
subroutine mkeya(keyword,value,nmax,n)
subroutine mkeyf(keyword,value,nmax,n)
subroutine mkeyi(keyword,value,nmax,n)
subroutine mkeyr(keyword,value,nmax,n) |newline
   |newline
subroutine keyfin |newline
   |newline
logical function keyprsnt(keyword)
\endtt}
{\tt keyword} is a character string of appropriate length, {\tt value} is
the internal variable to be assigned, and {\tt default} is the default
value to return if the user did not set a particular keyword (if, for
example, a ``log'' filename is an option, the default might be to write
the ``log'' information to the terminal).

Routine {\tt keyf} differs from {\tt keya} in that values given for
{\tt keyf} are treated as MIRIAD datasets, with wildcard expansion
performed.  Routine {\tt mkeyf} differs from {\tt mkeya} in a similar
manner.

The routines beginning with {\tt m} (for multiple) offer a means of
obtaining {\tt value} in a slightly different way, with {\tt nmax} being
the maximum number of values to return ({\tt value} is declared as
{\tt value(nmax)}), with {\tt n} being the actual number of values returned
(if no value is entered, a zero value is returned).

\beginsub{Capabilities and Limitations}

\item{$\bullet$} The keyword MUST be in lower case.
\item{$\bullet$} A default value must be given by the programmer, even
it is not later used.
\item{$\bullet$} The {\tt key} routines may be called in any order, except
that {\tt keyini} must be called first and {\tt keyfin} must be called
last.
\item{$\bullet$} Keyword values are ``exhausted'' (that is, erased) once
they have been passed to the {\tt key} routines.
\item{$\bullet$} The entry of a range of keyword values is supported
when the entries are a comma-separated list.  For example, a keyword
(call it TRC) can specify the range of values from 45 through 50 as
follows (only 45 and 50 are inputs, but the programmer can use these values
as endpoints, with increments being programmer-set):
{\ninepoint\begintt
taskname TRC=45,50
\endtt}
The internal code of the program might be:
{\ninepoint\begintt
.
.
.
integer n1,n2
.
.
.
call keyi('trc',n1,-1)
call keyi('trc',n2,-1)
.
.
.
\endtt}
After these calls, the value of {\tt n1} is 45 and the value of {\tt n2}
is 50.  Had the user entered only one value for TRC (for example, TRC=45),
the programmer could have discovered this by calling {\tt keyprsnt} prior
to the second call to {\tt keyi}:  if only one value had been given, 
{\tt keyprsnt} would have returned {\tt .FALSE.} after the first call to
{\tt keyi} because the keyword had been ``exhausted''.  An alternate way to
test whether a keyword has been exhausted is to pass a default value which
is clearly illegal (such as a blank string for a file name).

\beginsection Error Handling

Many MIRIAD routines perform error checking internally, and bomb out
if an error is detected. Other MIRIAD routines pass back a status
value (generally the last subroutine argument). A status value of zero
indicates success, -1 indicates end-of-file, and a positive value indicates
some other error (what the positive values indicate is system dependent).
Two routines can be called to indicate an error:
{\ninepoint\begintt
subroutine bug(severity,text)
subroutine bugno(severity,number)
\endtt}
Here {\tt severity} is a single character, being either {\tt 'w', 'i','e'}
or {\tt 'f'}, meaning warning, informational, error and fatal, respectively.
When {\tt bug} or {\tt bugno} is called with a fatal error, it will not
return. Rather, it will cause the task to exit. For routine {\tt bug},
{\tt text} is a character string describing the error. For routine
{\tt bugno}, {\tt number} is a status value, returned by a MIRIAD
 routine.

\beginsection Text I\/O

Though standard Fortran-77 routines would appear to be adequate for text i/o,
there are invariably minor differences between systems, mainly related to
carriage control. Additionally, placing them in a module of routines forces
the programmer to follow the `handle' (logical unit number) convention.
Following is a list of text IO routines used in MIRIAD.
{\ninepoint\begintt
output ..... Print a character string on the terminal. |newline
   |newline 
txtopen .... Open a text file by name, returning a logical unit number.
txtopena ... Open an old text file in append mode.
txtread .... Read a character string from an open file.
txtwrite ... Write a character string to an open file.
txtclose ... Close the text file. |newline
   |newline 
logopen .... Open a log text file.
logwrit .... Write to an open log file, or the terminal.
logwrite ... Write to an open log file, or the terminal.
logclose ... Close the log file.
\endtt}
The calling sequence for the text-handling routines are:
{\ninepoint\begintt
subroutine output  (text) |newline
   |newline 
subroutine txtopen (handle,name,status,iostat)
subroutine txtread (handle,text,length,iostat)
subroutine txtwrite(handle,text,length,iostat)
subroutine txtclose(handle) |newline
   |newline 
subroutine logopen (name,flags)
subroutine logwrit (text)
subroutine logwrite(text,more)
subroutine logclose
\endtt}
{\tt txtopen} opens a text file with name {\tt name}.  {\tt status} can be
either {\tt 'old'} or {\tt 'new'}. When opening a new file, any old files
which exist with the same name may be deleted.  {\tt txtread} and
{\tt txtwrite} read and write a character string, {\tt text}, of
{\tt length} characters. {\tt length} is passed back from {\tt txtread},
whereas it is passed into {\tt txtwrite}. It may be zero in either case.
All these routines return an i/o status variable, {\tt iostat}, indicating
whether a problem occurred (see the previous section for an explanation of
the {\tt iostat} parameter.

{\tt logopen} opens a log file with name {\tt name}, passing options
{\tt flags}.  {\tt logwrit} and {\tt logwrite} write to the log file, and
{\tt logclose} closes the log file.

\beginsection A Simple Example

The code below serves to illustrate some of the subroutines discussed in
this chapter.
{\ninepoint\begintt
        program copyhdx
        implicit none
c
c= COPYHDx - Copy items from one data-set to another.
c& rjs
c: utility
c+
c       COPYHDx is a Miriad task which copies items from one Miriad data-set
c       to another. There is no interpretation of the items at all.
c
c       This code is altered to serve as an illustration in the manuals.
c@ in
c       Name of the input data set. No default.
c@ out
c       Name of the output data set. This must already exist. No default.
c@ items
c       A list of items to be copied across. At least one value must be
c       given.
c--
\endtt}
{\ninepoint\begintt
        character version*(*)
        integer maxitems
        parameter(version='version 1.0 29-Jan-91')
        parameter(maxitems=32)
c
        character items(maxitems)*16,in*64,out*64
        integer tIn,tOut,nitems,iostat,i
        logical hdprsnt
\endtt}
{\ninepoint\begintt
        call output('Copyhdx: '//version)
        call keyini
        call keya('in',in,' ')
        call keya('out',out,' ')
        call mkeya('items',items,maxitems,nitems)
        call keyfin
\endtt}
{\ninepoint\begintt
        if(in.eq.' ')call bug('f','Input must be given')
        if(out.eq.' ') call bug('f','Output must be given')
        if(nitems.eq.0)
     *    call bug('f','A list of items to copy must be given')
.
.
etc.
\endtt}
The in-code documentation specifies the keywords that the user will enter
({\tt in}, {\tt out}, and {\tt items}; in-code documentation is discussed
in more detail in a different chapter of this manual).  Program variables
are declared, then {\tt output} confirms to the user which program is
being run (as well as stating the version number and date).  Next, the
{\tt key} routines obtain the task parameters, followed immediately by
a check to determine the validity of the keyword values (calling {\tt bug}
with the {\tt 'f'} (for fatal) option to exit the program if the values
are not acceptable).

A user could run this program either through the {\tt miriad} command
line interface, or by entering

{\ninepoint\begintt
copyhdx in=myinfile out=myoutfile items=string1,text2,chars3
\endtt}
