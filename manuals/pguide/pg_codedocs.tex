%------------------------------------------------------------------------
% Code Documentation Chapter
%------------------------------------------------------------------------

\beginsection On-line and Paper Manuals

This documentation is unambiguously mandatory.  It is used to generate
on-line information, and to generate MIRIAD manuals. Programs and
subroutines that do not contain this documentation will get lost in the
shuffle (users will not be able to determine required input parameters;
other programmers will not be aware of the routine's existence, or how
to use it; the routine will not appear in MIRIAD manuals).  This
documentation should be at the head of the source code.  In FORTRAN
notation (C, UNIX csh, and VMS command notation is analogous), the
documentation ``directives'' are:
{\ninepoint\begintt
c=  [routine name] [one-line description]   (for programs/scripts)
c*  [routine name] [one-line description]   (for subroutines)
c&  programmer ID
c:  comma-separated list of categories
c+
c   start of multi-line description block
c@  keyword                                       (for tasks)
c   multi-line keyword description                (for tasks)
c<  standard keyword                              (for tasks)
c-- 
\endtt}
For FORTRAN, comment lines can begin with either an uppercase or a 
lowercase {\tt c}, C comment lines begin with a {\tt /*}, UNIX csh
script comment lines begin with a {\tt \#}, and VMS command file
comments begin with a {\tt \$!}.  C programmers take note:  once a
{\tt /*} is entered, everything until the next {\tt */} is considered
a comment; it is the programmer's responsibility to determine where to
place the {\tt */}.

On-line documentation is placed in \$MIR/doc/prog for executables
(program docs have a *.doc filename extension, while UNIX script docs
and VMS command docs have a *.cdoc filename extension), and in
\$MIR/doc/subs for subroutines (routines from \$MIR/src/subs have a
*.sdoc filename extension, while ``test'' subroutines coded within a
main program have a *.tdoc filename extension).

\beginsection The doc Program

In-code documentation is extracted from the source code by the {\bf doc}
program.  {\bf doc} recognizes everything between the {\tt c=} ({\tt c*}
for subroutines) and the {\tt c--} lines as program documentation (whether
the text is code or comments), and it creates a file with extension
{\tt .cdoc} to be placed in MIRIAD's on-line doc directory
(\$MIR/doc), with subdirectory ``prog'' used to store program docs, and
subdirectory ``subs'' used to store subroutine docs.

Assumptions about the in-code documentation made by {\bf doc} include:

\item{$\bullet$} The one-line description is less than 66 characters.

\item{$\bullet$} The programmer ID will be found in the file containing
the list of MIRIAD programmers, \$MIR/cat/miriad.pgmrs.

\item{$\bullet$} {\parskip=0.0in The entries in the comma-separated list of
categories will be: \newline
\vskip 0.1in
(for executable programs)
{\ninepoint\begintt
 General             Utility             Data Transfer       Visual Display
 Calibration         uv Analysis         Map Making          Deconvolution
 Plotting            Map Manipulation    Map Combination     Map Analysis
 Profile Analysis    Model Fitting       Tools               Other
\endtt}
(for subroutines)
{\ninepoint\begintt
 Baselines           Calibration         Convolution         Coordinates
 Display             Error-Handling      Files               Fits
 Fourier-Transform   Gridding            Header-I/O          History
 Image-Analysis      Image-I/O           Interpolation       Least-Squares
 Log-File            Low-Level-I/O       Mathematics         Model
 PGPLOT              Plotting            Polynomials         Region-of-Interest
 SCILIB              Sorting             Strings             Terminal-I/O
 Text-I/O            Transpose           TV                  User-Input
 User-Interaction    Utilities           uv-Data             uv-I/O
 Zeeman              Other
\endtt}
(for UNIX scripts and VMS commands)
{\ninepoint\begintt
 System Operation    Programmer Tool     User Utility        Other
\endtt}
Only the above categrories are recognized, and if any other category is
given, {\bf doc} will place the routine in category ``Other''.
}
\item{$\bullet$} For programs and scripts (but not for subroutines),
{\bf doc} interprets the first character on the first non-empty line of the
multi-line documentation block as an ``alignment column''.  When {\bf doc}
generates \LaTeX\ files used in printing the manuals, lines that have a
space in this column are typeset ``verbatim''.  No subsequent line should
start before the ``alignment column''.

\item{$\bullet$} For conversion of in-code docs to \LaTeX\ files, the
documentation block between the {\tt c+} and the {\tt c--} lines may
contain any characters, except that lines to be typeset as ``verbatim''
may not contain the tilde character (one character must be reserved for
internal use as an  ``escape'' character, and the tilde has been
selected).

\item{$\bullet$} For conversion of in-code docs to \LaTeX\ files, double
backslashes in description blocks are converted to single backslashes,
while single backslashes are discarded (single backslashes are interpreted
as ``escape'' characters, not intended to be printed:  a double
backslash in in-code docs will produce a single backslash in the \LaTeX\ 
output).

{\bf doc} is used to:
\item{$\bullet$} Generate on-line documentation files from in-code docs.

\item{$\bullet$} Format on-line docs for readability by MIRIAD users.

\item{$\bullet$} Generate either \LaTeX\ or \TeX\ files from either in-code
docs or on-line docs.

\beginsection MIRIAD Programs

By way of illustration, below is the in-code documentation for
MIRIAD task VARPLOT (\$MIR/src/prog/vis/varplot.for), which
uses the ``directives'' noted previously.  A discussion of the
documentation follows.
{\ninepoint\begintt
c= varplot - Plot uv variables
c& lgm
c: uv analysis, plotting
c+
c       VARPLOT makes X,Y plots selected variables from a uv data set.
c       Only integer, real, and double precision variables maybe plotted.
c       When curser is in the plot window, the following keys are active:
c          X - expand window in X to give one column of plots
c          Y - expand window in Y to give one row of plots
c          Z - expand window in both X and Y to show only one plot
c          N - step to "next" plot in x or y or both depending on expansion
c< vis
c< device
c@ xaxis
c       Name of variable to be plotted along x-axis (default = ut time).
c@ yaxis
c       Name of variable to be plotted along y-axis (no default)
c@ multi
c       Make multiple plots or a single plot? Yes yields multiple plots
c       No yields a single plot with multiple lines as needed
c       (default = yes)
c@ compr
c       Compress number of x or y variables to be plotted by averaging
c       over spectral windows. Currently only SYSTEMP can be averaged.
c--
\endtt}
The task's name is ``varplot'', it's one-line description is ``Plot uv
variables'', the responsible programmer is ``lgm'', and the program is
categorized as both a ``uv analysis'' program and a ``plotting'' program.
It has a general description (the text following the {\tt c+} line), and
it has 6 keywords that the user may input:  ``xaxis'', ``yaxis'', ``multi'',
and ``compr'' are described in the documentation, while ``vis'' and
``device'' descriptions are taken from the \$MIR/cat/keywords file (which
is discussed below).  The on-line documentation file for this task is
\$MIR/doc/prog/varplot.doc.

For keywords where the {\tt c<} ``directive'' is used, if there are
non-standard aspects to the program, or if the programmer wants additional
explanation to be given, then those lines of additional explanation may
be given immediately after the line with the {\tt c<}.

\beginsection The Keywords File

Several keywords are used by many different programs for the same purpose,
and so have standardized descriptions.  These standard descriptions are
contained in the file \$MIR/cat/keywords.  The intent of this file is to
avoid repeating the same lengthy explanation throughout the manuals, and to
ensure consistent descriptions for the keywords across the various tasks
that use them.  At present, standard descriptions are available for the
following keywords:
{\ninepoint\begintt
----------- ----------- ----------- ----------- -----------
in          device      line        out         region
select      server      vis
----------- ----------- ----------- ----------- -----------
\endtt}
The effect of using this standard keywords file is (1) on-line docs will
contain the full description of the item, and (2) manuals will be printed
with an abbreviated description plus a note to ``refer to'' the
appropriate section for a fuller description (e.g., both the ``server''
keyword and the ``device'' keyword descriptions in the manuals refer the
reader to the section describing TV devices).  

{\bf Format of the Keywords File}:  For each keyword in the file, the
first line begins with {\tt @ keyword}.  The next few lines begin with
{\tt >}, and provide the short description seen in the manuals.  The
remaining verbose lines are the full description seen in the on-line
documentation (where a reference to a section of a manual is deemed
inappropriate).

When writing the in-code documentation for a MIRIAD task, read the
descriptions in the keywords file (more keywords can be expected to be
included as time passes) to determine whether it is appropriate to use them
in the task being written:  if so, use it; if not, write up your own
description in the in-code documentation.

\beginsection MIRIAD Subroutines

By way of illustration, below is the in-code documentation for
MIRIAD subroutine axistype (\$MIR/src/subs/axistype.for),
which uses the ``directives'' noted previously.
{\ninepoint\begintt
c* Axistype - Find the axis label and plane value in user friendly units
c& mchw
c: plotting
c+
        subroutine AxisType(lIn,axis,plane,ctype,label,value,units)
c
        implicit none
        integer lIn,axis,plane
        character ctype*9,label*13,units*13
        double precision value
c
c Find the axis label and plane value in user friendly units.
c
c  Inputs:
c    lIn        The handle of the image.
c    axis       The image axis.
c    plane      The image plane along this axis.
c  Output:
c    ctype      The official ctype for the input axis.
c    label      A nice label for this axis.
c    value      The value at the plane along this axis.
c    units      User friendly units for this axis.
c--
\endtt}
Note that the programmer has woven executable code into the documentation
(the lines that are not commented out):  anything between the {\tt c+} and
the {\tt c--} is considered to be part of the documentation, even though
the lines are actually part of the subroutine code itself (this only
affects the way that {\bf doc} interprets in-code documentation, not the
way that the code is interpreted by the compiler).

A subroutine source code file (or a program source code file) may contain
multiple subroutines, each documented as above.

\beginsection Code History

All source code files should contain comments (near the start of the file)
describing the creation and modification history.  This is quite important
in the MIRIAD development environment, where programs are spread
across many computers and programmers are separated by many miles.

Below are typical history comments (taken from the ``key'' routines,
subroutine key.for):
{\ninepoint\begintt
c************************************************************************
c  The key routines provide keyword-oriented access to the command line.
c
c  History:
c    rjs    6jun87    Original version.
c    bs     7oct88    Converted it to use iargc and getarg. Added -f flag.
c    rjs    8sep89    Improved documentation.
c    nebk  10sep89    Added mkeyr.  I think rjs will not like it (Too right!).
c    rjs   19oct89    Major rewrite to handle @ files.
c    rjs   15nov89    Added keyf routine, and did the rework needed to support
c                     this. Added mkeyf. Modified mkeyr.
c    pjt   26mar90    Added mkeya. like mkeyr (again, bobs will not like this)
c    pjt   10apr90    some more verbose bug calls.
c    rjs   23apr90    Made pjt's last changes into standard FORTRAN (so the
c                     Cray will accept it).
c    pjt   10may90    Make it remember the programname in keyini (se key.h)
c                     for bug calls - introduced progname
c    rjs   22oct90    Check for buffer overflow in keyini.
c    pjt   21jan90    Added mkeyi, variable index is now idx, exp is expd
c************************************************************************
\endtt}

History comments are not optional.  While the format is not ironclad, it has
become a convention to have the word ``History:'', followed on subsequent
lines with (a) the programmer's initials, (b) the date of the modification,
and (c) one or two lines describing what was done.

\beginsection Other Internal Comments

These are comments that the programmer inserts as an aid in reading the
code.  For example:
{\ninepoint\begintt
c
c begin loop to find the optimal distance
c
    (actual code)
c
c finished loop to find the optimal distance.  variable x contains answer.
c
\endtt}
While these kinds of comments are optional, one is well-advised to include
them, both for those times when code has to be revised, and as an aid for
those who inherit responsibility for the code after the author has departed.

\beginsection Free Advice

{\bf No Editorials} - don't insert comments like ``what a hassle'', or
``this is copyrighted by me''.  Prior to distribution of the code out-of-BIMA,
someone has to go through and manually remove these things, then distribute
the modified code to the other MIRIAD sites.  In particular, don't
insert obscenities.

{\bf Brevity} - short is better.  Don't write a treatise.  A
short, weak explanation is far more likely to be read than a long-winded
technical dissertation, in which case the short, weak explanation is of more
value than the technically precise ``discussion''.  Users read MIRIAD
on-line docs only to figure something out, and then want to immediately get
back to their astronomy.  After writing a program, and having been close to
it for a month, one tends to lose perspective, making all facts about the
program seem of equal value, with each fact worthy of an explanation that
says ``here's a lot of information on the topic'', rather than a short
``how-to''.

{\bf Clarity} - for programs, make keyword explanations
clear, where ``clear'' means ``clear to the naive user'', not ``clear to the
programmer who wrote the code''.  Test it out on someone who's unfamiliar
with the system.  For subroutines, give enough information that other
MIRIAD programmers can use them without poring over the code to see
what you are doing, but not so much that reading the docs is an exercise in
its own right.

{\bf Referrals} - If at all possible, avoid referring the user
to a manual (e.g., the User's Guide).  In particular, avoid referring the
user to a manual that the user might not have.  If you feel that the item
requires a full explanation, give a short one anyway, and note that a fuller
explanation exists in the User's Guide under the appropriate section (if
that is indeed true).
