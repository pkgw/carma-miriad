\par\centerline{\bf Baselines}
{\eightpoint\begintt
blname ........ pjt .... Returns baseline name
bselect ....... mchw ... Find if given antenna is in given baseline code
findbase ...... pjt .... Returns the baseline number
findsrc ....... pjt .... Find index of word in a list of words
getants ....... pjt .... Returns the number of antennas and antenna pair
\endtt}
\par\centerline{\bf Calibration}
{\eightpoint\begintt
addpoly ....... pjt .... Add a polynomial to the database of polynomials
ampscal ....... pjt .... Scale amplitudes of a calibrator
basant ........ jm ..... Routine to determine antennas from baseline number.
caclose ....... pjt .... Close a calibration set
cadread ....... pjt .... Read data from a calibration set
\endtt}
{\eightpoint\begintt
cadwrite ...... pjt .... Write data into a calibration set
caerror ....... pjt .... Write a calibration I/O error
caflag ........ pjt .... Flag data in a calibration set
calget ........ jm ..... Routine to retrieve an interpolated calibrator flux.
caopen ........ pjt .... Open a calibration set
\endtt}
{\eightpoint\begintt
casread ....... pjt .... Read source index data
caswrite ...... pjt .... Write source index data
chkpoly ....... pjt .... Check if the current polynomial is set
code2s ........ pjt .... Convert ascii code to integer slot code
findbase ...... pjt .... Returns the baseline number
\endtt}
{\eightpoint\begintt
findsrc ....... pjt .... Find index of word in a list of words
fitpoly ....... pjt .... Fit a polynomial
flipper ....... pjt .... Flip phases into a continuous string
gainfact ...... mchw ... Determine the self-cal gain at a given time and baseline.
gainfin ....... mchw ... Close up the selfcal gains routines.
\endtt}
{\eightpoint\begintt
gaininit ...... mchw ... Initialise the "gains" routines.
getants ....... pjt .... Returns the number of antennas and antenna pair
getclo3 ....... pjt .... Return sign for closure calculation for 3 baselines
getpoly ....... pjt .... Read in the phase-amplitude calibration fit
getslot ....... pjt .... Get all slots that match wildcard slotcode
\endtt}
{\eightpoint\begintt
inipoly ....... pjt .... Forced initialize polynomials
lsqfill ....... pjt .... Add sums-of-squares to least squares matrix
miniflip ...... lgm .... Flip phases into a continuous string
phaseamp ...... pjt .... Convert to phase/amplitude
phasedis ...... pjt .... Find consistent antenna phases
\endtt}
{\eightpoint\begintt
phiwrap ....... pjt .... Returns the number of 2pi wraps
putpoly ....... pjt .... Write out the phase-amplitude calibration fit
putsrc ........ pjt .... Write out source stuff
rdhdia ........ pjt .... Read an integer array header variable
readbrk ....... pjt .... Read breakpoint data
\endtt}
{\eightpoint\begintt
readset ....... pjt .... Read in data from a calibration set
s2code ........ pjt .... Convert integer slot code to 4 character slot code
scalunit ...... pjt .... Set calibration unit conversion mode
setsflag ...... pjt .... Set flags along time axis according to selected sources
solve ......... pjt .... Solve a matrix
\endtt}
{\eightpoint\begintt
squares ....... pjt .... Compute sums-of-squares
tabflux ....... jm ..... Return the flux of a calibrator source at an input freq.
taver ......... pjt .... Scalar averaging an array
vectav ........ pjt .... Vector averaging with lookup index table
wrhdia ........ pjt .... Write an integer array header variable
\endtt}
{\eightpoint\begintt
writbrk ....... pjt .... Write break point data
writeset ...... pjt .... Write out data to a calibration set (timesorted)
writflag ...... pjt .... Write out flags to a calibration data set
\endtt}
\par\centerline{\bf Convolution}
{\eightpoint\begintt
convl ......... mjs .... Performs the convolution of the image with the beam.
convlini ...... mjs .... Initialize the convolution routines.
hannsm ........ jm ..... Hanning smooth a data array..
hcoeffs ....... jm ..... Calculated coefficients for Hanning smoothing.
\endtt}
\par\centerline{\bf Coordinates}
{\eightpoint\begintt
deghms ........ bpw .... Write out ra and dec in hms/dms from input in degrees
iscoords ...... bpw .... Test if input string can represent coordinates and convert it
radhms ........ bpw .... Write out ra and dec in hms/dms from input in radians
sfetra ........ bpw .... Transformation between equatorial and other coordinate systems
\endtt}
\par\centerline{\bf Display}
{\eightpoint\begintt
ctrlfin ....... jm ..... Close down the control panel.
ctrlinit ...... jm ..... Initialise the panel panel.
ctrlopen ...... jm ..... Open the panel panel.
pxtotv ........ jm ..... Convert image pixels to TV device positions.
tvchan ........ jm ..... Display a given image channel.
\endtt}
{\eightpoint\begintt
tvchar ........ jm ..... Get characteristics of the display device.
tvclose ....... jm ..... Close the display device.
tvcursor ...... jm ..... Read the location of the image display device's cursor.
tveras ........ jm ..... Erase a channel on the image display device.
tvflush ....... jm ..... Flush data to the display device.
\endtt}
{\eightpoint\begintt
tvline ........ jm ..... Write a raster line to the image display device.
tvlocal ....... jm ..... Put the display device into an interactive loop with the user.
tvlut ......... jm ..... Load a image display device's lookup table.
tvopen ........ jm ..... Open an image display device.
tvreset ....... jm ..... Reset the image display device.
\endtt}
{\eightpoint\begintt
tvscrl ........ jm ..... Scroll a display on a TV device.
tvselpt ....... jm ..... Interactive point/range selection on a display device.
tvtext ........ jm ..... Write a character string to the image display device.
tvtopx ........ jm ..... Convert TV device position to image pixels.
tvview ........ jm ..... Change the viewing window to a subportion of image memory.
\endtt}
{\eightpoint\begintt
tvwind ........ jm ..... Set/read current window (viewport) of a display device.
tvzoom ........ jm ..... Zoom a region of an image on a TV device.
wedge ......... jm ..... Load a wedge onto a TV device with user specified ranges.
\endtt}
\par\centerline{\bf Error-Handling}
{\eightpoint\begintt
assert ........ pjt .... Assert a condition, otherwise bug out
assertf ....... pjt .... Assert a file existence condition, otherwise bug out
asserti2 ...... pjt .... Assert a condition, otherwise bug out
bug ........... mjs .... Give an error message, and abort if needed.
bugno ......... mjs .... Give a error message associated with a particular error number.
\endtt}
\par\centerline{\bf Files}
{\eightpoint\begintt
filedel ....... pjt .... Delete a file
findname ...... jm ..... Return the expanded name of a file.
fullname ...... bpw .... Expand an environment variable in front of a filename
remext ........ bpw .... Remove the extension part of a filename
\endtt}
\par\centerline{\bf Fits}
{\eightpoint\begintt
fitcdio ....... rjs .... Sequential read or write of a FITS header.
fitrdhda ...... rjs .... Read a character value from a FITS file header.
fitrdhdd ...... rjs .... Read a double precision value from a FITS file header.
fitrdhdi ...... rjs .... Read an integer value from a FITS file header.
fitrdhdl ...... rjs .... Read a logical value from a FITS file header.
\endtt}
{\eightpoint\begintt
fitrdhdr ...... rjs .... Read a real value from a FITS file header.
fitsrch ....... rjs .... Search for a keyword in the header of an old FITS file.
fitwrhda ...... rjs .... Write a string to a FITS file header.
fitwrhdd ...... rjs .... Write a double precision keyword to a FITS file header.
fitwrhdh ...... rjs .... Write a string to a FITS file header.
\endtt}
{\eightpoint\begintt
fitwrhdi ...... rjs .... Write an integer value to a FITS file header.
fitwrhdl ...... rjs .... Write a logical value to a FITS file header.
fitwrhdr ...... rjs .... Write a real value to a FITS file header.
fuvclose ...... rjs .... Close a UV FITS file.
fuvopen ....... rjs .... Open a FITS uv file.
\endtt}
{\eightpoint\begintt
fuvrdhd ....... rjs .... Get coordinate information about a UV FITS file.
fuvread ....... rjs .... Read visibility data from UV FITS file.
fuvsetlm ...... rjs .... Set the ranges of parameters for a FITS UV file.
fuvtoff ....... rjs .... Get time offset for a UV FITS file.
fuvwrhd ....... rjs .... Save UV FITS file coordinate information.
\endtt}
{\eightpoint\begintt
fuvwrite ...... rjs .... Write data to a UV FITS file.
fxyclose ...... rjs .... Close a FITS image file.
fxyopen ....... rjs .... Open a FITS image file.
fxyread ....... rjs .... Read a row of data from a FITS image.
fxysetlm ...... rjs .... Set the pixel range for scaling a FITS image.
\endtt}
{\eightpoint\begintt
fxysetpl ...... rjs .... Select the plane of interest in a FITS image.
fxywrite ...... rjs .... Write a row of a FITS image.
\endtt}
\par\centerline{\bf Fourier-Transform}
{\eightpoint\begintt
convl ......... mjs .... Performs the convolution of the image with the beam.
convlini ...... mjs .... Initialize the convolution routines.
fftcc ......... pjt .... Complex to complex 1D FFT routine.
fftcr ......... pjt .... Complex to real 1D FFT routine.
fftrc ......... pjt .... Real to complex 1D FFT
\endtt}
\par\centerline{\bf Gridding}
{\eightpoint\begintt
corrfun ....... mchw ... Generate the gridding convolution correction function.
gcffun ........ mchw ... Generate the gridding convolution function.
intpini ....... mjs .... Initialize the interpolation i/o routines.
intprd ........ mjs .... Read a row of interpolated data
intprini ...... mjs .... Reinitialize the interpolation i/o routines.
\endtt}
\par\centerline{\bf Header-I/O}
{\eightpoint\begintt
haccess ....... mjs .... Open an item of a data set for access.
hclose ........ mjs .... Close a Miriad data set.
hdaccess ...... mjs .... Finish up access to an item.
hdcopy ........ mjs .... Copy a headfer variable from one data set to another.
hdelete ....... mjs .... Delete an item from a data-set.
\endtt}
{\eightpoint\begintt
hdprobe ....... mjs .... Determine characteristics of a header variable.
hdprsnt ....... mjs .... Determine if a header variable is present.
hisclose ...... mjs .... This closes the history file.
\endtt}
\par\centerline{\bf History}
{\eightpoint\begintt
addhist ....... pjt .... Add history to a dataset
hisappn ....... pjt .... Append history from a file to an open dataset
hisinput ...... jm ..... Copy task input parameters to a history file.
\endtt}
\par\centerline{\bf Image-Analysis}
{\eightpoint\begintt
title ......... mchw ... Write title in standard format into LogFile.
\endtt}
\par\centerline{\bf Image-I/O}
{\eightpoint\begintt
xyclose ....... mjs .... Close up an image file.
xyflgrd ....... mjs .... Read image masking information (flags format).
xyflgwr ....... mjs .... Write image masking information (flags format).
xymkrd ........ mjs .... Read the masking information for an image (runs format).
xymkwr ........ mjs .... Write image masking information (runs format).
\endtt}
{\eightpoint\begintt
xyopen ........ mjs .... Open an image file.
xyread ........ mjs .... Read a row from an image.
xysetpl ....... mjs .... Set which plane of a cube is to be accessed.
xywrite ....... mjs .... Write a row to an image.
\endtt}
\par\centerline{\bf Interpolation}
{\eightpoint\begintt
corrfun ....... mchw ... Generate the gridding convolution correction function.
gcffun ........ mchw ... Generate the gridding convolution function.
intpini ....... mjs .... Initialize the interpolation i/o routines.
intprd ........ mjs .... Read a row of interpolated data
intprini ...... mjs .... Reinitialize the interpolation i/o routines.
\endtt}
\par\centerline{\bf Least-Squares}
{\eightpoint\begintt
linlsq ........ bpw .... Return parameters of a straight line fit
llsqu ......... bpw .... Linear least squares fitting
lsqfill ....... pjt .... Add sums-of-squares to least squares matrix
nllsqu ........ bpw .... Nonlinear least squares fitting
solve ......... pjt .... Solve a matrix
\endtt}
\par\centerline{\bf Low-Level-I/O}
{\eightpoint\begintt
hisopen ....... mjs .... Open the history file.
hisread ....... mjs .... Read a line of text from the history file.
hiswrite ...... mjs .... Write a line of text to the history file.
hopen ......... mjs .... Open a data set.
hread ......... mjs .... Hwrite -- Read and write items.
\endtt}
{\eightpoint\begintt
hsize ......... mjs .... Determine the size (in bytes) of an item.
packi2 ........ mjs .... Pack normal integers into 16 bit integers
rdhda ......... mjs .... Read a string-valued header variable.
rdhdc ......... mjs .... Read a complex-valued header variable.
rdhdd ......... mjs .... Read a double precision-valued header variable.
\endtt}
{\eightpoint\begintt
rdhdi ......... mjs .... Read an integer-valued header variable.
rdhdr ......... mjs .... Read a real-valued header variable.
scrclose ...... mjs .... Close and delete a scratch file.
scropen ....... mjs .... Open a scratch file.
scrread ....... mjs .... Read real data from a scratch file.
\endtt}
{\eightpoint\begintt
scrwrite ...... mjs .... Write real data to the scratch file.
unpacki2 ...... mjs .... Unpack 16 bit integers into normal integers
wrhda ......... mjs .... Write a string-valued header variable.
wrhdc ......... mjs .... Write a complex-valued header variable.
wrhdd ......... mjs .... Write a double precision valued header variable.
\endtt}
{\eightpoint\begintt
wrhdi ......... mjs .... Write an integer valued header variable.
wrhdia ........ pjt .... Write an integer array header variable
wrhdr ......... mjs .... Write a real valued header variable.
\endtt}
\par\centerline{\bf Mathematics}
{\eightpoint\begintt
aricomp ....... pjt .... Parse a Fortran-like expression.
ariexec ....... pjt .... Evalute an expression, previously parsed by AriComp.
gaus .......... rjs .... Generate gaussianly distributed random variables.
j1xbyx ........ mchw ... Calculate j1(x)/x
nextpow2 ...... mchw ... Find the next power of two.
powell ........ bpw .... Minimization of a function, without derivative information.
prime ......... jm ..... Returns a prime less than or equal to N.
r8tyx ......... mchw ... Radix, 8 iterations
randset ....... rjs .... Set random number generator seed.
uniform ....... rjs .... Return uniformly distributed random variables.
\endtt}
\par\centerline{\bf Model}
{\eightpoint\begintt
model ......... mchw ... Calculate model visibilities, given a model image.
modelini ...... mchw ... Ready the uv data file for processing by the Model routine.
\endtt}
\par\centerline{\bf PGPLOT}
{\eightpoint\begintt
fatpoint ...... pjt .... Draw fat, visible points
pgerase ....... pjt .... Erase the screen - and nothing more
plotone ....... lgm .... Plot one or more set of x,y points on a single graph
win ........... pjt .... Subroutine package for interactive plots, using PGPLOT.
wincoord ...... pjt .... Change PGPLOT coordinates to a window
\endtt}
{\eightpoint\begintt
wincurs ....... pjt .... Get a character from a window
winloc ........ pjt .... Set window screen locations
winnear ....... pjt .... Return the nearest point
winnorm ....... pjt .... Normalize plots and add a margin
winnormy ...... pjt .... Normalize plots and add a margin
\endtt}
{\eightpoint\begintt
winpick ....... pjt .... Set the active windows
winpick1 ...... pjt .... Set an active window
winpoint ...... pjt .... Draw points
winqscal ...... pjt .... Queries set window user scales for selected window
winscale ...... pjt .... Set window user scales
\endtt}
{\eightpoint\begintt
winscalx ...... pjt .... Set window user X-scales
winscaly ...... pjt .... Set window user Y-scales
winset ........ pjt .... Set max unzoomed window matrix
winshow ....... pjt .... Put the plots onto the screen
winsize ....... pjt .... Size windows to the data
\endtt}
{\eightpoint\begintt
winsymb ....... pjt .... Return win2pgplot symbol marker
wintoscr ...... pjt .... Convert a point to screen coordinates
wintousr ...... pjt .... Convert a point to screen coordinates
\endtt}
\par\centerline{\bf Plotting}
{\eightpoint\begintt
axistype ...... mchw ... Find the axis label and plane value in user friendly units
fatpoint ...... pjt .... Draw fat, visible points
pgerase ....... pjt .... Erase the screen - and nothing more
pghline ....... mchw ... Histogram line plot for pgplot.
plotone ....... lgm .... Plot one or more set of x,y points on a single graph
\endtt}
{\eightpoint\begintt
win ........... pjt .... Subroutine package for interactive plots, using PGPLOT.
wincoord ...... pjt .... Change PGPLOT coordinates to a window
wincurs ....... pjt .... Get a character from a window
winloc ........ pjt .... Set window screen locations
winnear ....... pjt .... Return the nearest point
\endtt}
{\eightpoint\begintt
winnorm ....... pjt .... Normalize plots and add a margin
winnormy ...... pjt .... Normalize plots and add a margin
winpick ....... pjt .... Set the active windows
winpick1 ...... pjt .... Set an active window
winpoint ...... pjt .... Draw points
\endtt}
{\eightpoint\begintt
winqscal ...... pjt .... Queries set window user scales for selected window
winscale ...... pjt .... Set window user scales
winscalx ...... pjt .... Set window user X-scales
winscaly ...... pjt .... Set window user Y-scales
winset ........ pjt .... Set max unzoomed window matrix
\endtt}
{\eightpoint\begintt
winshow ....... pjt .... Put the plots onto the screen
winsize ....... pjt .... Size windows to the data
winsymb ....... pjt .... Return win2pgplot symbol marker
wintoscr ...... pjt .... Convert a point to screen coordinates
wintousr ...... pjt .... Convert a point to screen coordinates
\endtt}
\par\centerline{\bf Polynomials}
{\eightpoint\begintt
addpoly ....... pjt .... Add a polynomial to the database of polynomials
chkpoly ....... pjt .... Check if the current polynomial is set
evalpoly ...... pjt .... Evaluate a polynomial
fitpoly ....... pjt .... Fit a polynomial
getpoly ....... pjt .... Read in the phase-amplitude calibration fit
\endtt}
{\eightpoint\begintt
inipoly ....... pjt .... Forced initialize polynomials
lsqfill ....... pjt .... Add sums-of-squares to least squares matrix
putpoly ....... pjt .... Write out the phase-amplitude calibration fit
rpolyzr ....... bpw .... Roots of a real polynomial.
squares ....... pjt .... Compute sums-of-squares
\endtt}
\par\centerline{\bf Region-of-Interest}
{\eightpoint\begintt
boxes ......... mjs .... Summary of region of interest routines.
boxinfo ....... mjs .... Determine bounding box of the region of interest.
boxinput ...... mjs .... Read command line box specification.
boxmask ....... mjs .... AND in a mask to the region of interest.
boxrect ....... mjs .... Determine in a region-of-interest is rectangular.
\endtt}
{\eightpoint\begintt
boxruns ....... mjs .... Return region of interest in "runs" form.
boxset ........ mjs .... Set default region of interest.
getplane ...... mjs .... Read portion of a plane, specified by runs format.
putplane ...... mjs .... Writes a portion of a plane, specified by runs format.
putruns ....... mjs .... Writes the blanking file of an image.
\endtt}
\par\centerline{\bf SCILIB}
{\eightpoint\begintt
ismax ......... pjt .... Return index of maximum value of a real array.
ismin ......... pjt .... Return index of minimum value of a real array.
isrchfge ...... pjt .... Search real vector for target.
isrchfgt ...... pjt .... Search real vector for target.
isrchfle ...... pjt .... Search real vector for target.
\endtt}
{\eightpoint\begintt
isrchflt ...... pjt .... Search real vector for target.
isrchieq ...... pjt .... Search integer vector for target.
isrchige ...... pjt .... Search integer vector for target.
isrchigt ...... pjt .... Search integer vector for target.
isrchile ...... pjt .... Search integer vector for target.
\endtt}
{\eightpoint\begintt
isrchilt ...... pjt .... Search integer vector for target.
isrchine ...... pjt .... Search integer vector for target.
whenfeq ....... pjt .... Return locations equal to target.
whenfge ....... pjt .... Return locations greater than or equal to the target.
whenfgt ....... pjt .... Return locations greater than the target.
\endtt}
{\eightpoint\begintt
whenfle ....... pjt .... Return locations less than or equal to the target.
whenflt ....... pjt .... Return locations less than the target.
whenfne ....... pjt .... Return locations not equal to target.
whenige ....... pjt .... Return locations greater than or equal to the integer target.
whenigt ....... pjt .... Return locations greater than the integer target.
whenile ....... pjt .... Return locations less than or equal to the integer target.
whenilt ....... pjt .... Return locations less than the target.
\endtt}
\par\centerline{\bf Strings}
{\eightpoint\begintt
atod .......... bpw .... Convert a string into a double precision.
atoi .......... bpw .... Convert a string into an integer.
getfield ...... bpw .... Extract a field from a string.
getparm ....... bpw .... Extract a parameter from a string.
gettok ........ bpw .... Extract a token from a string.
\endtt}
{\eightpoint\begintt
indek ......... bpw .... Get position of substring in a string, returning length if not found
isalnum ....... bpw .... Return true if char is an alphanumeric character
isalpha ....... bpw .... Return true if char is a letter (a-z, A-Z)
isdigit ....... bpw .... Return true if char is 0, 1, 2, 3, 4, 5, 6, 7, 8 or 9
islower ....... bpw .... Return true if char is a lowercase letter (a-z)
\endtt}
{\eightpoint\begintt
isupper ....... bpw .... Return true if char is an uppercase letter (A-Z)
itoa .......... bpw .... Convert an integer into a string.
lcase ......... bpw .... Convert a string to lower case.
len1 .......... bpw .... Determine the unpadded length of a character string.
match ......... bpw .... Check if a string occurs in a list of valid strings
\endtt}
{\eightpoint\begintt
matchdcd ...... bpw .... Check if a string occurs in a list of valid strings, with extras
mitoa ......... bpw .... Convert many integers into a string.
nelc .......... bpw .... Return the length of the string
padleft ....... bpw .... Right justify a string to length characters.
rtfmt ......... bpw .... Construct a format during run time
\endtt}
{\eightpoint\begintt
rtoa .......... bpw .... Convert a real value into a string.
scanchar ...... bpw .... Scan a string for a character.
spanchar ...... bpw .... Skip over a particular character.
substr ........ bpw .... Returns the n-th substring from the input string
ucase ......... bpw .... Convert string to upper case.
\endtt}
\par\centerline{\bf Terminal-I/O}
{\eightpoint\begintt
delay ......... mjs .... Delay a specified length of time.
output ........ bpw .... Output a line of text to the user.
prompt ........ bpw .... Read input from the user
ucursor ....... mjs .... Check Ultra cursor pos and 'exit' button state.
udef .......... mjs .... Set up and display Ultra control panel.
\endtt}
{\eightpoint\begintt
ufin .......... mjs .... Finish Ultra frame buffer and control panel.
uinit ......... mjs .... Indicate to ultra where control panel host is.
ulocal ........ mjs .... Allow user to fiddle the ultra frame buffer.
\endtt}
\par\centerline{\bf Text-I/O}
{\eightpoint\begintt
linetype ...... bpw .... Read standard linetype keyword and transfer information to uvio
logclose ...... bpw .... Finish up with the log file.
logopen ....... bpw .... Initialise the log file routines.
logwrit ....... bpw .... Write a line to the log file.
logwrite ...... bpw .... Write a line to the log file.
\endtt}
{\eightpoint\begintt
txtclose ...... bpw .... Close a text file
txtopen ....... bpw .... Open a text file
txtopena ...... nebk ... Open a text file for appendation.
txtread ....... bpw .... Read a line from a text file
txtwrite ...... bpw .... Write a line to a text file
\endtt}
\par\centerline{\bf Transpose}
{\eightpoint\begintt
trnfin ........ mchw ... Close up the transpose routines.
trnini ........ mchw ... Initialise the transpose routines.
trnread ....... mchw ... Read back a plane of the reordered cube.
trnwrite ...... mchw ... Write a plane of the cube in its initial order.
\endtt}
\par\centerline{\bf TV}
{\eightpoint\begintt
pxtotv ........ jm ..... Convert image pixels to TV device positions.
tvchan ........ jm ..... Display a given image channel.
tvchar ........ jm ..... Get characteristics of the display device.
tvclose ....... jm ..... Close the display device.
tvcursor ...... jm ..... Read the location of the image display device's cursor.
\endtt}
{\eightpoint\begintt
tveras ........ jm ..... Erase a channel on the image display device.
tvflush ....... jm ..... Flush data to the display device.
tvline ........ jm ..... Write a raster line to the image display device.
tvlocal ....... jm ..... Put the display device into an interactive loop with the user.
tvlut ......... jm ..... Load a image display device's lookup table.
\endtt}
{\eightpoint\begintt
tvopen ........ jm ..... Open an image display device.
tvreset ....... jm ..... Reset the image display device.
tvscrl ........ jm ..... Scroll a display on a TV device.
tvselpt ....... jm ..... Interactive point/range selection on a display device.
tvtext ........ jm ..... Write a character string to the image display device.
\endtt}
{\eightpoint\begintt
tvtopx ........ jm ..... Convert TV device position to image pixels.
tvview ........ jm ..... Change the viewing window to a subportion of image memory.
tvwind ........ jm ..... Set/read current window (viewport) of a display device.
tvzoom ........ jm ..... Zoom a region of an image on a TV device.
wedge ......... jm ..... Load a wedge onto a TV device with user specified ranges.
\endtt}
{\eightpoint\begintt
wedge ......... jm ..... Load a wedge onto a TV device with user specified ranges.
\endtt}
\par\centerline{\bf User-Interaction}
{\eightpoint\begintt
ctrlchck ...... jm ..... Check if a particular buttons has been pushed.
ctrlclr ....... jm ..... Clear any memory in the control panel of buttons being pressed.
ctrldef ....... jm ..... Define a control panel button, etc.
ctrlfin ....... jm ..... Close down the control panel.
ctrlopen ...... jm ..... Open the panel panel.
\endtt}
{\eightpoint\begintt
ctrlset ....... jm ..... Set the value of a control panel button, etc.
ctrlview ...... jm ..... Pop up the control panel on the workstation screen.
ctrlwait ...... jm ..... Wait for a button to be pressed.
fatpoint ...... pjt .... Draw fat, visible points
keya .......... pjt .... Retrieve a character string from the command line.
\endtt}
{\eightpoint\begintt
keyd .......... pjt .... Retrieve a double precision from the command line.
keyf .......... pjt .... Retrieve a filename string (with wildcards) from the command line.
keyfin ........ pjt .... Finish access to the 'key' routines.
keyi .......... pjt .... Retrieve an integer from the command line.
keyini ........ pjt .... Initialise the `key' routines.
\endtt}
{\eightpoint\begintt
keyl .......... pjt .... Retrieve a logical value from the command line
keyprsnt ...... pjt .... Determine if a keyword is present on the command line.
keyr .......... pjt .... Retrieve a real value from the command line.
mkeya ......... pjt .... Retrieve multiple character values from the command line.
mkeyf ......... pjt .... Retrieve multiple filenames.
\endtt}
{\eightpoint\begintt
mkeyi ......... pjt .... Retrieve multiple integer values from the command line.
mkeyr ......... pjt .... Retrieve multiple real values from the command line.
options ....... bpw .... Get command line options.
pgerase ....... pjt .... Erase the screen - and nothing more
progname ...... pjt .... Return name of the program currently running
\endtt}
{\eightpoint\begintt
token ......... mchw ... Obtain token delimited by _ ()[]
win ........... pjt .... Subroutine package for interactive plots, using PGPLOT.
wincoord ...... pjt .... Change PGPLOT coordinates to a window
wincurs ....... pjt .... Get a character from a window
winloc ........ pjt .... Set window screen locations
\endtt}
{\eightpoint\begintt
winnear ....... pjt .... Return the nearest point
winnorm ....... pjt .... Normalize plots and add a margin
winnormy ...... pjt .... Normalize plots and add a margin
winpick ....... pjt .... Set the active windows
winpick1 ...... pjt .... Set an active window
\endtt}
{\eightpoint\begintt
winpoint ...... pjt .... Draw points
winqscal ...... pjt .... Queries set window user scales for selected window
winscale ...... pjt .... Set window user scales
winscalx ...... pjt .... Set window user X-scales
winscaly ...... pjt .... Set window user Y-scales
\endtt}
{\eightpoint\begintt
winset ........ pjt .... Set max unzoomed window matrix
winshow ....... pjt .... Put the plots onto the screen
winsize ....... pjt .... Size windows to the data
winsymb ....... pjt .... Return win2pgplot symbol marker
wintoscr ...... pjt .... Convert a point to screen coordinates
\endtt}
{\eightpoint\begintt
wintousr ...... pjt .... Convert a point to screen coordinates
\endtt}
\par\centerline{\bf Utilities}
{\eightpoint\begintt
angles ........ mchw ... Convert angle in degrees/hours to a string
arctan ........ bpw .... Arctangent function
basant ........ jm ..... Routine to determine antennas from baseline number.
boxnr ......... bpw .... Find in which box the input value lies.
bsrcha ........ mchw ... Search a list of strings for a particular string.
\endtt}
{\eightpoint\begintt
bsrchi ........ mchw ... Search a list of integers for a particular integer.
dangle ........ jm ..... Convert degrees/hours value into a formatted string.
dayjul ........ jm ..... Format a conventional calendar day into a Julian day.
findname ...... jm ..... Return the expanded name of a file.
freelun ....... mjs .... Release a logical unit, previously allocated with GetLun.
\endtt}
{\eightpoint\begintt
getbeam ....... mchw ... Get beam from image header.
getlun ........ mjs .... Get a logical unit number.
hangle ........ jm ..... Convert hours value (in radians) into a formatted string.
hsorta ........ jm ..... Perform an index heapsort on a character string array.
hsortad ....... jm ..... Perform a dual index heapsort on a character string array.
\endtt}
{\eightpoint\begintt
hsortar ....... jm ..... Perform a dual index heapsort on a character string array.
hsortd ........ jm ..... Perform an indexed heapsort on a double precision array.
hsorti ........ jm ..... Perform an indexed heapsort on a integer array.
hsortr ........ jm ..... Perform an indexed heapsort on a real array.
hsortrr ....... jm ..... Perform a dual index heapsort on a real array.
\endtt}
{\eightpoint\begintt
imminmax ...... jm ..... Return Miriad image minimum and maximum value.
imscale ....... jm ..... Autoscale a map.
itime ......... rjs .... Returns the present date.
julday ........ jm ..... Format a Julian day into a conventional calendar day.
linetype ...... bpw .... Read standard linetype keyword and transfer information to uvio
\endtt}
{\eightpoint\begintt
nel ........... bpw .... Return the number used values in an array
nelc .......... bpw .... Return the length of the string
nfig .......... bpw .... Find number of digits in a number
rangle ........ jm ..... Convert degrees value (in radians) into a formatted string.
rfac .......... jm ..... Calculate n factorial.
\endtt}
{\eightpoint\begintt
sortidxa ...... pjt .... Index sort of a an array of character values
sortidxd ...... pjt .... Index sort of a an array of double precision values
sortidxi ...... pjt .... Index sort of a an array of integer values
sortidxr ...... pjt .... Index sort of a an array of real values
teken ......... bpw .... Returns the sign of a real value
\endtt}
{\eightpoint\begintt
todayjul ...... jm ..... Format the current day into a Julian day.
\endtt}
\par\centerline{\bf uv-Data}
{\eightpoint\begintt
amphase ....... pjt .... Compute amplitude and phase.
basant ........ jm ..... Routine to determine antennas from baseline number.
bselect ....... mchw ... Find if given antenna is in given baseline code
expi .......... mchw ... Extract complex exponent of input in radians
oneamp ........ mchw ... Convert visibility to amp/phase or real/imag
\endtt}
{\eightpoint\begintt
pghline ....... mchw ... Histogram line plot for pgplot.
phase ......... mchw ... Extract phase in radians from complex number
shadowed ...... bpw .... (no description)
uvdatget ...... mchw ... Get information about the UVDat routines.
uvdatinp ...... mchw ... Get command line uv data parameters.
\endtt}
{\eightpoint\begintt
uvdatprb ...... mchw ... Determine what data have been selected in the uvdat routines.
uvdatrd ....... mchw ... Read uv data from a multi-file set.
uvfit1 ........ mchw ... Fit a constant to an "object" returned by uvinfo.
uvfit2 ........ mchw ... Fit a line to an "object" returned by uvinfo.
uvwdatrd ...... mchw ... Read wideband correlator data.
\endtt}
{\eightpoint\begintt
width ......... mchw ... Calculate wideband channel width
\endtt}
\par\centerline{\bf uv-I/O}
{\eightpoint\begintt
basant ........ jm ..... Routine to determine antennas from baseline number.
selapply ...... bpw .... Call the appropraite uv routine to set the uv selection.
selinput ...... bpw .... Get the users uv selection specification.
selprobe ...... bpw .... Check if a particular uv data has been selected.
uvclose ....... rjs .... Close a uv file
\endtt}
{\eightpoint\begintt
uvcopyvr ...... rjs .... Copy variables from one uv file to another.
uvdatget ...... mchw ... Get information about the UVDat routines.
uvdatget ...... mchw ... Get information about the UVDat routines.
uvdatinp ...... mchw ... Get command line uv data parameters.
uvdatinp ...... mchw ... Get command line uv data parameters.
\endtt}
{\eightpoint\begintt
uvdatprb ...... mchw ... Determine what data have been selected in the uvdat routines.
uvdatprb ...... mchw ... Determine what data have been selected in the uvdat routines.
uvdatrd ....... mchw ... Read uv data from a multi-file set.
uvdatrd ....... mchw ... Read uv data from a multi-file set.
uvflgwr ....... rjs .... Write uv flags after a read.
\endtt}
{\eightpoint\begintt
uvgetvr ....... rjs .... Get the values of a uv variable.
uvinfo ........ rjs .... Get information about the last data read with uvread.
uvnext ........ rjs .... Skip to the next uv record.
uvopen ........ rjs .... Open a uv data file.
uvprobvr ...... rjs .... Return information about a variable.
\endtt}
{\eightpoint\begintt
uvputvr ....... rjs .... Write the value of a uv variable.
uvrdvr ........ rjs .... Return the value of a UV variable.
uvread ........ rjs .... Read in some uv correlation data.
uvrewind ...... rjs .... Reset the uv data file to the start of the file.
uvscan ........ rjs .... Scan a uv file until a variable changes.
\endtt}
{\eightpoint\begintt
uvselect ...... rjs .... Select or reject uv data.
uvset ......... rjs .... Set up the uv linetype, and other massaging steps.
uvtrack ....... rjs .... Set flags and switches associated with a uv variable.
uvupdate ...... rjs .... Check whether any "important" variables have changed.
uvwdatrd ...... mchw ... Read wideband correlator data.
\endtt}
{\eightpoint\begintt
uvwread ....... rjs .... Read in the wideband uv correlation data.
uvwrite ....... rjs .... Write correlation data to a uv file.
uvwwrite ...... rjs .... Write wide-band correlation data to a uv file.
\endtt}
\par\centerline{\bf Zeeman}
{\eightpoint\begintt
zed ........... nebk ... Zeeman fit of I spectrum to V spectrum.
zedfudge ...... nebk ... Calculate sigma fudge factor, for Zeeman experiments.
zedfunc ....... nebk ... Return Zeeman maximum likelihood chi**2 value.
zedihat ....... nebk ... Calculate Zeeman estimate of a true I spectrum
zedrho ........ nebk ... Estimate Zeeman splitting, for spectrally correlated noise.
\endtt}
{\eightpoint\begintt
zedscale ...... nebk ... Determine conversion factor from channels to magnetic field.
zedvhat ....... nebk ... Calculate Zeeman estimate of a true V spectrum
\endtt}
\par\centerline{\bf Other}
{\eightpoint\begintt
binfid ........ nebk ... Adjust window size and fiddle bin sizes for spacial avg'ing
binrd2 ........ nebk ... For vxy images, apply specified binning criterion
binup ......... nebk ... Bin up data in array
getpb ......... pjt .... Determine the primary beam associated with a image.
\endtt}
