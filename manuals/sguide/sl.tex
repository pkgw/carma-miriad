%------------------------------------------------------------------------
% MIRIAD Subroutine Library Manual
%------------------------------------------------------------------------
%  History:
%            27jan91 mjs  Original version.
%            06feb91 bpw  Capture info from in-code docs.
%            15feb91 mjs  Mods.
%            17mar91 mjs  Changed "Miriad" to "MIRIAD"
%------------------------------------------------------------------------
\input sl_format
\pageno=1
%------------------------------------------------------------------------
% Chapter 1
%------------------------------------------------------------------------
\beginchapter{1}{HOW TO USE THIS MANUAL}

The {\bf MIRIAD Subroutine Libraries Manual} is a collection of facts
useful to MIRIAD programmers. It contains lists of subroutines
called by MIRIAD programs and documentation on those subroutines.
The sole purpose of this manual is to serve as a look-up reference book
of subroutines that are callable by MIRIAD programs.

\vskip 0.5in
\par It does not explain how to use subroutines, nor gives any reasoning
as to why they are used (see the {\bf MIRIAD Programmer's Guide} for that
information).

\vskip 0.5in
\par Chapters 3 to 6 comprise an index of the different subroutine
libraries used in MIRIAD: the MIRIAD subroutine library,
special MIRIAD subroutines in development, the PGPLOT library and
the LINPACK library. Chapters 7 and 8 give the full description of all
MIRIAD subroutines. PGPLOT and LINPACK subroutines are described
in their respective manuals.

\vskip 0.5in
\par There is no information in this manual that is intended to be useful
to MIRIAD users (see the {\bf MIRIAD User's Guide} or the {\bf MIRIAD
Cookbook} for that information).

\endchapter
%------------------------------------------------------------------------
% Chapter 2
%------------------------------------------------------------------------
\beginchapter{2}{GENERAL INFORMATION}

\beginsection MIRIAD Programmers
\par The ``Pgmr'' field (the programmer's initials) is the means by which
programmers identify themselves within MIRIAD source code. This
is typically used in comment statements within the source code to
identify subroutine ownership or responsibility, and to specify who
made changes to the code.
{\eightpoint\begintt
Pgmr     Name                      INTERNET               phone
----  --------------    ------------------------------ ------------
bpw   Bart Wakker       wakker@sirius.astro.uiuc.edu   217-244-4207
jm    Jim Morgan        morgan@astro.umd.edu           301-405-6853
lgm   Lee Mundy         lgm@astro.umd.edu              301-405-1529
mjs   Mark Stupar       mstupar@sirius.astro.uiuc.edu  217-244-4208
mchw  Mel Wright        wright@bkyast.berkeley.edu     415-642-0420
nebk  Neil Killeen      nkilleen@rpepping.oz.au
pjt   Peter Teuben      teuben@astro.umd.edu           301-405-1540
rjs   Robert Sault      rsault@rpepping.oz.au
wh    Wilson Hoffman    hoffman@bkyast.berkeley.edu    415-642-7768
hr    Harold Ravlin     hr@sirius.astro.uiuc.edu       217-244-4208
rmc   Dick Crutcher     crutcher@sirius.astro.uiuc.edu 217-333-9581
----  --------------    ------------------------------ ------------
\endtt}

\beginsection Subroutine Categories Description
\par This section lists the functional categories of MIRIAD
subroutines. All MIRIAD subroutines are placed into one of the
categories below.  For descriptions, see the section on documentation
formats in the {\bf MIRIAD Programmer's Guide}.
{\eightpoint\begintt
------------------  ------------------  ------------------  ------------------
Baselines           Calibration         Convolution         Coordinates
Display             Error-Handling      Files               Fits
Fourier-Transform   Gridding            Header-I/O          History
Image-Analysis      Image-I/O           Interpolation       Least-Squares
Log-File            Low-Level-I/O       Mathematics         Model
PGPLOT              Plotting            Polynomials         Region-of-Interest
SCILIB              Sorting             Strings             Terminal-I/O
Text-I/O            Transpose           TV                  User-Input
User-Interaction    Utilities           uv-Data             uv-I/O
Zeeman              Other
------------------  ------------------  ------------------  ------------------
\endtt}

\beginsection List of MIRIAD Manuals
\par Below is a list of all MIRIAD manuals. \par
\medskip\par{\parindent=0.5cm
MIRIAD Cookbook                    \par
MIRIAD User's Guide                \par
MIRIAD Programmer's Guide          \par
MIRIAD Subroutine Libraries Manual \par
MIRIAD Executable Modules Manual   \par
MIRIAD PGPLOT Manual               \par
MIRIAD LINPACK Manual              \par
}

\endchapter
%------------------------------------------------------------------------
% Chapter 3
%------------------------------------------------------------------------
\beginchapter{3}{LIST OF MIRIAD LIBRARY SUBROUTINES}

This chapter lists MIRIAD subroutines compiled into MIRIAD's
own library (libmir.a). All source code is in FORTRAN (in which case it
has file extension ``.for'' and is processed by MIRIAD's RATTY
preprocessor) or C. Refer to Chapter 7 of this manual for descriptions
and specification of the calling sequence.

\beginsection Sorted by Functional Category
\par\input sl_ch3f

\beginsection Sorted by Source Code of Origin
\par\input sl_ch3s

\beginsection Sorted Alphabetically
\par\input sl_ch3a

\beginsection Sorted by Responsible Programmer
\par\input sl_ch3p

\endchapter
%------------------------------------------------------------------------
% Chapter 4
%------------------------------------------------------------------------
\beginchapter{4}{LIST OF PGPLOT LIBRARY SUBROUTINES}

This chapter lists PGPLOT subroutines compiled into MIRIAD's
own PGPLOT library (libpgplot.a).  All source code is in FORTRAN,
and is not processed by MIRIAD's RATTY preprocessor.  Refer to
the PGPLOT manual for a fuller description of the subroutines, and
for a specification of the calling sequence.

\beginsection Modifications to PGPLOT

Several modifications to PGPLOT have been made by MIRIAD
programmers.  All are minor, and they are listed below.
\vskip 0.25in

Routines {\bf cgidefs.h} and {\bf cgiconstants.h} have been added, and
references to system-owned CGI includes have been changed to references
to local includes.  CGI can no longer be assumed to exist on a Sun system,
and PGPLOT's CGI driver {\bf cfopenvws.c} requires these two include files
for programs to be loaded.  As noted in the code, copying these two include
files into PGPLOT does not violate Sun copyrights.
\vskip 0.25in

PGPLOT routine {\bf cfopenvws.c} now requires local CGI include files rather
than system-owned CGI include files.
\vskip 0.25in

SunView driver {\bf svdriv.c} now accepts any delimiter when a user
specifies window size, rather than only a comma (which was intercepted by
MIRIAD and interpreted as a keyword's parameter delimiter).  This
modification has been accepted by Cal Tech to be a part of future PGPLOT
releases.
\vskip 0.25in

Driver controller {\bf noXgrexecSUN.f} has been added.  It is the same as
PGPLOT routine {\bf grexecSUN.f}, except that there is no X-driver called
(for use by installations without X).
\vskip 0.25in

The contents of the manual directory have been moved to MIRIAD's own
manuals directory.
\vskip 0.25in

\beginsection Sorted by Functional Category
\par\input sl_ch4f

\beginsection Sorted Alphabetically
\par\input sl_ch4a

\par\endchapter
%------------------------------------------------------------------------
% Chapter 5
%------------------------------------------------------------------------
\beginchapter{5}{LIST OF LINPACK LIBRARY SUBROUTINES}

\par\input sl_ch5

\par\endchapter
%------------------------------------------------------------------------
% Chapter 6
%------------------------------------------------------------------------
\beginchapter{6}{LIST OF MIRIAD TEST SUBROUTINES}

This chapter lists MIRIAD subroutines that are compiled and loaded
into main programs only (and thus, are not generally available without
copying the code directly from the source program). These subroutines
are listed here and documented in Chapter 8 so that MIRIAD
programmers are aware of their existence. If any are deemed to be of
``general utility'', then it can be removed from its main program and
compiled into the MIRIAD subroutine library (libmir.a).
\par All source code is in FORTRAN (in which case it has file extension
``.for'' and is processed by MIRIAD's RATTY preprocessor) or C.
Refer to Chapter 8 of this manual for a fuller description of the
subroutines, and for a specification of the calling sequence.

\beginsection Sorted by Functional Category
\par\input sl_ch6f

\beginsection Sorted by Source Code of Origin
\par\input sl_ch6s

\beginsection Sorted Alphabetically
\par\input sl_ch6a

\beginsection Sorted by Responsible Programmer
\par\input sl_ch6a

\endchapter
%------------------------------------------------------------------------
% Chapter 7
%------------------------------------------------------------------------
\beginchapter{7}{MIRIAD SUBROUTINE DOCUMENTATION}

\par\input sl_ch7subs

\par\endchapter
%------------------------------------------------------------------------
% Chapter 8
%------------------------------------------------------------------------
\beginchapter{8}{MIRIAD TEST SUBROUTINE DOCUMENTATION}

\par\input sl_ch8test

\par\endchapter
%------------------------------------------------------------------------
% TOC
%------------------------------------------------------------------------
\closeout\toc

\bgroup
\pageno=-1
\footline={\hss}
\headline={\hss}
\input sl_title

\footline={\hss\tenrm\folio\hss}
\centerline{\bit Contents}
\vskip 0.5in
\input toc
\endchapter

\end
